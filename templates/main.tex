\documentclass[12pt,oneside]{book}
%\usepackage{slovak}
\usepackage[utf8x]{inputenc}
%\usepackage{url}
\usepackage{amsfonts , amsmath, amsfonts, amssymb}
\usepackage{a4wide}
%\usepackage{alltt}
%\usepackage{epsfig}
\usepackage{graphicx} 
\usepackage[usenames,dvipsnames]{color}
\usepackage{enumerate}
\usepackage{subfigure}
\usepackage{listings}
\usepackage{ifthen}
\lstset{language=python, numbers=left, frame=shadowbox, tabsize=2}

\renewcommand\baselinestretch{1.10} %riadkovanie jeden a pol

\usepackage{amsmath}

\input ../tex/macro.tex

\input ../tex/config.tex


\ifx\pdfoutput\undefined\relax\else\pdfinfo{ /Title (\mftitle) /Author (\mfauthor) /Creator (PDFLaTeX) } \fi
\begin{document}

\frontmatter

\input ../tex/frontMatter.tex

\mainmatter

\input mainMatter.tex

\backmatter

\input ../tex/backMatter.tex

\nocite{*}
\bibliographystyle{alpha}
\bibliography{reference}

\chapter*{Abstrakt}
\begin{tabular}{l l}
Autor: & Michal Nánási \\
Názov práce: & Biological sequence annotation with hidden Markov models\\
Univerzita: & Univerzita komenského v Bratislave \\
Fakulta: & Fakulta matematiky, fyziky a informatiky\\
Katedra: & Katedra informatiky \\
Vedúci diplomovej práce: & Mgr. Broňa Brejová PhD.\\
Rozsah práce: & \pageref{LastPage} \\
Rok: & 2010
\end{tabular}

\bigskip


Skryté Markovove modely (HMM) sú dôležitým nástrojom na modelovanie
biologických sekvencii a ich anotácii. Anotovaním sekvencie myslíme priradenie
popisov jednotlivým symbolom sekvencie podľa ich významu. Napríklad ak hľadáme
gény, tak sa snažíme rozdeliť DNA sekvenciu na časti, ktoré kódujú proteíny
(gény) a na tie, ktoré génmi nie sú.  Skryté Markovove modely definujú
pravdepodobnostnú distribúciu sekvencii a ich anotácii.

Dekódovanie zo skrytých Markovovych modelov je obyčajne realizované pomocou
Viterbiho algoritmu. Viterbiho algoritmus nájde najpravdepodobnejšiu anotáciu
len pre podtriedu všetkých skrytých Markových modelov.  Vo všeobecnosti je anotácia
sekvencii NP-ťažká a preto Viterbiho algoritmus môžeme použiť len ako
heuristickú metódu.

V posledných rokoch sa ukázalo, že v určitých aplikáciach iné dekódovacie
metódy majú lepšie výsledky ako Viterbiho algoritmus. V tejto práci predkladáme
novú dekódovaciu metódu, ktorá berie do úvahy neurčitosť v presnej polohe hraníc
jednotlivých regiónov. Naša metóda považuje anotácie, ktoré sa len trochu líšia
v hraniciach regiónov za rovnaké. Nazvali sme ju dekódovanie s najvyššou
predpokladanou odmenou (the highest expected reward decoding -- HERD) a je
založená na dekódovaní pomocou maximalizácie strednej hodnoty presnosti hraníc
(maximum expected boundary accuracy decoding) \cite{Gross2007}.

Náš algoritmus sme testovali na probléme detekcie rekombinácií v genóme vírusu
HIV a porovnali sme ho s existujúcim nástrojom, ktorý sa volá preskakujúce HMM
(jumping HMM). HERD má lepšiu presnosť predikcie v rámci špecifickosti a
senzitívnosti správne predikovaných jednofarebných regiónov.

\medskip
\noindent
{\sc Kľúčové slová}: Skryté Markovove modely, anotácia sekvencií, rekombinácie vírusov

\end{document}
