\chapter{Alignment with repeats}

\section{Repeats}
\begin{reformulate*}
Tu chcem popisat co su to repeaty, ake problemy mozu robit v zarovnani (nejake
priklady) a podobne.
\end{reformulate*}
\section{Finding repeats}
\begin{reformulate*}
Popiseme ako sa daju hladat repeaty, napriklad aj nasim jednoduchym modelom,
alebo napriklad aj tandem repeat finderom.
\end{reformulate*}
\section{Other software}
\begin{reformulate*}
Chceme tu popisat tantan a este ako sa standardne riesia repeaty -- maskovanim
\end{reformulate*}
\section{Our models}
\begin{reformulate*}
Popiseme nase oba modely. sunflower, a ten tantan-like
\end{reformulate*}
\section{Decoding methods}
\begin{reformulate*}
Tu popiseme ake dekodovacie metody pozname: viterbi, posterior, block \{viterbi,
posterior\}
\end{reformulate*}
\section{Implementation details \& optimizations}
\begin{reformulate*}
Implementacne detaily, optimalizacie, cachovanie
\end{reformulate*}
\section{Data}
\begin{reformulate*}
Ako sa generovali data a na com sa vlastne testovalo.
\end{reformulate*}
\section{Results}
\begin{reformulate*}
Presny popis kazdeho experimentu, tabulky, bla bla bla bla
\end{reformulate*}

\label{LastPage}
