We study two important problems in computational biology:  sequence
annotation and  sequence alignment. In the thesis we concentrate on the use of hidden Markov
models (HMMs), well established generative probabilistic models.  

In the first part, we study the sequence annotation problem, specifically the
two-stage HMM decoding algorithms and the computational complexity of related problems. In
particular, we demonstrate that two-stage algorithms can be used to increase
the accuracy of decoding, and we prove the NP-hardness for three
problems appropriate for the first stage: the most probable set problem, the most probable restriction problem
and the most probable footprint problem.

The second part of the thesis focuses on  alignment of sequences that contain
tandem repeats. Tandem repeats are highly repetitive elements withing genomic
sequences that cause biases in alignments. To address this issue, we introduce
a new HMM that models alignments containing tandem repeats, combine it with
existing and new decoding algorithms, and evaluate our approach experimentally.

In both problems, we use the decoding algorithms to improve the accuracy of
HMM predictions. Decoding algorithms are often neglected, and most of
the development is focused on the structure of an HMM. However, a proper
selection of a decoding method can lead to significant improvements in the
predictions.
