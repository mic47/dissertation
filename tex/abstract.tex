We study two important problems in computational biology: the sequence
annotation and the sequence alignment. In the theses we use the hidden Markov
models (HMMs), a well established generative probabilistic models.  

In the first part we study the sequence annotation problem, specifically the
two-stage algorithms and the computational complexity of related problems. In
particular, we demonstrate that two-stage algorithms can be used to increase
accuracy of decoding algorithms, and we prove the NP-hardness for three related
problems: the most probable set problem, the most probable restriction problem
and the most probable footprint problem.

Second part of the thesis focus on the alignment of sequences that contain
tandem repeats. Tandem repeats are highly repetitive elements withing genomic
sequences that causes biases in alignments. To address this issue, we introduce
new HMM that models alignment containing tandem repeats, combine it with
existing and new decoding algorithms, and evaluate our approach experimentally.

In both problems, we use the decoding algorithms to improve accuracy of
predictions from HMMs. Decoding algorithms are often underestimated and most of
the development is focused on the structure of an HMM. However, proper
selection of an decoding method can lead to significant improvements in the
predictions.
