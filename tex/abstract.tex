In this thesis we study the hidden Markov models (HMMs) and their applications
to two important computations biology problems: the sequence annotation and the
sequence alignment. The large part of this thesis consider the perspective of
the decoding algorithm. In the sequence annotation problem, we studied special
class of decoding algorithms, which is a technique that used to be used to
improve the time complexity of decoding algorithms. We have demonstrated that
this technique can be used to improve the accuracy of the predictions of the
decoding algorithm. We study the computational complexity of such problems and
prove that three problems are NP-hard or NP-complete: the most probable set
problem, the most probable restriction problem and the most probable footprint
problem.

Then? we studied the sequence alignment problem. We specialize on the problem
of tandem repeats, which are hard to align and causes biases in the alignments
which spreads further into the methods that use these alignments. We study the most probable 
state method, that
Nieco o anotaciach a teoretickych vysledkoch

Nieco o zarovnaniach a teoretickych vysledkoch
