\chapter{Alignment}
\begin{note}
Prerekvizity: co je to sekvencia DNA, pripadne sekvencia proteinu
\end{note}
\section{Definitions}

Parts of two sequences are homologous, if they have evolved from same part of
their ancestral sequence. Aim of sequence alignment is to identify homologous
sequences. There are several evolution events, that usually happens:
mutation of residue into another residue, deletion of part of the sequence,
insertion of residues into the sequence. We omit rearrangements of sequence,
because it cannot be represented by alignment. Alignment is one way of
representing of comparisons of two sequences. We obtain alignment of $k$
sequences by inserting dashes into sequence in a way, that they will have same
length. Therefore we can represent alignment as matrix or table. Row of the
alignment is a sequence with inserted dashes and column is list of residues from
all row on certain position.

Alignments have biological meaning. We try to create alignment in a way, that
homologous residues are in same row. Dashes represents either the parts of
sequence that were deleted during evolution (deletions) or that on those
positions were inserted some sequences to other sequence (insertions). If we
have alignment of sequences from current organisms then without some prior
knowledge it is not possible to distinguish between insertions and deletions.
Therefore we will refer to those as to indels.

\begin{example} 
Consider following evolution history of hypothetical DNA sequences of two current organism $X$ and $Y$:
\begin{verbatim}
X:      C C     G C G A C C T T G C             A C C A
X':     C C     G           T T G C             A G C A
P:      A C T G G           T C G C T G A G C T A G C A
Y':     T C T G G           C C           G C T A G C A
Y:      T C T A G           C C           G A T A G C A
\end{verbatim}
$X$ and $Y$ evolved from parent sequence $P$ through sequences $X'$ and $Y'$.
During evolution from $P$ to $X'$, four events occurred: deletion of 
sequences ``TG''  and ``TGAGCT'' and mutations of two bases. Evolution of $X$
was ended by  
positions bases were substituted. In change from $X'$ to $X$, one base was
changes and siequence ``CGACC'' was inserted.  Similarly, from 
REmove this paragraph

\begin{verbatim}
X:      C C     G C G A C C T T G C             A C C A
Y:      T C T A G           C C           G A T A G C A
\end{verbatim}
We obtain actual alignment by removing columns that contains only gaps a
replacing gaps with dashes (gap symbols). 
\begin{verbatim}
X:      C C - - G C G A C C T T G C - - - A C C A
Y:      T C T A G - - - - - C C - - G A T A G C A
\end{verbatim}
If two residues are in same column in alignment then they are homologous.
Being homologous does not tell that those symbols are equal. 
\end{example}

There is only one alignment that reflects evolution history. Our goal is to find
this alignment, or at least find the alignment, that is as close as possible.
\correction{We
will consider only pairwise alignments (alignments of two sequences).}{Urvat ak
zacneme robit multiple}

\section{Scoring Schemes}

Since we want to construct alignments that have biological meaning, we have to
develop a way, how to assess the quality of an alignment. On way of doing it is
to define scoring scheme. There are many ways how to develop good scoring
scheme. We will assess 

\section{Needleman-Wunsch algorithm}

Alignment with scoring scheme with previous section can be found by
Needleman-Wunsch algorithm. 
%TODO: nieco o tom ze to bol povodne kubicky, ale sankoff spravil lepsiu verziu

This algorithm uses score table $S$, affine gap model with gap penalty $d$ and
gap opening penalty $g$ is zero. To align sequences $X$ and $Y$ with length $n$
and $m$ respectively, we define matrix $M$ of size $n\times m$. $M[i,j]$ will be
the score of the best alignment of sequences $X[:i]$ and $Y[:j]$. We can compute
$M[i,j]$ by following equations:

\begin{align}
M[0,0] &= S(X_0,Y_0)\\
M[0,i] &= i\cdot d, 0< i < m\\
M[i,0] &= i\cdot d, 0< i < n\\
M[i,j] &= \max
\begin{cases}
 M[i-1,j-1]+S(X_i,X_j)\\M[i,j-1]+d\\
 M[i-1,j]+d
\end{cases}, 0<i<n,0<j<m
\end{align}

By computing $M[n-1,m-1]$ we have the score of the alignment of $X$ and $Y$ with
the highest score. We can reconstruct such alignment by back-tracing matrix $M$.
Matrix 

To cope with gap opening penalty, we have to slightly change the algorithm.
We define two other matrices $M_X$ and $M_Y$ of same size as $M$. $M_X[i,j]$
will contain the score of alignment of sequences $X[:i]$ and $Y[:j]$ than ends
with gap in sequence $X$. $M_Y$ is analogous. Now we show how to compute $M,M_X$
and $M_Y$. \todo{Skontroluj to}

\begin{align}
M[0,0] &= S(X_0,Y_0)\\
M[0,i] &= M_X[0,i] = i\cdot d+g, 0 < i < m\\
M[i,0] &= M_Y[i,0] = i\cdot d+g, 0 < i < n\\
M[i,j] &= \max
\begin{cases}
 M[i-1,j-1]+S(X_i,X_j)\\
 M[i,j-1]+d\\
 M[i-1,j]+d\\
 M_X[i,j]\\
 M_Y[i,j]
\end{cases}, 0<i<n,0<j<m\\
M_X[i,0] &= -\infty, 0\leq i< n\\
M_Y[0,i] &= \infty, 0 \leq i< m\\
M_X[i,j] &= \max
\begin{cases}
M[i-1,j]+g+d\\
M_X[i-1,j]+d
\end{cases}, 0<i<n,0<j<m\\
M_Y[i,j] &= \max
\begin{cases}
M[i,j-1]+g+d\\
M_Y[i,j-1]+d
\end{cases}, 0<i<n,0<j<m
\end{align}
\todo{Je vlastne vobec treba taketo vypisovat?}
As in previous case, values of $M,M_X$ and $M_Y$ can be computed by simple
dynamic and by back-tracing these matrices we can obtain optimal alignment of
$X$ and $Y$.
Both algorithms have time complexity of $O(nm)$ and memory complexity $O(mn)$.
Using tricks described in later section, the memory and even time comple
\section{Gap Models} 
\section{Algorithmic Improvements}
