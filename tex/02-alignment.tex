\chapter{Alignments}
\label{CHAPTER:ALIGNMENT}
In this chapter we define alignments, define basic scoring scheme for alignments
and describe basic algorithms that compute biologically relevant alignments of
two sequences. Ten we describe several algorithmic improvements and heuristics
that reduce the computational complexity of the sequence alignment algorithms.
In the end we will discuss the possibilities of the extension of the algorithm
to different gap models.

Parts of two sequences are homologous, if they have evolved from same sequence
in their common ancestor. Aim of sequence alignment is to identify homologous
sequences. Sequences can be modified by different evolution events:
mutation of a residue into another residue, deletion of a part of the sequence,
insertion of residues into the sequence. There are also large-scale
rearrangement events like duplications (some subsequence is duplicated and
copied into other part of the sequence), inversions (some subsequence is
reversed) or relocations in which part of the sequence change position.  We will
ignore them for now because they cannot be represented by traditional
alignments. An alignment is a data structure that represents comparisons of two
or more sequences. We obtain an alignment of $k$ sequences by inserting dashes
into individual sequences so that they have the same length. We can represent an
alignment as a matrix or a table. Each row of the alignment is a sequence with
inserted dashes, and each column is list of residues from all rows at the same
position.


An alignment has the following biological meaning: homologous residues (those that
have evolved from a common ancestor) are in same column. Dashes represents
either the parts of sequence that were deleted during evolution (deletions) or
positions where some residues were inserted into some other sequence
(insertions). In an alignment is not possible to distinguish between insertions and
deletions -- we cannot tell if something was inserted into one sequence or if
there was deletion in the other sequences. Therefore we will refer to insertions
and deletions as to \firstUseOf{indels}.

\begin{example} 
Consider the  following evolution history of hypothetical DNA sequences of two
current organisms $X$ and $Y$. There was a parent sequence $P$ which evolved into
$X'$ and $Y'$ and after that $X'$ evolved into $X$ and $Y'$ evolves into $Y$.
These sequences are shown bellow: 
\begin{verbatim}
X:      C C     G C G A C C T T G C             A C C A
X':     C C     G           T T G C             A G C A
P:      A C T G G           T C G C T G A G C T A G C A
Y':     T C T G G           C C           G C T A G C A
Y:      T C T A G           C C           G A T A G C A
\end{verbatim}
%$X$ and $Y$ evolved from parent sequence $P$ through sequences $X'$ and $Y'$.
During evolution from $P$ to $X'$, four events occurred: deletion of 
sequences ``TG''  and ``TGAGCT'' and mutations of two bases. During evolution
from $X'$ to $X$, one base was changed and sequence ``CGACC'' was inserted.  
Similarly during evolution from $P$ to $Y'$, the two bases mutated and two
sequences were deleted. During evolution from $Y'$ to $Y$ one base mutated and
sequence ``CGACC'' was inserted. 

From evolution history described above, we can create an alignment of current
sequences $X$ and $Y$ by removing ancestral sequences $X',Y'$ and $P$, 
%\begin{verbatim}
%X:      C C     G C G A C C T T G C             A C C A
%Y:      T C T A G           C C           G A T A G C A
%\end{verbatim}
%We obtain actual alignment by 
removing columns that contain only gaps and
replacing gaps with dashes (gap symbols). 
\begin{verbatim}
X:      C C - - G C G A C C T T G C - - - A C C A
Y:      T C T A G - - - - - C C - - G A T A G C A
\end{verbatim}
In this alignment symbols that are in same column are truly homologous (they
evolved from same symbol in $P$).
%If two residues are in same column in alignment then they are homologous.
As you can see, homologous symbols do not have to be equal.
\end{example}

There is only one alignment that reflects the true evolution history. Our goal
is to find this alignment, or at least some alignment, that is as close as
possible. The alignment shown above is a \firstUseOf{global alignment} because it
is alignment of whole sequences $X$ and $Y$. A local alignment is alignment of
parts of sequences: local alignment of sequences $X$ and $Y$ is a global alignment
of strings $\bar{X}$ and $\bar{Y}$ where $\bar{X}$ is a substring of $X$ and
$\bar{Y}$ is a substring of $Y$.  Since global alignments do not consider
rearrangement events\footnote{Duplication, reversal or
translocation.}, local alignments are useful in such scenarios.  We will mostly
consider global alignments, but most of the methods can be extended also to local
alignments.

In this chapter we will review basic methods for constructing alignments. We will
discuss basic scoring schemes and algorithms that finds optimal alignment under
such scheme.

%At first
%we will discuss pair alignments (alignment of two sequences).
%\correction{Later in this
%chapter we will review some methods how to construct multiple alignments.}{urvat
%ak to nie je pravda}

\section{Scoring Schemes}

Since we want to construct alignments that have biological meaning, we have to
develop a method for assessing the quality of an alignment. One way of doing so
is to define a scoring scheme. Scoring scheme is a method that assigns to every
alignment a real number (called score). The better alignments should have higher
score than the worse. Once we have scoring scheme we will search for an
alignments of the input sequences with the highest score.  There are many ways
how to develop good scoring scheme. 


Scoring schemes covered in this chapter will independently score each column of
an alignment, which does not contain a gap. Gaps will be scored by a penalty
that depends on the length of the gap (the number of consecutive dashes). Score
of an entire alignment is the sum of the scores of all ungapped columns plus the
sum of the scores of all gaps.

In particular
we will assume that all sequences are from a finite alphabet $\Sigma$. For DNA,
$\Sigma=\{A,C,G,T\}$, for proteins $\Sigma$ contains $20$ codes of amino acids.
We will score a column containing residues $a$ and $b$ by $S[a,b]$ where $S$ is
a matrix of size $|\Sigma|\times|\Sigma|$ called \firstUseOf{substitution
matrix}.  A gap of length $x$ has a score $g+dx$, where $g$ is the gap opening
penalty and $d$ is the gap extension penalty. Both are usually negative since we
want alignments that contains many columns with same or similar symbols. With
positive gap penalty there will be tendency towards alignments with many gaps
which reduce the number of aligned residues.  We call this gap scoring scheme an
affine gap model. In general, gaps can be penalized by arbitrary function $f(x)$
(non-affine gap penalties will be discussed in the end of this chapter).

Intuition behind this scoring system is probabilistic. Assume that we have
alignment of $X$ and $Y$ without gaps and that each base in DNA evolves
independently. Therefore for every pair of residues $a,b$ there is probability
that $a$ will evolve into $b$. We denote this probability as $p(a,b)$. Therefore the probability that $X$ will evolve
into $Y$ is product of probabilities that $X_i$ will evolve into $Y_i$
\[
\prob{X\text{ evolved into }Y} = \prod_{i=0}^{|X|-1}p(X_i,Y_i)
\]
If $X$ and $Y$ were generated independently be some random model $R$, then probability that we see $X_i$
and $Y_i$ is $\prob{X_i\mid R}\prob{Y_i\mid R}$ ($R$ usually samples symbols
of sequence independently from some distribution). The probability that we $X$
and $Y$ under model $R$ is
\[
\prob{X,Y\mid R} = \prod_{i=0}^{|X|-1}\prob{X_i\mid R}\prob{Y_i\mid R}
\]
As a measure of  $X$ and $Y$ being homologous we can the ration of the
probability that $X$ evolved into $Y$ and the probability that $X$ and $Y$ are
independent.
By taking logarithm of this ratio we have
\[
\log\left(
\frac{\prob{X\text{ evolved into }Y}}{\prob{X,Y\mid R} }\right)
= \sum_{i=0}^{|X|-1}\log\left(\frac{p(X_i,Y_i)}{\prob{X_i\mid R}\prob{Y_i\mid
R}}
\right)
\]
By taking $S[a,b]=\frac{p(a,b)}{\prob{a\mid R}\prob{b\mid R}}$ we have one
possible scoring model. $S[a,b]$ is positive if is more likely that $a$ evolved
into $b$ and negative if it is more likely that $a$ and $b$ were generated
independently.  Values of $S$ are usually probabilities multiplied by some
constant and rounded to integers to avoid use of floating point numbers. 

There are two ways of deriving substitution matrices. One way is to derive them
from alignments that were constructed manually by biologists. Such matrices are
for example PAM or BLOSUM matrices. Other possible solution is to use a
theoretical model of evolution, for example Jukes-Cantor model. More details can
be found in \cite{Durbin1998}.

\bigskip
{\Large\bf Podtialto su zapracovane pripomienky }
\bigskip


\section{Needleman-Wunsch algorithm}

Alignment with scoring scheme with previous section can be found by
Needleman-Wunsch algorithm \cite{Durbin1998}.
\todo{nieco o tom ze to bol povodne kubicky, ale sankoff spravil lepsiu verziu}
This algorithm uses arbitrary score table $S$, affine gap model with gap penalty
$d$ with
gap opening penalty $g=0$ (this model is sometimes called linear gap model). To
align sequences $X$ and $Y$ with length $n$ and $m$ respectively, we define
matrix $M$ of size $n\times m$. $M[i,j]$ will be the score of the best alignment
of sequences $X[:i]$ and $Y[:j]$. We can compute $M[i,j]$ by following
equations:

\begin{align} 
M[-1,-1] &= 0\\
M[-1,i] &= i\cdot d, 0< i < m\\\
M[i,-1] &= i\cdot d, 0< i < n\\
M[i,j] &= \max
\begin{cases}
 M[i-1,j-1]+S(X_i,X_j)\\M[i,j-1]+d\\
 M[i-1,j]+d
\end{cases}, 0\leq i<n,0\leq j<m \label{ALIGN:ALGO:AFFINE}
\end{align}

By computing $M[n-1,m-1]$ we have the score of the alignment of $X$ and $Y$ with
the highest score. We can reconstruct such alignment by back-tracing matrix $M$.
Back-tracing procedure will be described later.

To cope with gap opening penalty, we have to slightly change the algorithm.
We define two other matrices $M_X$ and $M_Y$ of same size as $M$. $M_X[i,j]$
will contain the score of alignment of sequences $X[:i]$ and $Y[:j]$ than ends
with gap in sequence $X$. $M_Y$ is analogous. Now we show how to compute $M,M_X$
and $M_Y$. 

\todo{Je vlastne vobec treba taketo vypisovat?}
\begin{align}
M[-1,-1] &= 0\\
M[-1,i] &= M_X[-1,i] = i\cdot d+g, 0 < i < m\\
M[i,-1] &= M_Y[i,-1] = i\cdot d+g, 0 < i < n\\
M_X[i,-1] &= -\infty, 0\leq i< n\\
M_Y[-1,i] &= \infty, 0 \leq i< m\\
M[i,j] &= \max
\begin{cases}\label{ALIGN:ALGO:REALAFFINESTART}
 M[i-1,j-1]+S(X_i,X_j)\\
 M[i,j-1]+d\\
 M[i-1,j]+d\\
 M_X[i,j]\\
 M_Y[i,j]
\end{cases}, 0\leq i<n,0\leq j<m\\
M_X[i,j] &= \max
\begin{cases}
M[i-1,j]+g+d\\
M_X[i-1,j]+d
\end{cases}, 0\leq i<n,0\leq j<m\\
M_Y[i,j] &= \max
\begin{cases}
M[i,j-1]+g+d\\
M_Y[i,j-1]+d
\end{cases}, 0\leq i<n,0\leq j<m\label{ALIGN:ALGO:REALAFFINEEND}
\end{align}
As in previous case, values of $M,M_X$ and $M_Y$ can be computed by simple
dynamic and by back-tracing these matrices we can obtain optimal alignment of
$X$ and $Y$.  Both algorithms have time complexity of $O(nm)$ and memory
complexity $O(mn)$.  Using tricks described later, the memory and even time
complexity will be improved.

\section{Dynamic Programming}\label{DYNPROG}

In this section we show how to compute those values effectively and describe
several methods how to decrease memory footprint and/or time complexity. 

Both algorithm from previous sections contain recursive procedure, that update
matrix $M$ (or three matrices $M,M_X$ and $M_Y$). It used only values from
neighbouring cells which have at least one coordinate smaller. Let $F$ be the
function, that takes as input matrix $M$ (or $M, M_X$ and $M_Y$ in case of
$g\not=0$) and coordinate $(i,j)$ and computes
$M[i,j]$ according equation \ref{ALIGN:ALGO:AFFINE} (or equations
\ref{ALIGN:ALGO:REALAFFINESTART}-\ref{ALIGN:ALGO:REALAFFINEEND}).
Let $F'$ be same function as $F$, but with $\max$ replaced with $\arg\max$.
Clearly, $F'(M,(i,j))$ will return which cell was used to computation of
$F(M,(i,j))$.

%For simplicity, we will assume that both sequences has same length $n$. This
%will allow us to express complexity in simple way.  At first we show how
%to compute compute those algorithms in $O(n^2)$ time and $O(n^2)$ 
Needleman-Wunch algorithm can be computed by following code. Note that in case
of gap opening penalty is not zero, matrix $M$ should be replaced with triple of
matrices $(M,M_X,M_Y)$. 

\lstset{showstringspaces=false}
\begin{lstlisting}
Initialize M[i,-1] and M[-1,i]
for i in 0...n-1
  for j in 0...m-1
    M[i,j] = F(M,(i,j))
(i,j) = (n-1,m-1)
while i > 0
  (i',j') = F'(M,(i,j))
  (a,b) = (X[i],Y[j])
  if i' = i then a = '-'
  if j' = j then b = '-'
  print column of alignment (a,b)
  (i,j) = (i',j')
\end{lstlisting}

Lines 2-5 fills matrix $M$ and lines 6-12 implements the back-tracing procedure.
Time complexity of this algorithm is $O(nm)$ and memory requirements are $O(mn)$
since we keep in memory matrix $M$.

Note, that for computing $i$-th row of matrix $M$ we need only values from row
$i$ and $i-1$. Therefore if we want to know just score of the optimal alignment
we can compute it in $O(m+n)$ memory: after computing of row $i$, we can discard
row $i-1$.

In following subsections we will review several techniques that can be used to
improve performance of dynamic programming used to compute alignment.

\ subsection{Restricting Search Space}

The commonly used technique that is used to speedup (and decrease memory
requirements) of sequence alignment is to restrict the search space of dynamic
programming. By restricting search space we mean to compute alignment only in
some parts of matrix $M$. We assume that omitted parts of matrix correspond to
alignments with low score. In this section we will review few techniques that are
used to with local or global alignment.

If sequences are quite similar, the optimal global alignment will not be too far
from diagonal of matrix $M$. Therefore it is not necessary to compute parts of
matrix $M$, that are too far from diagonal (distance from diagonal is user
defined value). However, this is not useful for local alignment or global
alignment of too distant sequences. Now we will discuss some more advanced
techniques to restricting search space of dynamic programming.

\subsubsection{Seeds}

Seeds are usually used to reduce time complexity of local alignments.  They were
introduced in BLAST algorithm \cite{Altschul1990}.  Seed is position in both
sequences, that can be extended to alignment with high score. After is seed
found, it is extended with extension algorithm to local alignment.

Seeds are usually found by some heuristic algorithm, that computes set
\firstUseOf{candidate seeds}. After that, each candidate seed is extended into
local alignment. Those which were not extended to high-scoring alignment are
discarded. Such candidate seeds are called \abbreviation{false-positive}{FP}.
Candidate seeds that were extended to high-scoring alignments are called
\abbreviation{true-positive}{TP}.  Seeds that were not found by our heuristics
are called \abbreviation{false-negatives}{FN} and rest is called
\abbreviation{true-negatives}{TN}. It is important that heuristics should have
high sensitivity ($TP/(TP+FN)$) and high specificity ($TN/(TN+FP)$). If
heuristics have low sensitivity, many true high-scoring local alignment will not
be found. Low specificity implies longer running time. 

Traditional approach is to find candidate seeds positions $i$ and $j$ in
sequences $X$ and $Y$, such that $X[i:i+\tau]=Y[j:j+\tau]$ for some constant
$\tau$. This approach is used in BLAST \cite{Altschul1990}. Other approaches is
to allow some constant number of mismatches \cite{Kent2002}. \firstUseOf{Space
seeds} allow to contain gaps \cite{Ma2002}. \firstUseOf{Vector seeds}
\cite{Brejova2005vector} are extension of space seeds and \firstUseOf{daughter
seeds} \cite{Csuros2005} are generalization of previous seed methods. All of
those methods vary in speed, sensitivity and specificity.

Extension is done in both ways, usually using equation \ref{ALIGN:ALGO:AFFINE}
(equation is altered in reverse direction). Extension is stopped, when some
criteria is reached. For example, BLAST introduced X-drop heuristics: extension stops
if score of alignment is lower than best score that was seen so far minus user
defined constant \cite{Altschul1997}.

%space seeds, daughter seeds

\subsubsection{Stepping Stone Algorithm}
\label{SECTION:SSA}

\abbreviation{Stepping stone algorithm}{SSA} was used in
\cite{Meyer2002,Pairagon2009} and it is suitable for global alignment. It uses
local alignment algorithm as subprogram. Idea of SSA is to use good local
(generated by some local alignment tool, i.e. BLAST \cite{Altschul1997})
alignment as anchors. Anchor is similar to seed: it is short alignment which we
expect to be in optimal global alignment\footnote{Or at least to be ``near''
optimal global alignment.}.  However, local alignments tools give local
alignments that do not have to be consistent with each other. Set of local
alignments is consistent if all local alignments can be together in one global
alignment.  Therefore there is need for algorithm, that will choose consistent
subset of alignments.

To get such subset $S$, SSA uses following greedy method. It start with
$S=\emptyset$. In every step it finds alignment $A$, such that $S\cup\{A\}$ is
consistent set of alignment and $A$ has highest score possible. Once there is no
such alignment, iteration will stop and $S$ contains consistent set of
alignments which will be used as anchors.

Unlike seeds, these anchors will not be extended to global alignment. Since
local alignment can contain some errors (they were generated by some faster
local alignment tool), SSA will relax them. If $X_i$ and $Y_j$ were aligned in
some anchor, then $X_i$ can be aligned to positions from $j-\tau$ to $j+\tau$ in
global alignment\footnote{Or aligned to gap in that region.} for user defined
constant $\tau$. Similarly, $Y_j$ can be aligned to positions from $i-\tau$ to
$i+\tau$.

Time complexity and memory requirements of stepping stone algorithm are of order
of $O(\sqrt{|X|^2+|Y|^2})$ \cite{Meyer2002}.

%At first it runs
%local alignment tool, like BLAST \cite{}, to get set of good local alignments
%(local alignment is good if it's score and $p$-value is higher than some
%threshold).  Intention is to use good local alignments as anchors: we will
%search only for global alignments that are near such local alignments. However,
%local alignments do not have to be consistent. 

%Therefore we have to choose subset of alignments that we will used as anchors.

%Meyer {\it et al. (2002)} gave following greedy algorithm: The sort local
%alignments according their score. They will use the 
%They start with empty set of anchors. At every
%step take take  


\subsection{Checkpointing}

Checkpointing is general trick in dynamic programming, that allows us to save
$O(\sqrt n)$ rows of dynamic programming matrix while doubling running time
\cite{Grice1997}. 
%We will treat matrices $M,M_X$ and $M_Y$ as one matrix
%containing triples.

In order to compute $i$-th row of matrix $M$, we need only row $i-1$. As
mentioned above, to compute score of the best alignment we need to remember only
two rows (previous and current which we are computing).  If we want to
back-trace the optimal alignment, after we compute it's score, we need all
rows of matrix $M$ again, therefore we would have to compute them again.
Since we need for back-tracing rows in decreasing order, this would lead to
$O(m^2n^2)$ time, since every time we need previous row we would have to recompute
it from beginning.

Checkpointing solves this problem by remembering every $k$-th row of matrix $M$.
For this we need to remember $\lceil n/k\rceil$ rows.  While back-tracing, we
will remember additional block $B$ of consecutive $k$ rows in interval
$<ik,(i+1)k-1>$. We can compute such block on $O(kn)$ time using simple
non-optimized dynamic programming described above.  If we need row that
is outside $B$, we replace block $B$ with block that contain such row. Since we
access every row only once and in decreasing order, we recompute every block at
most once. Therefore time complexity of back-tracing will be still $O(n^2)$. We
have to remember every $k$-th row and one block of size $k$, so memory
complexity is $O(\lceil n/k\rceil n+ kn)$. If we set $k=\sqrt n$ then memory
complexity will be $O(n\sqrt n)$ while time complexity will remain same.


 
\subsection{Hirschberg Algorithm}

\abbreviation{Hirschberg algorithm}{HA} is divide and conquer approach to reduce
memory requirements of sequence alignment \cite{Hirschberg1975}. Idea is
following: if we want alignment of sequences $X$ and $Y$ which has bases $X_i$
and $Y_i$ aligned, we have do dynamic programming only in submatrices
$M_1=M[0:i-2,0:j-2]$ and $M_2=M[i-1:n-1,j-1:m-1]$. If $i=\lceil n/2\rceil$ then
total number of cells in those matrices is roughly half of the number of cells
in $M$. Hirschberg algorithm incorporates procedure how to compute such $j$ for
$i=\lceil n/2\rceil$ in optimal alignment of both sequences in $O(nm)$ time and
$O(n+m)$ memory\footnote{There is one problem. $X_i$ does not have to be aligned
to $X_j$, because $X_i$ is indel. In such case, there is $j$ such that $X_i$ is
aligned to gap that comes right after $X_j$ and we have to change $M_1$ to
$M[0:i-2,0:j-1]$}. HA use such $j$ to determine submatrices $M_1,M_2$. Optimal
alignment is concatenation of optimal alignments in matrices $M_1$ and $M_2$.

To determine $j$, such that $X_i,i=\lceil n/2\rceil$ is aligned to $Y_j$ (or
$X_i$ is aligned to gap right after $Y_j$) HA uses following algorithm: Let
$B(X,Y)$ be the algorithm, that computes for $X$ and $Y$ vector $LL(k)$, where
$LL(k)=M[n-1,k]$ ($LL$ is the last row of $M$). This can be computed in
$O(nm)$ time and $O(n+m)$ memory using algorithm from section
\ref{DYNPROG}.  We compute $LL_1=B(X[0:i],Y)$ and $LL_2=B( (X[i,n])^R,Y^R)$.
While $LL_1[k]$ contains score of optimal alignment of $X[0:i]$ and $Y[0:k]$,
$LL_2^R[k]$ contains score of optimal alignment of $X[i,n]$ and $Y[k,m]$.
$j$ is such column, that maximizes $\max\{LL_1[j]+d+LL_2^R[j],
LL_1[j-1]+LL_2^R[j] \}$.

Since total size of the subproblem is always at most half of the size of the
original problem, the running time of the algorithm will double and therefore
running time of HA is still $O(|X||Y|)$. HA keeps in memory only constant number
of rows of $M$ and reconstructed alignment and therefore memory requirements are
$O(|X|+|Y|)$.

Hirschberg Algorithm (HA) reduce memory more than checkpointing, but HA can be
only applied to algorithms that use back-tracing. For example Forward-Backward
algorithm can be improved by checkpointing but not by HA. More in chapter
\ref{CHAPTER:HMM}.

 
\subsection{Exploiting Sequence Repetition}

This technique reduce time complexity of alignment algorithm to $O(n^2/log n)$.
Unlike four-russian trick \cite{GusfieldBook}, this technique enables usage of
any cost matrix.  This idea combines LZ78 factorization \cite{Lempel1976} and
$O(A+B)$ algorithm for computing row minima/maxima in totally monotone matrix of
size $A\times B$ \cite{Aggarwal1987}. We will discuss only two main ideas behind
this algorithm.

\subsubsection{Totally Monotone Matrices}

\begin{definition}\cite{Crochemore2002}
Matrix $A$ of size $n\times m$ is \firstUseOf{totally monotone} (with concave condition),
if and only if for all $0\leq i,j< n, 0\leq k<l<m$ following condition holds:
if $M[i,k]\leq M[j,k]$ then $M[i,l]\leq M[j,l]$.
\end{definition}

If $M$ is totally monotone then we can compute maximum of every row (or column)
in $O(n+m)$ time by SMAWK algorithm \cite{Aggarwal1987} (This problem is called
\firstUseOf{row maxima}).

\begin{definition}\cite{Crochemore2002}
Let $M$ be matrix and $M'$ be it's
submatrix. \firstUseOf{Input border} $I_{M'}$ of $M'$ is the left and top
borders of $G$ and \firstUseOf{output border} $O_{M'}$ of $M'$ is the right and
bottom border of $M'$. Elements of $I_{M'}$ are ordered in clockwise direction
and elements of $O_{M'}$ are ordered in counter-clockwise direction.
\end{definition}

Intuition behind input and output border is following. If we look on the
computation of alignment algorithm inside submatrix $M'$, input border is input
to this computation and output border is output of this computation.

\begin{definition}\cite{Crochemore2002}
Let $M'$ be submatrix of $M$, $I_{M'}=\{i_0,i_1,\dots,i_{k-1}\}$ be its input
border, and $O_{M'}=\{o_0,o_1,\dots,o_{l-1}\}$ be its output border then
matrices
$DIST$ and $OUT$ of size $k\times l$ are defined in following way:
$DIST[a,b]$ is the cost of optimal alignment from cell $I_a$ to cell $I_b$.
$OUT[a,b]=I_a+DIST[a,b]$.
\end{definition}

Clearly, $O_b=\max_{0\leq a < k}OUT[a,b]$. Therefore by computing row maxima in
matrix $OUT$, we can compute values of output borders. Matrices $DIST$ and $OUT$
are totally monotone \cite{Crochemore2002}.  If we have matrices $DIST$ and
$OUT$ in advance and $M'$ is of size $m'\times n'$ then we can compute values of
output border in time $O(n'+m')$ by SMAWK algorithm.

Now we show how to divide matrix $M$ into number of submatrices in a way, that
it is possible to effectively represent matrices $DIST$ and $OUT$.

%Let $M$ be dynamic programming matrix for alignment algorithm for some sequences
%$X$ and $Y$. Let $M' =
%M[i:j,k:l]$ be rectangular submatrix of $M$. Values of $M'$ depends on 
%values cells in submatrices $M[i-1,j:l]$ and $M[i:k,j-1]$ and on value
%$M[i-1,j-1]$. Let $I_{M'}$ be the set of such cells. Let $O_{M'}$ be the 
%the set of cells in $M[i-1,j:l]$ and $M[i:k,j-1]$ (the cells in the top row or
%right column of $M'$). Let $I_0,I_1,\dots$ be enumeration of cells in $I_{M'}$
%and $O_0,O_1,\dots$ be enumeration of cells in $O_{M'}$.  Now we will construct
%the  matrix $DIST_{M'}$ of size $|I_{M'}|\times|O_{M'}|$ where  
%$DIST_{M'}[i,j]$ is the cost of optimal alignment from cell $I_i$ to cell $O_j$.
%Let $OUT[i,j]=I_i+DIST[i,j]$. Both $DITS$ and $OUT$ are totally monotone with
%convex condition 

\subsubsection{LZ78 factorization}

LZ78 is compression algorithm that uses dynamic dictionary that is being built
while sequence is compressed \cite{Lempel1976}. The way how it parse sequence
can be used to accelerate dynamic programming \cite{Crochemore2002,Weimann2009}. We will be interested in
factorization of input sequence by this algorithm. LZ78 factorization divide
sequence $S$ into $k$ strings $S_0,\dots,S_{k-1}$, where $S_0S_2\dots S_{k-1}=S$ and
for every index  $i,0< i <k$ there is index $0\leq j<i$ such that $S_j$ is
prefix of $S_j$ of size $|S_i|-1$. We will call $S_j$ to be a
\firstUseOf{predecessor} of $S_i$.  We have guarantee, that number of strings
$k$ is in  $O(\frac{n}{\log n})$ where $n$ is the length of $S$
\cite{Lempel1976}. 

To align sequences $X$ and $Y$, we factorize them into sequences of strings
$\{X_i\}_{0\leq i < k_s}$ and $\{Y_i\}_{0\leq i<k_t}$.  Every pair $(X_i,Y_j)$
define pair of rectangular block $B_{i,j}$ submatrix of $M$.  All $B_{i,j}$ are
disjoint and all blocks cover matrix $M$. We say that block $B_{k,l}$ is
predecessor of block $B_{i,j}$ if one of the following conditions is true:

%If we want to align sequences $S$ and $T$, we factorize $S$ and $T$
%into sequences of strings $\{S_i\}_{0\leq i < k_s}$ and $\{T_i\}_{0\leq i<
%k_t}$. Every pair $(S_i,T_j)$ defines rectangular block of matrix $M$.
%We say that $(S_k,T_l)$ is predecessor of $(S_i,T_j)$ if one of the following
%conditions is true:

\begin{itemize}
\item $S_k$ is predecessor of $S_i$ and $l=j$
\item $k=i$ and $T_l$ is predecessor of $T_j$
\item $S_k$ is predecessor of $S_i$ and $T_l$ is predecessor of $T_j$
\end{itemize}

Note, that every block can has only $0,1$ or $3$ predecessors. Only block
$B_{0,0}$ will have $0$ predecessors and only blocks $B_{i,0}$ and $B_{0,i}$
($i>0$) have only $1$ predecessor.  

\todo{Mozno by nebolo odveci popisat aj tu strukturu, ak bude cas}
{\it Crochemore et al.} described data structure that represents $DIST$ and $OUT$
matrices for block \nocite{Crochemore2002} $B_{i,j}$. That structures can be
computed from $DISTS$ and $OUT$ matrices of theirs predecessors in time
$O(|X_i|+|Y_j|)$. Details about this algorithm can be found in
\cite{Crochemore2002}. Time complexity of this algorithm if proportional to the
sum of the sizes of all blocks, which is $O(n^2/\log n)$ where $n$ is the length
of the longer sequence.

\bigskip
{\large\bf Odtialto hore povazujem vsetko za ``hotove''. }
\bigskip



\section{Non-Affine Gap Models} 

\todo{Mozno by sa dalo dodat aj viac citacii} Reason why affine gap models are
used in sequence alignment is because they are simple, easy to compute and gave
reasonable results. While affine gap score works fine for short gaps, penalty
for long gaps is too high. Use of different gap models can improve quality of
reconstructed alignments \cite{Gill2004,Cartwright2009}.

Let $f(x)$ be the gap penalty where $x$ is the length of the gap. In case of
affine gap models, $f(x)=g+dx$. In case of general gap function, we have to
reformulate equation \ref{ALIGN:ALGO:AFFINE} in following way:
\begin{align}
M[i,j] &= \max
\begin{cases}
 M[i-1,j-1]+S(X_i,X_j)\\
 M[i,j-y]+f(y)\\
 M[i-x,j]+f(x)
\end{cases}, 0<x\leq i<n,0<y\leq j<m\label{ALIGN:ARBITRARYGAPEQUATION}
\end{align}
This algorithm has time complexity of $O(n^3)$ where $n$ is the length of the
longest sequence. Note that this was the original Needleman-Wunsch 
algorithm \cite{Needleman1970} and later it was improved to $O(n^2)$ algorithm
\cite{Sankoff1972} for alignment without gap penalties.
Unlike previous recursive equations, int this algorithm $M[i,j]$ depends not
only on neighbouring cells, but also on all previous in same row and column.
Therefore some techniques like Hirschberg algorithm or Checkpointing cannot be
used with general gap penalties.

\subsection{Convex/Concave Gap Functions}\label{SECTION:CONVEX}

Arbitrary gap penalties leads to slow running time, however if we place some
restrictions on gap penalty, we can use faster algorithms. In this section we
show algorithm that computes optimal alignment on $O(nm\log n)$ time if gap
penalty is convex or concave. In some cases this algorithm can be improved to
$O(nm)$ time. Originally this algorithm was discovered independently by Miller
{\it et al. (1988)} and Galil {\it et al. (1989)} \nocite{Miller1988,Galil1989}.

We will present a variant of $O(nm\log n)$ algorithm for concave gap functions which
we believe is easiest to explain . Algorithm for convex gap function is similar.
We consider only concave functions defined on natural numbers (gap length can be only
natural number). Our definition of concave function is similar to one used in
\cite{GusfieldBook}.

\begin{definition}
We assume that $f(x)$ function defined on natural numbers. $f$
is concave if and only if 
\[f(x+1)-f(x)\geq f(x)-f(x-1)\]
for all $x\in\mathbb{N}$.
\end{definition}
\begin{note}
It is easy to show that if $f$ is concave then $f(x+d)-f(x) \leq f(x+e+d) -
f(x+e)$ for nonnegative $d$ and $e$.
\end{note}

We will improve dynamic programming that computes matrix $M$ according equation
\ref{ALIGN:ARBITRARYGAPEQUATION}. In this equation there are two
problematic\footnote{In terms of time complexity} terms: $M[i,j-y]+f(y)$ and
$M[i-x,j]+f(x)$. In dynamic programming from section \ref{DYNPROG} we  have
to add additional nested cycles that goes through all $x$ and $y$. We show data
structure that can compute $\max_{0\leq y < j}M[i,j-y]+f(y)$ in $O(\log n)$ time
(amortized). This data structure can handle also second problematic term so we
will focus only on the first one.

$\max_{0\leq y < j}M[i,j-y]+f(y)$ is equivalent to
\begin{equation}
\max_{0\leq k < j}M[i,k]+f(j-k)\label{CONVEX:MAXFUNCTION}
\end{equation}
From now on we will use this second equation since it is more suitable for this
approach.  Our data structure is list ($L_j$) of such $k$'s that can be in
future maximum. We will try to keep this list as small as possible. $L_j$ will
called \firstUseOf{candidate list} and members of $L_j$ will be called
\firstUseOf{candidates}. $L_j$ is candidate list for computing $M[i,j]$.
$L_{j+1}$ will be always computed from $L_j$ by adding $j$ and removing some
elements of $L_j$. For simplicity, let $G(k,j) = M[i,k]+f(j-k)$. If $k\in L_j$,
than \firstUseOf{rank} of $k$ is $G(k,j)$. Note that rank of candidates will
change when we move from $L_j$ to $L_{j+1}$.

We will explain this algorithm in iterative way: we will be adding to algorithm
new features until it will have desired time complexity. 

%Let $L_j$ be the list of such $k$'s. We will call $L_j$ a \firstUseOf{candidate
%list} and members of $L_j$ \firstUseOf{candidates}. $L_j$ will be created from
%$L_{j-1}\cup\{j-1\}$ by removing some elements. Let $G(k,j) =
%M[i,k]+f(j-k)$.  \firstUseOf{Rank} of candidate $k$ is $G(k,j)$.  
We start with $L_j$ containing all $k$ that are smaller then $j$.  We can find
element with maximal rank by computing rank for all candidates from candidate
list $L_j$. This leads to $O(j)$ algorithm to find maximum and $O(1)$ algorithm
to update candidate list (we have to add element $j$ to list $L_{j}$ to create
list $L_{j+1}$). This is equivalent to original Needleman-Wunsch algorithm
\cite{Needleman1970}. At first, we will use following lemma that we will use to
decrease size of candidate lists. This lemma and proof is almost same as in
\cite{GusfieldBook} (with difference that we use concave function instead of
convex function).

\begin{lemma}\label{OneStrikeAndOut}
If $G(k,j)\leq G(k',j)$ for some $k<k'<j$ then
$G(k,j')\leq G(k',j')$ for all $j'\geq j$. 
\end{lemma}

Use of this lemma is following. Once we found out that $G(k,j)\leq
G(k',j),k<k'$, we can remove $k$ from $L_j', j'\geq j$ without affecting result
of algorithm (because rank of $k$ will be always smaller than rank of $k'$).
Before we continue, we have to prove this lemma.

\begin{proof}

Since $M[i,k]+f(j-k)\leq M[i,k']+f(j-k')$ then 
$M[i,k]-M[i,k']\leq f(j-k')-f(j-k)$. From convexity and that $j'>j$ we have
$f(j-k')-f(j-k)\leq f(j'-k')-f(j'-k)$ and therefore
$M[i,k]-M[i,k']\geq f(j'-k')-f(j'-k)$ which 
is equivalent to
$M[i,k]+f(j'-k)\leq M[i,k']+f[(j'-k')$ or $G(k,j')\leq G(k',j')$.

\end{proof}

Using this lemma, we can improve our algorithm: 
when we move from $j$ to $j+1$, we set $L_{j+1}=L_{j}\cup\{j\}$ and after that
we remove from $L_{j+1}$ candidates that have lower or equal rank than following
candidate (we will remove such candidates while they exists in $L_{j+1}$).
%when we move from
%$j-1$ to $j$ we add set $L_j=L_{j-1}\cup\{j-1\}$ and we compute $G(k,j)$ for
%all $k\in L_j$ and remove all such $k$ where there exists $k'\in L_j$ such that
%$G(k',j)\geq G(k,j)$. i
This can be done on $O(|L_{j}|)$ time. After removing such candidates,
candidates in $L_{j+1}$ will have decreasing rank. Therefore first item in
$L_{j+1}$ is the maximal one, so we can find it $O(1)$ time.  This improvement
still does not improve worst-case behaviour of our algorithm, but it might
significantly reduce size of $L_j$. 

Using this algorithm, $L_j$ will be always ordered by rank. However when we move
to next column ($j+1$) the ranks will change and we have to remove some elements 
to have this property again. This can be improved if we can compute column in
which  candidates will broke order.

%Nice property of $L_j$ is that it's members are ordered by rank. However, 
%when we move to column $j+1$, ranks of candidates will change and order might be
%broken and therefore we have to traverse through whole $L$ to fix that.
\begin{definition}
Let $H(k,k')=l, k<k'$ be minimal $l$, such that $G(k,l)\leq G(k',l)$. If such $l$ does
not exists then $l$ is $\infty$ 
\end{definition}

If rank of $k$ is greater than rank of $k'$ then $H(k,k')$ is the first column
(of first candidate list $L_{H(k,k')}$) where rank of $k$ is less or equal to
rank of $k'$. In this column candidate $k$ will be removed for sure.  Note that
such $l$ can be find using binary search (lemma \ref{OneStrikeAndOut} allows us
to do it) in $O(\log n)$ time for any concave function. For some concave
function we can find it in constant time.

\begin{lemma}\label{TimeLemma}
Let $k,k',k''\in L_j$ and $k<k'<k''$. If $H(k,k')\geq H(k',k'')$ then $k'$ can be
removed from $L_j$.
\end{lemma}

\begin{proof}
Since candidates in $L_j$ are ordered by rank then $G(k,j)>G(k',j)>G(k'',j)$.
From definition of $H$ we know that 
$G(k',j')\leq G(k'',k')$ for  all $j'>=H(k',k'')$. Since
$H(k,k')\geq H(k',k'')$ then $G(k,j')\geq G(k',j')$ for all $j'<j$. Therefore rank
of $k'$ will never be maximum and $k'$ can be removed.

%and both are in $L_j$ then $k$ has higher rank that $k'$. Therefore
%it is not maximum. $k'$ have in $H(k',k'')$ lower rank and will be removed from
%candidate list (using lemma \ref{OneStrikeAndOut}). However, since $H(k,k')\geq
%H(k',k'')$ rank of $k'$ will never be maximum.
\end{proof}

Now we can formulate final algorithm. After updating $L_j$ from $L_{j-1}$,
$L_j$ will satisfy following invariant.
\begin{invariant}\label{LogGapInvariant}
All consecutive elements $k,k'$ in $L_j$ satisfy following properties:
\begin{enumerate}
\item $k<k'$
\item $G(k,j)>G(k',j)$ (decreasing rank)
\item if $k'$ is not last element of $L_j$ and $k''$ is element right after $k'$
in $L$ then $H(k,k')> H(k',k'')$ (increasing ``time'' when consecutive
candidates will broke order of ranks)
\end{enumerate}
\end{invariant}

Since rank is decreasing, we can find maximum in $O(1)$ time. Now we show how
to compute $L_{j+1}$ from $L_j$. $L$ will be our working list.
Let $L^k$ be $k$-th element of $L$ and $l$ be index of last element of $L$. 
$L_{j+1}$ can be computed by following algorithm.
\begin{enumerate}
\item $L=L_j$
\item If $G(L^0,j+1)< G(L^1,j+1)$ then remove $L^0$ from
list (lemma \ref{OneStrikeAndOut}).
\item While $H(L^{l-1},L^{l})<H(L^l,j-1)$ or $G(L^l,j+1)<G(j,j+1)$ do following: remove $L^l$ from list (lemma
\ref{TimeLemma} and lemma \ref{OneStrikeAndOut}).
\item $L_{j+1}=L\cup\{j\}$.
\end{enumerate}
First element is also first element that can broke order so we have to only it
(step 2). Since we will add candidate $j$, we have to remove from the end of the
list all candidates with smaller rank than $j$. Same arguments hold for ``time''
when consecutive candidates will broke order of ranks. Since both (ranks and
times) are ordered, we need to remove candidates only from the end of the list.

It is easy to see that if $L_j$ satisfy invariant then $L_{j+1}$ will satisfy
invariant. It is also clear that this algorithm will never remove candidate that
will be maximum later, which conclude correctness of algorithm.

Algorithm described above has running time $O(l\log(n))$, however amortized time
complexity for computing all columns is $O(n\log n)$ since every candidate will
be added/removed to/from list at most once. If we can compute function $H$ in
constant time, time complexity is $O(n)$.

Using this algorithm, we can compute alignment with concave gap functions in
$O(nm\log n)$ time and $O(nm)$ for concave gap functions if $H$ can be computed
in constant time.

There is also more complicated $O(nm\alpha(n)$ algorithm for this problem
\cite{Klawe1990} where $\alpha(n)$ is inverse Ackerman function. However this algorithm
is very complex and we doubt that it will be useful in practice. However, we are
not aware that somebody do experimental comparison of these two algorithms.




%tu chcem definiciu convexnej gap funkcie
%algoritmus ktory je v case n^2 log(n)
%algoritmus, ktory bezi v case n^2 alpha(n)
%kubicky algoritmus v worstcase, ale prakticky konstantny (budem musiet najst 
%citaciu -- nepodarilo sa mi to najst. Zaujimave\ldots)


