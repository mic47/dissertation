\section{Sequence Alignments}

In this section we introduce the sequence alignment problem, basic algorithms
for computing optimal alignments of two sequences.

Parts of two sequences are homologous, if they have evolved from same sequence
in their common ancestor. Aim of sequence alignment is to identify homologous
sequences. Sequences can be modified by different evolution events:
mutation of a residue into another residue, deletion of a part of the sequence,
insertion of residues into the sequence. There are also large-scale
rearrangement events like duplications (some subsequence is duplicated and
copied into other part of the sequence), inversions (some subsequence is
reversed) or translocations in which part of the sequence change position.  We will
ignore them for now because they cannot be represented by traditional
alignments. An alignment is a data structure that represents comparisons of two
or more sequences. We obtain an alignment of $k$ sequences by inserting dashes
into individual sequences so that they have the same length. We can represent an
alignment as a matrix or a table. Each row of the alignment is a sequence with
inserted dashes, and each column is a list of residues from all rows at the same
position.


An alignment has the following biological meaning: homologous residues (those that
have evolved from a common ancestor) are in same column. Dashes represents
either the parts of sequence that were deleted during evolution (deletions) or
positions where some residues were inserted into some other sequence
(insertions). In an alignment it is not possible to distinguish between insertions and
deletions -- we cannot tell if something was inserted into one sequence or if
there was deletion in the other sequences. Therefore we will refer to insertions
and deletions as to \firstUseOf{indels}.

\begin{example} 
Consider the  following evolutionary history of hypothetical DNA sequences of two
current organisms $X$ and $Y$. There was a parent sequence $P$ which evolved into
$X'$ and $Y'$ and after that $X'$ evolved into $X$ and $Y'$ evolves into $Y$.
These sequences are shown bellow: 
\begin{verbatim}
X:      C C     G C G A C C T T G C             A C C A
X':     C C     G           T T G C             A G C A
P:      A C T G G           T C G C T G A G C T A G C A
Y':     T C T G G           C C           G C T A G C A
Y:      T C T A G           C C           G A T A G C A
\end{verbatim}
%$X$ and $Y$ evolved from parent sequence $P$ through sequences $X'$ and $Y'$.
During evolution from $P$ to $X'$, four events occurred: deletion of 
sequences ``TG''  and ``TGAGCT'' and mutations of two bases. During evolution
from $X'$ to $X$, one base was changed and sequence ``CGACC'' was inserted.  
Similarly during evolution from $P$ to $Y'$, the two bases mutated and two
sequences were deleted. During evolution from $Y'$ to $Y$ one base mutated and
sequence ``CGACC'' was inserted. 

From evolutionary history described above, we can create an alignment of current
sequences $X$ and $Y$ by removing ancestral sequences $X',Y'$ and $P$, 
%\begin{verbatim}
%X:      C C     G C G A C C T T G C             A C C A
%Y:      T C T A G           C C           G A T A G C A
%\end{verbatim}
%We obtain actual alignment by 
removing columns that contain only gaps and
replacing gaps with dashes (gap symbols). 
\begin{verbatim}
X:      C C - - G C G A C C T T G C - - - A C C A
Y:      T C T A G - - - - - C C - - G A T A G C A
\end{verbatim}
In this alignment, symbols that are in the same column are truly homologous (they
evolved from the same symbol in $P$).
%If two residues are in same column in alignment then they are homologous.
As you can see, homologous symbols do not have to be equal.
\end{example}

There is only one alignment that reflects the true evolutionary history. Our goal
is to find this alignment, or at least some alignment, which is as close as
possible. The alignment shown above is a \firstUseOf{global alignment} because it
is an alignment of whole sequences $X$ and $Y$. A \firstUseOf{local alignment}
is an alignment of parts of sequences: a local alignment of sequences $X$ and
$Y$ is a global alignment of strings $\bar{X}$ and $\bar{Y}$ where $\bar{X}$ is
a substring of $X$ and $\bar{Y}$ is a substring of $Y$.  Since global
alignments do not consider rearrangement events\footnote{Duplication, reversal
or translocation.}, local alignments are useful to align sequence parts that
did not underwent such events. We will mostly consider global alignments, but
most of the methods can be extended also to local alignments.

In this section we will review basic methods for constructing alignments. We
will discuss basic scoring schemes and algorithms that find optimal alignment
under these schemes. In this thesis we cover mostly probabilistic methods,
which are in section \ref{TODO} and chapter \ref{TODO}.

%At first
%we will discuss pair alignments (alignment of two sequences).
%\correction{Later in this
%chapter we will review some methods how to construct multiple alignments.}{urvat
%ak to nie je pravda}

\subsection{Scoring Schemes}

Since we want to construct alignments that have biological meaning, we have to
develop a method for assessing the quality of an alignment. One way of doing so
is to define a scoring scheme, which assigns to every alignment a real number
(called score). The alignments similar to the true alignment should have higher
score than the alignments that differ from the true alignment. Once we have a
scoring scheme, we will search for the alignment of the input sequences with
the highest score.

Typical scoring schemes used in practice score each column of
an alignment without gaps independently. Gaps are scored by a penalty
that depends on the length of the gap (the number of consecutive dashes). Score
of an entire alignment is the sum of the scores of all ungapped columns plus the
sum of the scores of all gaps.

\todo{Nejake referencie}
In particular
we will assume that all sequences are from a finite alphabet $\Sigma$. For DNA,
$\Sigma=\{A,C,G,T\}$, for proteins $\Sigma$ contains $20$ codes of amino acids.
We will score a column containing residues $a$ and $b$ by $S[a,b]$ where $S$ is
a matrix of size $|\Sigma|\times|\Sigma|$ called \firstUseOf{substitution
matrix}.  A gap of length $x$ has a score $g+dx$, where $g$ is the gap opening
penalty and $d$ is the gap extension penalty. Both are usually negative, since we
want alignments that contains many columns with same or similar symbols. With
positive gap penalty there will be tendency towards alignments with many gaps
which reduce the number of aligned residues.  We call this gap scoring scheme an
affine gap model. 

Matrices and gap penalties used as scoring schemes are usually derived from
frequencies of the substitutions and indels of real alignments. Example of such
matrices are PAM or BLOSUM matrices\cite{Durbin1998}. Additionally, scoring
matrices can be also derived from theoretical models of evolution, one such
example is Jukes-Cantor model\cite{Durbin1998}.

\begin{comment}
Intuition behind this scoring system is probabilistic. Assume that we have
alignment of $X$ and $Y$ without gaps and that each base in DNA evolves
independently. Therefore for every pair of residues $a,b$ there is probability
that $a$ will evolve into $b$. We denote this probability as $p(a,b)$. Therefore the probability that $X$ will evolve
into $Y$ is product of probabilities that $X_i$ will evolve into $Y_i$
\[
\prob{X\text{ evolved into }Y} = \prod_{i=0}^{|X|-1}p(X_i,Y_i)
\]
If $X$ and $Y$ were generated independently be some random model $R$, then probability that we see $X_i$
and $Y_i$ is $\prob{X_i\mid R}\prob{Y_i\mid R}$ ($R$ usually samples symbols
of sequence independently from some distribution). The probability that we $X$
and $Y$ under model $R$ is
\[
\prob{X,Y\mid R} = \prod_{i=0}^{|X|-1}\prob{X_i\mid R}\prob{Y_i\mid R}
\]
As a measure of  $X$ and $Y$ being homologous we take the ratio of the
probability that $X$ evolved into $Y$ and the probability that $X$ and $Y$ are
independent.
By taking logarithm of this ratio we have
\[
\log\left(
\frac{\prob{X\text{ evolved into }Y}}{\prob{X,Y\mid R} }\right)
= \sum_{i=0}^{|X|-1}\log\left(\frac{p(X_i,Y_i)}{\prob{X_i\mid R}\prob{Y_i\mid
R}}
\right)
\]
Therefore by setting $S[a,b]=\frac{p(a,b)}{\prob{a\mid R}\prob{b\mid R}}$ we have one
possible scoring model. $S[a,b]$ is positive if is more likely that $a$ evolved
into $b$ and negative if it is more likely that $a$ and $b$ were generated
independently.  Values of $S$ are usually probabilities multiplied by some
constant and rounded to integers to avoid use of floating point numbers. 

There are two ways of deriving substitution matrices. One way is to derive them
from alignments that were constructed manually by biologists. Such matrices are
for example PAM or BLOSUM matrices. Other possible solution is to use a
theoretical model of evolution, for example Jukes-Cantor model. More details can
be found in \cite{Durbin1998}.
\end{comment}


\subsection{Needleman-Wunsch algorithm}
\label{SECTION:NEEDLE}


The
alignment with the highest score can be found by the Needleman-Wunsch algorithm \cite{Durbin1998}.
This algorithm uses an arbitrary score table $S$, affine gap model with gap penalty
$d$ and gap opening penalty $g=0$ (if $g=0$ then this model is also called
linear gap model). To align sequences $X$ and $Y$ of length $n$ and $m$
respectively, we define matrix $M$ of size $n\times m$. Element $M[i,j]$ will be the
score of the best alignment of sequences $X[:i]$ and $Y[:j]$. We can compute
$M[i,j]$ by the following equations:

\begin{align} 
M[-1,-1] &= 0\\
M[-1,i] &= i\cdot d, 0< i < m\\\
M[i,-1] &= i\cdot d, 0< i < n\\
M[i,j] &= \max
\begin{cases}
 M[i-1,j-1]+S(X_i,X_j)\\M[i,j-1]+d\\
 M[i-1,j]+d
\end{cases}, 0\leq i<n,0\leq j<m \label{ALIGN:ALGO:AFFINE}
\end{align}

By computing $M[n-1,m-1]$, we have the score of the optimal (highest-scoring) alignment of $X$ and $Y$ \cite{Durbin1998}. 

To cope with gap opening penalty, we have to slightly change the algorithm.
We define two other matrices $M_X$ and $M_Y$ of same size as $M$. $M_X[i,j]$
will contain the highest score of an alignment of sequences $X[:i]$ and $Y[:j]$
that ends
with a gap in sequence $X$. $M_Y$ is analogous. Values of these matrices can be computed byy the following recurrences.

\begin{align}
M[-1,-1] &= 0\\
M[-1,i] &= M_X[-1,i] = i\cdot d+g, 0 < i < m\\
M[i,-1] &= M_Y[i,-1] = i\cdot d+g, 0 < i < n\\
M_X[i,-1] &= -\infty, 0\leq i< n\\
M_Y[-1,i] &= -\infty, 0 \leq i< m\\
M[i,j] &= \max
\begin{cases}\label{ALIGN:ALGO:REALAFFINESTART}
 M[i-1,j-1]+S(X_i,X_j)\\
 M_X[i,j]\\
 M_Y[i,j]
\end{cases}, 0\leq i<n,0\leq j<m\\
M_X[i,j] &= \max
\begin{cases}
M[i-1,j]+g+d\\
M_X[i-1,j]+d
\end{cases}, 0\leq i<n,0\leq j<m\\
M_Y[i,j] &= \max
\begin{cases}
M[i,j-1]+g+d\\
M_Y[i,j-1]+d
\end{cases}, 0\leq i<n,0\leq j<m\label{ALIGN:ALGO:REALAFFINEEND}
\end{align}

To compute the
value of $M[i,j]$ (and optionally $M_X$ and $M_Y$), these equations use only values
from neighbouring cells which have at least one coordinate smaller. 
Therefore we can order computation so that when we compute value $M[i,j]$ the
necessary values are already computed.
Let $F$ be
the function, that takes as input matrix $M$  and coordinates $(i,j)$ and computes $M[i,j]$ according equation
\ref{ALIGN:ALGO:AFFINE}.  Let $F'$ be the
same function as $F$, but with $\max$ replaced with $\arg\max$.  Function
$F'(M,(i,j))$ will thus return which cell was used in computation of $F(M,(i,j))$.
Note that if $g$ is not zero, then $F$ would take as input the matrices $M,M_X,$
and $M_Y$ and compute $M[i,j],M_X[i,j],$ and $M_Y[i,j]$ according 
 equations
\ref{ALIGN:ALGO:REALAFFINESTART}-\ref{ALIGN:ALGO:REALAFFINEEND}
%For simplicity, we will assume that both sequences has same length $n$. This
%will allow us to express complexity in simple way.  At first we show how
%to compute compute those algorithms in $O(n^2)$ time and $O(n^2)$

The Needleman-Wunsch algorithm can be implemented by the following code. For simplicity 
we show only computation of matrix $M$.

\lstset{showstringspaces=false}
%\vbox{
\begin{lstlisting}
Initialize M[i,-1] and M[-1,i]
for i in 0...n-1
  for j in 0...m-1
    M[i,j] = F(M,(i,j))
(i,j) = (n-1,m-1)
while i > 0 or j > 0
  (i',j') = F'(M,(i,j))
  (a,b) = (X[i],Y[j])
  if i' = i then a = '-'
  if j' = j then b = '-'
  print column of alignment (a,b)
  (i,j) = (i',j')
\end{lstlisting}
%}

Lines 2-5 fills matrix $M$ and lines 6-12 implement the back-tracing procedure.
Time complexity of this algorithm is $O(mn)$ and memory requirements are $O(mn)$
since we keep matrix $M$ in memory. The algorithm for
affine gap model with $g\not=0$ has the same complexity.


Note that for computing  $i$-th row of matrix $M$ we need only values from row
$i$ and $i-1$. Therefore if we want to know just the score of the optimal
alignment,
we can compute it in $O(m+n)$ memory: after computing of row $i$, we can discard
row $i-1$. However if we want find the optimal alignment, we have to keep matrix
$M$ in the memory or use one of the techniques that are described in section 
\ref{TODO: Aj tak sa to presunie}.

\section{Sequence Alignments with Pair HMM}

In this section we describe \abbreviation{pair hidden Markov models}{pHMM},
which are commonly used for studying relationships between different sequences,
relation between Needleman-Wunsh and pHMM, three existing decoding methods for
decoding pHMMs, and several examples of pHMM that were used for sequence
alignments. 

\subsection{Pair Hidden Markov Models}\label{SECTION:PAIRHMM}

Pair hidden Markov models are HMMs that generate output on two tapes, resulting
in two emitted sequences.  Every state can in one step generate one symbol in
each sequence, or one symbol in one of the sequences or no symbols at all.
Formally, every state generates a pair of strings $(a,b)$, where $a$ and $b$
are of length at most one.  Moreover, all pairs with generated with nonzero
probability by one state have the same length.  Formal definition of pair HMMs
is given below. We can use pHMM to define probabilistic scoring schemes for
alignments.

In particular, symbols generated by the same state are considered homologous
(are in same column of an alignment). Symbols that are generated by a state
that generates only in one sequence are aligned to a gap. 

\begin{definition}
A \abbreviation{pair hidden Markov model}{pHMM} is a tuple $H=(\Sigma,V,I,d,e,a)$, where $\Sigma$ is a finite
alphabet, $V$ is a finite set of states, $I$ is an initial distribution and $a$ is
a transition matrix, all defined as
in definition \ref{DEF:HMM}. $d^x_v$ and $d^y_v$ are state durations of state $v$
in sequence $x$ and $y$ respectively. For all $v\in V$,
$d^x_v\in \{0,1\}$ and $d^y_v\in \{0,1\}$.
Emission probability matrix $e$ is
a $|V|\times\left(|\Sigma\cup\{\varepsilon\}|\right)^2$ matrix with the following
properties:
\def\lala#1{\{#1\}}
\begin{enumerate}
\item
$\forall v\in V\forall \sigma_1,\sigma_2\in\Sigma\cup\{\varepsilon\}:
0\leq e_{v,(\sigma_1,\sigma_2)}\leq 1$

\item 
$\forall v\in V:
\sum_{\sigma_1,\sigma_2\in\Sigma\cup \lala{\varepsilon }}e_{v,(\sigma_1,\sigma_2)} = 1$

\item For all states $v$ if $e_{(v,\sigma_1,\sigma_2)}>0$ then
$d^x_v=|\sigma_1|$ and $d^y_v=|\sigma_2|$
\end{enumerate}

\end{definition}

A state path is defined the same as for HMMs with silent states. Restriction on
the state durations (condition $3$ in the definition) ensures that given a
state path $\pi$ and emitted sequence $X$ and $Y$, for every symbol from $X$
and $Y$ we can assign a state from $\pi$ that generated that symbol.  However,
given only two sequences $X$ and $Y$ and no state path, it is not possible to
determine which symbols of the two sequences were generated together.

%Before we continue, we will define cumulative duration times which will make
%further expressions simpler.
\todo{koniec feedbacku}

\begin{definition}
Let $\pi=\pi_0\dots\pi_l$ be a state path. Then the \firstUseOf{cumulative duration
times} are
$D^x_i(\pi)=\sum_{j=0}^{i}d^x_{\pi_j}$ and $D^y_i(\pi)=\sum_{j=0}^{i}=d^y_{\pi_j}$.
Additionally, $D^x_{-1}(\pi)=D^y_{-1}(\pi)=0$. If it will be clear from the context
which state path we are using, we will write $D^x_i$ and $D^y_i$ instead of
$D^x_i(\pi)$ and $D^y_i(\pi)$.
\end{definition}

Given sequences $X,Y$ and state path $\pi$ that generated them, we 
can tell which symbols were generated by which states. Since every state $v$
generates exactly $d^x_v$ and $d^y_v$ symbols in $X$ and $Y$ respectively,
state $\pi_i$ generated pair $(X[D^x_{i-1}:D^x_{i}],Y[D^y_{i-1}:D^y_{i}])$
States $\pi_0,\pi_1,\dots\pi_{i-1}$ generated first $D^x_{i-1}$ symbols in $X$
and first $D^y_{i-1}$ symbols in $Y$. 


\begin{example}
In the figure \ref{FIGURE:SIMPLEPHMM} is pair hidden Markov model modeling
sequence alignment with affine gap model. State $M$ is called match state and
generates pair of aligned residues and corresponds to matrix $M$ from the sequence
alignment algorithm. Insert states $I_X$ and $I_Y$ represents gaps,
generates residues in only one sequence and corresponds to matrices $M_X$ and
$M_Y$ from the sequence alignment algorithm from section \ref{SECTION:NEEDLE}.

Note that a state path uniquely define an annotation (match state corresponds to
match columns, insert states represent gap).  Finding the most probable state
path for this HMM is equivalent to the Needleman-Wunsch algorithm with affine gap
model \cite{Durbin1998}.

%\begin{figure}
%\begin{center}
%\includegraphics{../figures/simplePairHMM.pdf}
%\end{center}
%\caption[Example of pair hidden Markov model.]{
%State $M$ generates aligned pairs of residues residues, state $I_X$ ($I_Y$) generates
%symbol only in the first (second) sequence.
%}\label{FIGURE:EXAMPLEPAIRHMM}
%\end{figure}
%\section{Use of Pair Hidden Markov Models}\label{SECTION:SIMPLEPHMM}


%We can divide states of pHMM into three types.  Ones that generates symbol in
%both sequences (match states), states that generates symbol in only one sequence
%(indel states) and silent
%states. If pHMM generates  symbols in both sequences we consider those symbols
%to be homologous. We will consider symbols that were not generated by such state
%as indels. Alternative view is to imagine that indel states generate symbol in
%one sequence and gap symbol in the second sequence. In such view pHMM generates
%an alignment.

%Now we show classical pHMM for sequence alignment. It is equivalent to scoring
%scheme of Needleman-Wunsch algorithm with affine gap model. It consists of three
%states: one that generates aligned pairs, and two states for generating
%indels (one for each sequence). Model is shown in figure \ref{FIGURE:SIMPLEPHMM}. 
%
\begin{figure}
\begin{center}
\includegraphics{../figures/pairHMM.pdf}
\end{center}
\caption[Simple pair HMM model for alignment]{Pair hidden Markov model for
pairwise alignment. It has two transitions
parameters $e_{M,M}$ and $e_{I,I}$, since we set $e_{I,M} = 1 - e_{I,I}$ and
$e_{M,I}=\frac12-\frac12e_{M,M}$. Match state $M$ generates aligned pair of symbols
and states $I_X$ and $I_Y$ generates symbols only in $X$ or $Y$ respectively.
Initial distribution is uniform.
}\label{FIGURE:SIMPLEPHMM}
\end{figure}

%As mentioned before, score of the alignment is the probability of state path
%that correspond to such alignment. Therefore we can find the alignment with
%highest score by two dimensional version of Viterbi algorithm. Advantage of
%using this model instead of the Needleman-Wunsch algorithm is that pHMM gives
%probabilistic explanation of the alignments. 



\end{example}

\begin{definition}
Let $H=(\Sigma,V,I,d,e,a)$ be  a pHMM, $X$ and $Y$ be arbitrary sequences over
$\Sigma$ and $\pi$ be a state path. The probability that sequences $X$ and $Y$
were generated by a model $H$  with state path $\pi$ is
\begin{equation}
\Pr\left(X,Y,\pi\mid H\right)=
I_{\pi_0}
\left(
	\prod_{i=1}^{|\pi|}a_{\pi_{i-1},\pi_i}
\right)
\left(
	\prod_{i=0}^{|\pi|}e_{\pi_i,(X[D^x_{i-1}:D^x_{i}],Y[D^y_{i-1}:D^y_{i}])}
\right)
\end{equation}
If $D^x_{|\pi|-1}\not=|X|$ or $D^y_{|\pi|-1}\not=|Y|$ then
$\Pr\left(X,Y,\pi\mid H\right)=0$. 
\end{definition}

Similarly as for HMMs, we can define the probability that sequences $X$ and $Y$ were
generated by the model $H$.

\begin{definition}
Let $H=(\Sigma,V,I,e,a)$ be a pHMM and  $X$ and $Y$ be arbitrary sequences over
$\Sigma$. Then probability that sequences $X$ and $Y$ were generated together by
model $H$ is 
\begin{equation}
\Pr\left(X,Y\mid H\right)=\sum_{\pi}\Pr\left(X,Y,\pi\mid H\right)
\end{equation}
\end{definition}


\subsection{Viterbi algorithm for pair HMM}
\label{SECTION:PAIRHMMVITERBI}
Algorithms operating over pHMMs are similar to those for the regular HMMs, but
in general they have higher computational complexity because they combine
computation  over model states with sequence alignment.  In this section, we
describe two-dimensional version of the Viterbi algorithm, other algorithms are
analogous.

The Viterbi algorithm for HMMs computes variables $V[i,v]$ and $B[i,v]$. Every
variable is parametrized by a position in the sequence and a state. For
two-dimensional version, we will add position in the second sequence.

Let $V[i,j,v]$ be the probability of the most probable state path that generated
$X[:i+1]$ and $Y[:j+1]$ and ended in state $v$. Clearly, $\max_{v\in
V}V[|X|-1,|Y|-1,v]$ is the probability of the most probable state path that
generated $X$ and $Y$. Let $B[i,j,v]$ be the last but one state of the most
probable state path that generated $X[:i+1]$ and $Y[:j+1]$ and ended in state
$v$. To make it easier, we expect that all states but one are not silent -- they emit
symbol in at least one sequence. The one silent state $s$ (start state) will have $I_s=1$.
 Let $n=|X|$ and $m=|Y|$.


\begin{align}
V[-1,-1,s] &= 1\\
V[-2,i,v] &= V[j,-2,v] = 0, \forall v\in V,-1 \leq i < n, -1\leq j < m\\
V[i,j,v] &= \max_{u\in
V}V[i-d^x_{v},j-d^y_v,u]a_{u,v}e_{v,(X[i:i+d^x_v],Y[j:j+d^y_v])}\label{EQUATION:2DVITERBIF}\\
%V[-1,j,v] &= V[i,0,v] = 0 \\
%V[0,0,v] &= I_{v}e_{v,(?,?)} \\
B[i,j,v] &= \arg\max_{u\in
V}V[i-d^x_{v},j-d^y_v,u]a_{u,v}e_{v,(X[i:i+d^x_v],Y[j:j+d^y_v])}\label{EQUATION:2DVITERBIB}
\end{align}

In equations \ref{EQUATION:2DVITERBIF} and \ref{EQUATION:2DVITERBIB} boundaries for $i$ and $j$ are $
-1\leq i< n,-1\leq j< m$ and $i>-1$ or $j>-1$.


By finding the last state $v$ of the most probable state path and back-tracing
from $B[n-1,m-1,v]$ we can reconstruct the most probable state path. Time
complexity of this algorithm is $O(nm|V|^2)$ (or $O(nm(|V|+t)$ where $t$ is the
number of transitions of $H$) and memory requirements are $O(nm|V|)$. However,
we can use various tricks to decrease memory requirements of such algorithms, as
shown in the section \ref{SECTION:ALGORITHMICIMPROVEMENTS}.

\subsection{Generalized Pair HMMs}


A \abbreviation{generalized pair HMM}{GpHMM} (or pair hidden semi-Markov
process) are combination of HMM and GHMM. Every state generates two
duration lengths $d^x$ and $d^y$ from some joint distribution associated with
the current state $d_v(d^x,d^y)$ and after that it generates two strings $x'$
and $y'$ with lengths $d^x$ and $d^y$ according to the joint distribution
$e_{v,(x',y')}$. This probability distribution can by defined for example by
pair hidden Markov model.  As with GHMMs and unlike pHMMs, the state path is not
sufficient to determine which parts of the sequences were generated by which
state, we also need two sequences of durations.

Drawback of GpHMM is increased computational complexity. For example time
complexity of the Viterbi algorithm is $O(nmk^2d^4)$ where $k$
is the number of states of a pHMM and $d$ is the maximum duration length \cite{Meyer2002}.
GpHMMs were successfully used for gene-finding
\cite{SLAM2003,Alexanderson2004,Majoros2005,Meyer2002}. 

%Recall from section \ref{SECTION:PAIRHMM} that a pair HMM generates two
%sequences simultaneously and that from state path we are able to recover
%annotation.


%In following sections we will review several pHMMs and GpHMMs that were used to
%sequence alignments or other purpose, for example gene finding. We will review
%the application domain of the model, its topology, decoding function, parameter
%estimation and optimisation heuristics. But at first, we have to introduce small
%biological background.

\subsection{Decoding Methods}

In this section we review three decoding methods that were used in literature to
reconstruct pairwise alignments: the Viterbi algorithm, the Posterior
decoding and the Marginalized posterior decoding.

Let $H$ be an pHMM (or GpHMM) and $X$ and $Y$ be the sequences that we want to
align. The probability $\prob{X,Y\mid H}$ is the probability that $X$ and $Y$
were generated in model $H$.  From state path $\pi$ in a pHMM  we can
reconstruct a unique alignment $A_{\pi}$. The model defines the probability 
that
$A_{\pi}$ is the true alignment of $X$ and $Y$ under the assumption, that $X$ and $Y$ were
generated by model $H$:
  \[\prob{\pi\mid
X,Y,H}=\frac{\prob{\pi,X,Y\mid H}}{\prob{X,Y\mid H}}\]
  We can use $\prob{\pi\mid X,Y,H}$ as a score of 
alignment $A_{\pi}$. Path $\pi$ with the highest score can be found by a
two-dimensional version of the Viterbi
algorithm (section \ref{SECTION:PAIRHMMVITERBI}). Alignment $A_{\pi}$ can be
constructed from $X,Y$ and $\pi$ in a straightforward
way: for every match state from $\pi$ that generated $X[i]$ and $Y[j]$ we add
column $(X[i],Y[j])$. For every indel state in $\pi$ that generates $X[i]$ we
add to alignment column $(X[i],'-')$. An indel states for the second sequences are
analogous.  The two-dimensional Viterbi algorithm is used in  most of the
software tools we will discuss later.

Alternatively, we can decode pHMMs using a variant of Posterior decoding.  Two
variants of the Posterior decoding for the pHMMs were described in the
literature: the Posterior decoding and the Marginal posterior decoding
\cite{Lunter2008}.  Let $\prob{X[i]\sim Y[j]\mid X,Y,H}$ be the probability that
$X[i]$ and $Y[i]$ are aligned: the sum of the probabilities of all alignments
that
contain column $(X[i],Y[i])$. Let $\prob{X[i]\sim -_j\mid X,Y,H}$ be the
probability that $X[i]$ is aligned to a gap that is in $Y$ between positions $j$
and $j+1$. Similarly let $\prob{-_i\sim Y[j]\mid X,Y,H}$ be the probability that $Y[j]$
is aligned to a gap in $X$ between positions $i$ and $i+1$. Finally, let
$\prob{X[i]\sim - \mid X,Y,H}$ be the probability that $X[i]$ is aligned to a
gap at any position and let $\prob{-\sim Y[j]\mid X,Y,H}$ be the probability
that $Y[j]$ is aligned to a gap at any position.  Posterior probabilities
defined above can be computed by the two-dimensional version of the
Forward-Backward algorithm (probability that a symbol is aligned to a gap at any position
is the sum of the probabilities that symbol is aligned to a gap at position $i$
for all possible positions $i$).

Let alignment $A$ of sequences $X$ and $Y$ have length $n$ and
consists of columns $a_0,a_1,\dots,$ $a_{n-1}$. Each column is a pair
$a_i=(x_i,y_i)$ where $x_i$ and $y_i$ are symbols from $\Sigma\cup\{-\}$ \footnote{Note that $x$ and $y$ cannot be both gap symbols.}.
Let $d_A^x(i)$ be the number of non-gap symbols in $x_0,x_1,\dots x_{i}$,
let $d_A^y(i)$ be the number of non-gap symbols in $y_0,y_1,\dots, y_{i}$ and
define $d_A^x(-1)=d_A^y(-1)=0$. In this notation, $A[0:i]$ is an alignment of $X[:d_A^x(i)]$ 
and $Y[:d_A^y(i)]$. Then the posterior probability of an alignment column $a_i$ is
\[P(a_i)=
\begin{cases}
\prob{x_i\sim y_i\mid X,Y,H} & \text{if $x_i$ and $y_i$ are not gap symbols}\\
\prob{x_i\sim -_{d_A^y(i)-1}\mid X,Y,H}  & \text{if $y_i$ is gap symbol and $x_i$ not}\\
\prob{-_{d_A^x(i)-1}\sim y_i\mid X,Y,H}  & \text{if $x_i$ is gap symbol and $y_i$ not}
\end{cases}
\]

The \abbreviation{Posterior decoding}{PD} finds the alignment $A$ that maximizes
the product of the posterior probabilities of its columns: 
\[A = \arg\max_{A'\in Al(X,Y)}\prod_{0\leq i <
|A'|}P(a'_i)\] where $Al(X,Y)$ denote the set of all  alignments of sequences
$X$ and $Y$. Similarly we can define \abbreviation{Marginalized posterior
decoding}{MPD}: Let $P'(a_i)$ be the marginalized posterior probability:
\[
P'(a_i) = \begin{cases}
P(a_i) & \text{if $x_i$ and $y_i$ are not gap symbols}\\
\sum_{0\leq j < |Y|}\prob{x_i\sim -_{j}\mid X,Y,H}  & \text{if $y_i$ is gap symbol and $x_i$ not}\\
\sum_{0\leq j < |X|} \prob{-_{j}\sim y_i\mid X,Y,H}  & \text{if $x_i$ is gap symbol and $y_i$ not}
\end{cases}
\]
Then the MPD finds an alignment $A$ that maximizes the product of the
marginalized posterior probability:
\[A = \arg\max_{A\in Al(X,Y)}\prod_{0\leq i < |A'|)}P'(a'_i)\] 

The Posterior decoding and the Marginalized posterior decoding were used by
Lunter {\it et al.} and both produced better alignments than alignments found
by the Viterbi algorithm (more details in section \ref{SECTION:BIASES}). Once
the posterior probabilities of all possible columns of an
alignments are computed (in $O(|X||Y|k^2)$ time where $k$ is the number of
states of pHMM), we can find the alignment that maximizes the desired
function in $O(|X||Y|)$ time. Therefore the time complexity of the PD and MPD is 
$O(|X||Y|k^2)$ \cite{Lunter2008}. 

%Other decoding method can be done by the poster
%Other option is to
%use the  Posterior decoding: for every pairs of residues $X[i],Y[j]$ we compute
%posterior probability that $X[i]$ and $Y[j]$ are aligned $\prob{X[i]\sim Y[j]\mid X,Y,H}$.
%Score of an alignment is sum of posterior probabilities of the aligned residues.
%Later in this chapter we show different variants of the Posterior decoding for
%pHMM.


\subsection{Pair Hidden Markov Models with Gene Structures}

In this section we describe several pair hidden Markov models (or generalized
pair hidden Markov models) with gene structures incorporated into their
topology. These models were used either to align coding DNA or proteins
to a genome or to find genes.


First we introduce several comparative gene finders. Comparative gene finders
use evidence from two organisms to find genes. They use pHMM to simultaneously
align and annotate  two sequences. The advantage finding genes in two
organisms simultaneously is that we can use the evidence from two related
organisms to detect genes that are in both organisms.

\subsubsection{DoubleScan}
Meyer {\it et al. (2002)} developed comparative gene finder DoubleScan.
Generally, DoubleScan has three types of states: {\it match} states, which
generate same number of symbols in both sequences;   {\it emit} states, which
generate sequences only in one sequence and silent states.  

The basic structure of DoubleScan's GpHMM consists of three types of
substructures: substructures that emits exons, substructure that emits introns
and substructure that generate intergenic regions. Each structure has three
copies in the model: one for match states and two for emit states. Exon
substructure is a single state emitting codons (triplets that will be translated
into amino acids).  There is one additional intron substructure connected to
states that generate intergenic regions. This additional intron substructure is
in the model to avoid detecting of pseudogenes. The GpHMM has $54$ states.

%Every such
%structure has three version: one matching version, which emit aligned residues
%and one structure per sequence to emit indels.  Overview of DoubleScan's HMM
%structure is in figure \ref{}. 
\nocite{Meyer2002}

Emission probabilities of the {\it match exon} state were estimated from
relative frequencies in the training set with Dirichlet priors
\cite{Meyer2002,Durbin1998}.  Emission probabilities of other states were
generated  by marginalizing emissions of the match exon state. Transition
probabilities from the initial state were uniform, transition probabilities for
splice sites were estimated by splice site predictor. Other transitions were
observed from training data and tuned by hand.

DoubleScan uses the Viterbi algorithm as a decoding method.  To reduce the
running time of the Viterbi algorithm they use stepping-stone algorithm: first
they run BLASTN to find local alignments. Then DoubleScan chooses a consistent
subset of alignments by the greedy method described in section
\ref{SECTION:SSA}. They restrict the Viterbi algorithm to follow the alignments
in the subset allowing tolerance 15 bases.

\subsubsection{SLAM} 

\begin{figure}
\begin{center}
\includegraphics{../figures/slam.pdf}
\end{center}
\caption[HMM topology of SLAM's GpHMM]{
Topology of GpHMM used by SLAM (states for genes on reverse strand are omitted).
Gray states have geometric distribution (modeled with self-loops which are
omitted in the figure). Emissions of shaded states are modeled by the basic
three state pHMM. White states represent exons. Each has associated a duration
distribution and emissions are also modeled by three-state pHMM using $5$-th
order states that emit whole codons at once.  Since introns can be inside a
codon, the model contains an exon state for every possible interruption:
$E_{i,j},0\leq i,j<3$ is an exon that begins with end of the interrupted codon of
length $((3-i)\mod 3)$ and ends with the start of a codon of length $j$. $I$
stands for start codon and $F$ stand for stop codon: states
$E_{I,\cdot},E_{\cdot,F}$ and $E_{I,F}$ model exons adjacent to the beginning
and the end of a gene.  $Intron_i$ models intron that interrupted codon at the
$i$-th position ($0$ means that intron is not interrupting any codon).
}\label{FIGURE:SLAM} \end{figure}

SLAM \cite{SLAM2003}  is a comparative gene finder based on a generalized pair
hidden Markov model \cite{Alexanderson2004} with some states being also
high-order states (with dependence on previous emissions).  It predicts gene
structures for a pair of related eukaryotic organisms. SLAM's decoding method is
the Viterbi algorithm. 

Unlike DoubleScan, SLAM defines a true GpHMM because state durations are not
constant, but are sampled from some distribution (however, distribution they
used is not specified). Topology of the model can be found in figure
\ref{FIGURE:SLAM}.  Emission probabilities of pairs of codons were assigned from
a codon-based PAM matrix.


To reduce the running time of the algorithm first they  align the input
sequences using AVID global alignment tool \cite{Bray2003}. They restrict the
Viterbi algorithm so, that each base matched in the anchor alignment can be
realigned in an interval of size $3$.

\subsubsection{TWAIN}

TWAIN is another approach to use GpHMM to find genes \cite{Majoros2005} and is
interesting because of its decoding algorithm. Twain GpHMM has two types of
states: the states with fixed durations (which emit sequences of fixed length)
and the states with variable duration lengths which are associated with some
distribution.  The states with fixed durations corresponds to specific
signals, such as splice sites, start/stop codons, and so on.
Additionally, the states with variable duration are not connected with each
other; the transitions from/to such states are only to/from states with fixed
duration lengths. Therefore if we know the positions of all signals in the
sequence, we know that sequences between signals were generated by one state.
This can be exploited in the following way.

TWAIN at first annotates input sequences using gene-finder TIGRscan
\cite{Majoros2004} (each sequence is annotated independently) which finds
signals in the sequences and creates a parse graph: vertices of the parse graph are
the signals in the sequence (each signal corresponds to a state of the GpHMM).  Two
signals are connected with an edge if in the genome one signal can follow another:
for example a start codon can be followed by a stop codon or a donor site.
Therefore start codons are connected with following stop codons and
donor sites. Each edge is scored by the probability of the most probable state
path between those two states in the TIGRscan's HMM. To reduce the size of the
graph, some edges are omitted by heuristic.

TWAIN creates graph $G$ by cartesian product of two parse graphs and omits the
vertices that corresponds to pairs of different signals since they are unlikely
to be seen in the alignments. Each node in the graph corresponds to some state $s$
of the GpHMM and cell $c$ in the dynamic programming matrix of the Viterbi
algorithm.   Edges between cells $c_1,c_2$ correspond to alignments
generated by a single state from GpHMM (since states for $c_1$ and $c_2$ are
connected through generalized states).  The Viterbi algorithm is computed only
on cells corresponding to graph $G$ which significantly reduces running time
\cite{Majoros2005}.


\subsubsection{GeneWise}

GeneWise predict genes by aligning a protein  to similar gene structures in DNA
\cite{GeneWise2004}. Instead of pHMM defined as in section \ref{SECTION:PAIRHMM}
it uses probabilistic transducers. Probabilistic transducers are very similar to
pHMM. They are both probabilistic finite state machines, but unlike pHMMs which
generate two sequences, probabilistic transducers transform one sequence into
the other sequence.  The second difference is that transducers have emissions
associated with transitions, not with states.  Therefore transducer emission of
the form $e_{u\to v,(x,y)}=p$ means that during transition $u\to v$ symbol $x$
is read from the input sequence and symbol $y$ is written to the output with
probability $p$ ($x$ or $y$ might also be the gap symbol).  While pHMMs defines
distribution $\prob{X,Y\mid H}$, probabilistic transducers define distribution
$\prob{X\mid Y,H}$, the probability that sequence $Y$ will be transformed to
sequence $X$.

Advantage of transducers over the pHMM is that we can easily compose multiple
transducers together, while maintaining probabilistic interpretation of the
resulting model.  Composition of two transducers $A$ and $B$ is transducer $C$
that transforms sequence $X$ to some sequence $Y$ using transducer $A$ and then
transforms $Y$ to sequence $Z$ using transducer $B$. 

GeneWise model was created by composition of a gene prediction model $S$ which
translates genomic sequence to protein sequence and a protein homology model $T$
which maps protein sequence to a homologous protein sequence.

Gene prediction model $S$ consists of a single exon state which translates
series of codons into amino-acids. It has three submodels for modeling introns,
each consisting of $3$ states representing  splice sites, poly-pyrimidine tract
and central intron section (those $3$ states are associated with $4$
transitions, each transition emit one feature). 

Protein homology model $T$ is a simple pHMM from figure \ref{FIGURE:SIMPLEPHMM}
defined over protein alphabet.  The composition of the models $T$ and $S$ has
$30$ states. Authors removed unnecessary states to reduce the number of states
and transitions \cite{GeneWise2004}.


\subsubsection{Pairagon}

\begin{figure}
\begin{center}
\includegraphics{../figures/pairagon.pdf}
\end{center}
\caption[Topology of Pairagon generalized pair hidden Markov model.]{
Topology of Pairagon GpHMM. All states except states
$D,B$ and $A$ have a self-transition.
Shaded states corresponds to exons: $M$ emits
aligned pairs of symbols, $I_c$ is insertion in the cDNA and $I_g$ is
insertion in the genome. States $I^1_c,I^1_g,I^2_c$ and $I^2_g$ corresponds to
unaligned parts of the DNA and cDNA in the beginning and the end of the
sequences. States $D,In,B,BA,A$ correspond to intron structure and stand for 
Donor, Intron, Branch, Branch Acceptor and Acceptor respectively.
Donor site emits $8$ symbols and acceptor site emits $6$ symbols.
}\label{FIGURE:PAIRAGON}
\end{figure}


The aim of Pairagon is to find local alignments of \abbreviation{complementary
DNA}{cDNA} and genome \cite{Pairagon2009}. By aligning experimentally obtained
cDNA sequences to the genome we are able to confirm intron and exon structures
of genes.  Pairagon's HMM model consists of a simple pair HMM submodel, which
aligns cDNA to DNA and a $5$-state submodel for intron structures. The whole
topology is in figure \ref{FIGURE:PAIRAGON}. 

Model was trained using iterative maximum likelihood approach.  Initial
parameters were trained on the alignments from the \abbreviation{Mammalian Gene
Collection}{MGC}. In this phase, the parameters for intron submodel were set by
hand. The model was then used to align more MGC sequences. Final parameters were
estimated from the new alignments.

Decoding was done by the Viterbi algorithm. Runtime of the algorithm was
improved by the stepping-stone algorithm described in section \ref{SECTION:SSA}
and memory requirements were improved using the Treeterbi algorithm
\cite{Keibler2007}, which is similar to the On-line Viterbi algorithm discussed
in section \ref{SECTION:ONLINEVITERBI}.

\subsection{Alex Hudek -- dizertacka}

\subsection{Non-Geometric Indel Models}
In the simple pHMM described in figure \ref{FIGURE:SIMPLEPHMM}, gap length has
geometric distribution: the probability that a gap has length $n$ is
$e_{M,I}e_{I,I}^{n-1}(1-e_{I,I})$ (note that the probability that at particular
position will be gap with length zero is $1-e_{M,I}$). The Viterbi
algorithm is usually computed in log-space: instead of computing product of
probabilities of events\footnote{Event is emission or transition.}, we compute
the sum of logarithms of those probabilities, because computation in log-space
is numerically more stable. The Viterbi algorithm for the simple HMM
will become the same as the Needleman-Wunsch algorithm.  Gap penalty will be
$\log(e_{M,I})+\log(1-e_{I,I})+(n-1)\log(e_{I,I})$. By setting $d=\log(e_{I,I})$
and $g=\log(e_{M,I})+\log(1-e_{I,I})-d$ we see that this is exactly affine gap
penalty. Therefore we can say that affine gap penalties correspond to geometric
distribution of indel lengths.

As we mentioned in chapter \ref{CHAPTER:ALIGNMENT}, using non-affine gap model
can improve alignment quality.  Problem with geometric distribution (or affine
gap penalties) it that they are not realistic \cite{Cartwright2009,Lunter2008}.
Therefore some other distribution might be more appropriate, for example zeta
distribution \cite{Cartwright2009}, or combination of several geometric
distributions to approximate the distribution of gap length
\cite{Gill2004,Gill2006}.

GpHMM allow us to use arbitrary duration distributions.  On the other hand,
GpHMM are much slower to decode.  One way of incorporating a different gap
distribution into pair hidden Markov models without using their generalized
version is to use several (for example two) indel states for every sequence. For
example Lunter {\it et al. (2008)} used two component mixture models: instead of
one indel state for every sequence they use two. They report that this improved
quality of alignments. Similar approach is used in the multiple sequence aligner
FASTA \cite{Bradley2009}. \nocite{Lunter2008}

Modeling non-geometric distributions with several states can be problematic when
used with the Viterbi decoding \cite{Vinar2005}. Set of states with the same
emission distribution used for modeling non-geometric distribution is called
gadget. We discuss an example of such a problem for the two component mixture
model.

\begin{figure}
\begin{center}
\includegraphics{../figures/twoComponentMixtureModel.pdf}
\end{center}
\caption[The example of an HMM modeling two component geometric distribution]{
Shaded state $M$ represents the match state and states $I_1$ and $I_2$
represents indel states in the same sequence. Indel states for the other
sequence are omitted.
}\label{FIGURE:TWOCOMPONENT}
\end{figure}

Let $H$ be a simple pair hidden Markov model with two pairs of indel states.
Let $I_1$ and $I_2$ be indel states that generate gaps in the first sequence
connected with match state $M$ as in the figure \ref{FIGURE:TWOCOMPONENT}.  Gaps
in alignments (in the first sequence) that are generated by such a model have
length distribution $d(n)=(a_1p_1^{n-1}(1-p_1)+a_2p_2^{n-1}(1-p_2)), n>0$ and
$d(0) = 1-a_1 - a_2$ where $n$ is the length of the gap ($n=0$ means that there
is no gap), $a_1$ and $a_2$ are probabilities of entering state $I_1$ and $I_2$
respectively and $p_1$ and $p_2$ are probabilities of remaining in state $I_1$
and $I_2$ respectively.  This is equivalent to the generalized pair hidden
Markov model $H'$ with one indel state for every sequence which has
$d'(n)=d(n)/(1-d(0))$ as its duration distribution (in generalized states we
want to generate at least one gap. The $d(0)$ should by modeled by the
probabilities of incoming and outgoing transitions to the generalized state). Both models define the same
distribution of alignments  and running the Forward-Backward algorithm or the
Forward algorithm will give the same results. However, alignments constructed by
the Viterbi algorithm can be different for $H$ and $H'$.

Problem is that in the generalized model the Viterbi algorithm gaps of length
$n$ have ``score'' $d(n)$ but in the non-generalized pair hidden Markov model it
will be $m(n)=\max\{a_1p_1^{n-1}(1-p_1),a_2p_2^{n-1}(1-p_2)\}, n>0$ and
$m(0)=1-a_1-a_2$.  These two scores are different ($d(n)$ is always higher) and
therefore it is possible that the Viterbi algorithm reconstruct different
alignments. Therefore if we are using the Viterbi algorithm, we should either
construct a gadget so that $m'(n)$ will be a better approximation of $d(n)$
($d(n)$ is the distribution for the original model) or use a generalized pair
hidden Markov model for the Viterbi algorithm.

%co chcem povedat: niekedy je lepsie pouzit iny gapmodel -- jeden pre kratke
%gapy, jeden pre slhe gapy. Preto sa niekedy 

\subsection{Aligning Sequences with Variable Rate of Evolution}
\label{SECTION:FEAST} 
%\subsection{FEAST} 

The rate of evolution (the expected number of substitution per position in
sequence over some period of time) is not constant for the whole genome. It does
not have to be constant even within one gene. FEAST is pairwise local alignment
tool \cite{FEAST2011} that takes into account the variable evolution rate. The
simple pHMM from figure \ref{FIGURE:SIMPLEPHMM} is optimized for one fixed rate
of evolution.  FEAST contain $k$ such submodels, each trained for a different
rate of evolution.  Submodels are connected with a single silent state.  Since
FEAST is a local alignment tool, it also contains one additional submodel for
generating an unaligned pair of  sequences  at both ends of the sequences.

To construct an alignment (either local or global) FEAST uses the Viterbi
algorithm. Like many local aligners, FEAST uses a seeding heuristics to reduce
computational complexity of finding local alignments.  At first it uses six
different space seeds to get candidate seed and then extends those seeds using
x-drop heuristic \cite{Altschul1997}. The extension is done by an ungapped
version of the Forward algorithm, in contrast with the Viterbi algorithm usually
used for this purpose. 

Estimation of parameters was done by expectation maximization approach (with
Baum-Welsch or Viterbi training). They forced gap parameters to be the same in all
submodels.

Different rates of evolution were also used  in the whole genome aligner GRAPe
\cite{Satija2010}. GRAPe's HMM  consists of two submodels: one with fast
evolution rate and two component geometric mixture model for indels and one with
the lower evolution rate and geometric distribution of indel lengths. GRAPe uses
the Posterior decoding as a decoding method.


\subsection{Biases In Alignments}
\label{SECTION:BIASES}
Lunter {\it et al. (2008)} conducted an extensive survey concerning biases in
alignment. They considered three types of biases associated with gaps. These gap
biases are also discussed in \cite{Durbin1998}. By \firstUseOf{true alignment}
we mean an alignment that corresponds to the actual evolution history.  Since
true alignments are unknown for real data we can simulate evolution on randomly
generated sequences, this obtaining a dataset of ``true'' alignments.

\begin{itemize}

\item \firstUseOf{Gap wander} means that a gap is in a different location that in
the true alignment. Is is due to random short similarities around the borders of
gaps that are indistinguishable from true homologies.

\item \firstUseOf{Gap attraction} is occurs when two gaps are near each other.
In such case merging those gaps and introducing a few mismatches might lead to
higher score. 

\item \firstUseOf{Gap anihilation} occurs when there are two gaps of the same
length, one in each both sequence. Since indels are not so common, removing both
gaps while introducing new mismatches might increase the score of an alignment.

\end{itemize}


Biases are ordered by their frequency from the most occurring to the least
occurring \cite{Lunter2008}. Lunter {\it et al.} explore there problems with a
series of simulations.

They measure the alignment quality by \firstUseOf{sensitivity}, which is the
ratio of the correctly predicted alignment columns to all homologous columns in
the true alignment \cite{Lunter2008}. 

In the first experiment, authors use a simple model of evolution obtaining
alignments with the expected sequence identity $0.705$ with geometric gap model.
The sequences were then realigned using the Viterbi algorithm with the same
model as was used for simulation. Sensitivity was lowest for the  columns near
gaps ($56$\%) and the sequence identity for columns near gaps was $85$\% which
does not agree with the expected sequence identity $0.705$\%.  Moving away from
gaps the average sequence identity dropped to  $0.68$\%. The increased sequence
identity near gaps is due to gap wander bias. The gap attraction effect caused
that the number of gaps that are  near each other was lower than the expected
value obtained from the used gap model.

They also run the Viterbi algorithm parametrized by a range of substitution and
indel rates. The highest sensitivity was obtained for the parameters that were
identical to the parameters used for simulation. However, even then the
sensitivity was only $84$\% indicating that even if we have the right evolution
models, some biases in the alignments are inevitable.  

Lunter {\it et al.} also studied the effect of different decoding methods and
different models on the alignment quality. They simulated evolution with
parameters that are close to the parameters of human-mouse evolution. They
simulated for example the large-scale variation of GC content, GC-content
dependent indel and substitution rates and GC-independent local substitution
rate variation \cite{Lunter2008}.  From simulation they obtained $20,000$
homologous sequence pairs with average length of $700$ nucleotides. They add
flanking sequences of length $100$ nucleotides to the generated sequences  to
simulate local alignments.

After simulation they realigned homologous sequences using the Viterbi algorithm
(VA), the Posterior decoding (PD) and the Marginalized posterior decoding (MPD)
with different models: the three state pHMM ($H_S$); $H_S$ with two indel states
for
every sequence  representing the two component geometric mixture gap model ($H_M$) and the
full model with all parameters that were used for simulation ($H_F$).

They also introduce two additional measures of the alignment quality. The
\firstUseOf{false positive fraction (FPF)} is the proportion of the columns that
are ungapped in the true alignment but wrongly aligned by an algorithm
\cite{Lunter2008}. The \firstUseOf{the nonhomologous fraction (NHF)} is the
proportion of columns containing padding sequence among all columns aligned by
an algorithm.


The use of the different models has little impact on the sensitivity of the
constructed alignments. It is interesting that for the Viterbi algorithm the
sensitivity was lower for the full $H_3$ model than for the simple model $H_1$.
This might be explained by the multiple path problem.  With other decoding
algorithms the models $H_2$ and $H_3$ has slightly higher sensitivity then the
$H_1$ model. 

While the use of the ``better'' model does not significantly improve the quality
of alignments, using the Posterior decoding and Marginalized Posterior decoding
improved the sensitivity by approximately $2.5$\% regardless of the model. On
the other hand the FPF and the NHF was increased with use of the PD and MPD. The
sensitivity of the PD and the MPD were similar but the FPF was lower for the MPD
than for PD. 

The main outcome of this experiment is that proper decoding method can improve
the alignment quality while the use of a simpler model doest not significantly
reduce the alignment quality. However, Lunter {\it et al.} use in their
simulations models only relatively simple models of the evolution. By
incorporating information about gene stricture into alignment models combined
with the use of the right decoding algorithm we can improve alignments further.
This will be discussed in the next chapter. 



\section{Algorithmic improvements}

Needleman-Wunsch algorithm and decoding algorithm for HMMs and pHMMs uses
dynamic programming. In this section we review several algorithmic improvements
to these algorithms. Some of the techniques will not be universally applicable.
With these techniques, alignment algorithms could be used with sequences much
longer than several thousand bases.

%effectively and describe several methods how to decrease memory footprint and/or
%time complexity. 

%As in previous case, values of $M,M_X$ and $M_Y$ can be computed by simple
%dynamic programming and by back-tracing these matrices we can obtain optimal alignment of
%$X$ and $Y$.  Both algorithms have time complexity of $O(nm)$ and memory
%complexity $O(mn)$.  Using tricks described later, the memory and even time
%complexity will be improved.

\begin{comment}
In this section, we review several techniques that can be used to improve
performance of dynamic programming used to compute sequence alignment.  This is
important, because algorithms described above can be used with sequences that
are several thousands bases long. Full genomes can be several billions bases
long and using naive approach is not tractable. 
\end{comment}

\subsection{Restricting Search Space}

One commonly used technique for speedup (and decreasing memory requirements) of
sequence alignment is to restrict the search space of dynamic programming. We
compute alignment only in some parts of matrix $M$ and we assume that omitted
parts of matrix correspond to alignments with low score. These techniques are also 
applicable for pHMMs.

If the two sequences are quite similar, the optimal global alignment will not
be too far from the main diagonal of matrix $M$. Therefore it is not necessary
to compute parts of matrix $M$ that are too far from the main diagonal
\cite{Chao1992} (distance from diagonal a is user-defined value or can be
computed during alignment \cite{GusfieldBook}).  However, this method is not
useful for local alignment or global alignment of distant sequences. Now we
will discuss some more advanced techniques to restricting search space of
dynamic programming.

\subsubsection{Seeds}

Seeds are a technique frequently  used to reduce time complexity of local
alignment.  They were popularized by BLAST algorithm \cite{Altschul1990}.  A
\firstUseOf{seed} is a short alignment which is likely to be a part of an
alignment with high score. After a seed is found, it is extended with an
extension algorithm to a local alignment.

Seeds that cannot be extended to high-scoring alignments are discarded. Such
candidate seeds are called false positives.  Alignments that were not found by
our heuristics are called false negatives.  It is important that heuristics
used to find seeds has a small number of false negatives and a large number of
true positives, otherwise many true high-scoring local alignments will not be
found. On the other hand, high number of false positives implies longer running
time. 


The most traditional approach is to take as seeds all pairs of positions $i$
and $j$, such that $X[i:i+\tau]=Y[j:j+\tau]$ for some
constant $\tau$. This approach is used in BLAST \cite{Altschul1990}.  Various
generalization were developed to improve trade off between
speed and accuracy, such as seeds with mismatches \cite{Kent2002}, space seeds
\cite{Ma2002}, vector seeds \cite{Brejova2005vector} or daughter seeds
\cite{Csuros2005}.

Extension of a seed to a full alignment is done in both directions, usually
using equation \ref{ALIGN:ALGO:AFFINE} (equation is altered to reverse
direction). Extension is stopped, when some criterion is reached. For example,
BLAST introduced X-drop heuristics: extension stops if the score of an alignment
is lower than the best score that was seen so far minus some user-defined
constant \cite{Altschul1997}.

\subsubsection{Stepping-Stone Algorithm}
\label{SECTION:SSA}

\abbreviation{Stepping-stone algorithm}{SSA} 
\cite{Meyer2002,Pairagon2009} is suitable for global alignment algorithms. The idea is to
use good local alignments as anchors. An anchor is similar to a seed: it is a
short alignment which we expect to be in the optimal global alignment.
However, local alignment tools give local alignments that do not have to be
consistent with each other. A set of local alignments is consistent if all
local alignments can be together in one global alignment.
\begin{comment}
Let $A$ and $B$ be a
local alignments of sequences $X$ and $Y$ and $(i,j)$ be the position of the
last alignment column of $A$ in the dynamic programming matrix and $(k,l)$ be
the position of the first alignment column of $B$ in the dynamic programming
matrix. Then $A$ and $B$ are consistent if $i<k$ and $j<l$.  Therefore we need
to choose a consistent subset of alignments.
\end{comment}

SSA chooses a consistent set of local alignments  by a simple greedy method,
always adding highest-scoring alignment consistent with those selected so far.
Selected anchors will be extended to a global alignment. However, since local
alignments may contain errors, SSA will relax them. If $X_i$ and $Y_j$ were
aligned in some anchor, then $X_i$ can be aligned to positions from $j-\tau$ to
$j+\tau$ in the global alignment\footnote{Or aligned to a gap in that region.}
for some user defined constant $\tau$. Similarly, $Y_j$ can be aligned to
positions from $i-\tau$ to $i+\tau$. This technique is also called \firstUseOf{banding},
and it is often used. We used this technique in chapter \ref{TODO}.

\begin{comment}
Time complexity and memory requirements of the stepping-stone algorithm are of
order of $O(\sqrt{|X|^2+|Y|^2})$ under the assumption that the maximum area between
two anchors is independent of the sequence length \cite{Meyer2002}.
\end{comment}

\subsection{Reducing memory complexity}
\begin{comment}
Checkpointing is a general technique in dynamic programming, which allows reduce
the number of rows of the dynamic programming matrix to $O(\sqrt n)$ while
doubling the running time
\cite{Grice1997}. 
\end{comment}
One general technique to reduce the memory requirements of dynamic programing is \firstUseOf{checkpointing} \cite{Grice1997}.

In order to compute the $i$-th row of matrix $M$, we need only row $i-1$. As
mentioned in section \ref{SECTION:NEEDLE}, to compute the score of the best
alignment we need to store only
two consecutive rows.  However, if we want to
recover the optimal alignment, after we have computed its score, we need all
rows of matrix $M$ again, in decreasing order.

Checkpointing solves this problem by storing every $k$-th row of matrix $M$,
$\lceil n/k\rceil$ rows in total.  While back-tracing, we will remember an
additional block $B$ of consecutive $k$ rows in interval $[ik,(i+1)k)$. We can
compute such a block in $O(kn)$ time using the basic dynamic programming,
starting from row number $ik$ which is stored in memory.  Overall, we recompute
every block at most once, and therefore the time complexity will be $O(mn)$.
If we set $k=\sqrt n$ then the memory complexity will be $O(m\sqrt n)$. By
using $l$ recursive applications of this technique we can obtain algorithm with
memory complexity $O(m\sqrt[l]{ n})$.

Even more space reduction can be obtained using the Hirschberg algorithm \cite{Hirschberg1975}. The idea is
the following: if we want an alignment of sequences $X$ and $Y$ that has bases $X_i$
and $Y_j$ aligned, we have to  do dynamic programming only in submatrices
$M_1=M[0:i,0:j]$ and $M_2=M[i:n-1,j:m-1]$. If $i=\lceil n/2\rceil$ then
the total number of cells in those matrices is roughly half of the number of
cells in $M$. The Hirschberg algorithm incorporates a procedure to compute 
$j$ such that $X_i, i=\lceil n/2\rceil$ is aligned to $Y_j$ in optimal alignment of both sequences in $O(nm)$
time and $O(n+m)$ memory. If $X_i$ is aligned to a gap, it finds $j$ such
that $X_i$ is aligned to gap that comes right after $X_j$.
The algorithm then uses such $j$ to determine submatrices $M_1,M_2$ and finds alignments in them recursively.
The optimal alignment is a concatenation of optimal alignments in matrices $M_1$ and
$M_2$.

\begin{comment}
To determine $j$, such that $X_i,i=\lceil n/2\rceil$ is aligned to $Y_j$ (or
$X_i$ is aligned to gap right after $Y_j$) HA uses following algorithm: Let
$B(X,Y)$ be the algorithm, that computes for $X$ and $Y$ vector $LL(k)$, where
$LL(k)=M[n-1,k]$ ($LL$ is the last row of $M$). This can be computed in
$O(nm)$ time and $O(n+m)$ memory using algorithm from section
\ref{SECTION:NEEDLE}.  

We compute a vector $LL_1=B(X[0:i],Y)$. $LL_1[k]$ contains score of the optimal
alignment of $X[0:i]$ and $Y[0,k]$. Let $LL_2[k]$ contains score of the optimal
alignment of $X[i,n]$ and $Y[k,m]$. Then $LL_2=B( (X[i,n])^R,Y^R)^R$.
%which is the last row of the 
%dynamic programming matrix applied to the left part of matrix $M$, and $LL_2=B(
%(X[i,n])^R,Y^R)$, which is the last
%.
%While $LL_1[k]$ contains score of optimal alignment of $X[0:i]$ and $Y[0:k]$,
%$LL_2^R[k]$ contains score of optimal alignment of $X[i,n]$ and $Y[k,m]$.
Searched column $j$ is column that maximizes $\max\{LL_1[j]+d+LL_2[j],
LL_1[j-1]+LL_2[j] \}$.
\end{comment}

Since total size of the two subproblems is always at most half of the size of
the original problem, the running time of the algorithm will be roughly double
of the standard dynamic programming.  The Hirschberg algorithm keeps in memory
only a constant number of rows of $M$,  and therefore the memory requirements
are $O(n+m)$. The Hirschberg algorithm reduces memory more than checkpointing,
but checkpointing can be applied to a wider class of algorithms. 
Checkpointing can be used to decrease the memory complexity of the Viterbi
algorithm with back-tracing procedure and the Posterior decoding to $O(\sqrt n
m)$.  Similarly, this technique can be used to reduce memory complexity of the
algorithms for pHMMs to $O(nm\sqrt n )$ where $n$ is the length of the longer
sequences and $m$ is the number of states of HMM.

\subsubsection{The Viterbi Algorithm}
As we described in section \ref{SECTION:VITERBI}, in the Viterbi algorithm we compute values
$V[i, v]$ (the probability of the most probable state path ending in state $v$
and generating the prefix of the input sequence of length $i$) and $B[i, v]$
containing the previous state in the most probable state path, which is used in
back-tracing.

While back-tracing, we need access to values of matrix $B$ in decreasing order.
We need only matrix $B$ and last row of $F$. From $i$-th row of $F$ we can
recompute $i+1$-th row of $B$. Therefore we will remember every $k$-th row of
$V$ from which we will recompute blocks of matrix $B$. $B$ will be split into
blocks $B_0=B[1:k+1,:],B_1=[k+1:2k+2,:],\dots,B_i=B[ik+1:{i+1}k+1,:],\dots$ We
will keep in the memory exactly one block of $B$. If we need row that is in
block $B_i$ but $B_i$ is not in memory, we discard current block and compute
block $B_i$ from $V[ik,\cdot]$. Since we need rows of $B$ in decreasing order,
we recompute every block exactly once. If $k=\theta(\sqrt n)$ then the memory
complexity is  $O(n+m\sqrt n)$.

For two dimensional version we keep every $k$-th matrix $V[ik,\cdot,\cdot]$.
Algorithm is analogous to one-dimensional version or to the checkpointing
technique for the Needleman-Wunsch algorithm


\subsubsection{The Posterior Decoding}

In the Posterior decoding need to interlace computations of $F[i,v]$ and
$B[i,v]$ to compute $F[i,v]\cdot B[i,v]$ for all $0\leq i<n,v\in V$. We will
compute $F$ using the checkpointing technique (we compute every $k$-th row of
$F$ and keep in memory only one block).  $B[i,v]$ will be computed row by row
with the version of the  Backward algorithm that need only $O(m)$ memory. When
we compute $B[i,v]$, we also compute $F[i,v]$ by checkpointing technique (if $i$ is in
current block then return $F[i,v]$ from current block. Otherwise discard current
block and recompute the block for row $i$). With this approach we can
compute the posterior values with the $O(m\sqrt n)$ memory. We recompute every row at most
once and therefore the time complexity is $O(n(m+t))$.

\subsection{Exploiting Sequence Repetition}

This technique reduce time complexity of alignment algorithm to $O(n^2/\log n)$
This idea combines LZ78 factorization \cite{Lempel1976} and
$O(A+B)$ algorithm for computing row minima/maxima in totally a monotone matrix of
size $A\times B$ \cite{Aggarwal1987}. We will discuss only two main ideas behind
this algorithm. 
%This technique is the extension of the four-russian trick to
%unrestricted cost matrices. Four-russian trick use clever precomputation to
%speed up sequence alignment \cite{GusfieldBook}

\subsubsection{Totally Monotone Matrices}
\todo{Vyrazne skratit + spomenut 4 rusov ako originalnu techniku. + este by sa oplatilo spomenut tie rozne gap funkcie, aspon citacie na ne by to bolo treba}

\begin{definition}\cite{Crochemore2002}
A Matrix $M$ of size $n\times m$ is \firstUseOf{totally monotone} (with concave condition),
if and only if for all $0\leq i,j< n, 0\leq k<l<m$ the following condition holds:
if $M[i,k]\leq M[j,k]$ then $M[i,l]\leq M[j,l]$.
\end{definition}

If $M$ is totally monotone, then we can compute maximum of every row 
in $O(n+m)$ time by SMAWK algorithm \cite{Aggarwal1987} (This problem is called
\firstUseOf{row maxima}).

\begin{definition}\cite{Crochemore2002}
Let $M$ be a matrix and $M'$ be its
submatrix. \firstUseOf{Input border} $I_{M'}$ of $M'$ is the left column and top
row of $M'$ and \firstUseOf{output border} $O_{M'}$ of $M'$ is the right column and
bottom row of $M'$. Elements of $I_{M'}$ are ordered in clockwise direction
and elements of $O_{M'}$ are ordered in counter-clockwise direction.
\end{definition}

Intuition behind input and output border is following. If we look on the
computation of alignment algorithm inside submatrix $M'$, input border is input
to this computation and output border is output of this computation.

\begin{definition}\cite{Crochemore2002}
Let $M'$ be submatrix of $M$, $I_{M'}=\{i_0,i_1,\dots,i_{k-1}\}$ be its input
border, and $O_{M'}=\{o_0,o_1,\dots,o_{l-1}\}$ be its output border. Then
matrices
$DIST$ and $OUT$ of size $k\times l$ are defined in following way:
$DIST[a,b]$ is the cost of optimal alignment from cell $i_a$ to cell $o_b$.
$OUT[a,b]=i_a+DIST[a,b]$.
\end{definition}

Clearly, $o_b=\max_{0\leq a < k}OUT[a,b]$. Therefore by computing row maxima in
matrix $OUT$, we can compute values of output borders. Matrices $DIST$ and $OUT$
are totally monotone \cite{Crochemore2002}.  If we have matrices $DIST$ and
$OUT$ in advance and $M'$ is of size $m'\times n'$ then we can compute values of
output border in time $O(n'+m')$ by the SMAWK algorithm.

Now we show how to divide matrix $M$ into several submatrices so that
it is possible to efficiently represent matrices $DIST$ and $OUT$.

%Let $M$ be dynamic programming matrix for alignment algorithm for some sequences
%$X$ and $Y$. Let $M' =
%M[i:j,k:l]$ be rectangular submatrix of $M$. Values of $M'$ depends on 
%values cells in submatrices $M[i-1,j:l]$ and $M[i:k,j-1]$ and on value
%$M[i-1,j-1]$. Let $I_{M'}$ be the set of such cells. Let $O_{M'}$ be the 
%the set of cells in $M[i-1,j:l]$ and $M[i:k,j-1]$ (the cells in the top row or
%right column of $M'$). Let $I_0,I_1,\dots$ be enumeration of cells in $I_{M'}$
%and $O_0,O_1,\dots$ be enumeration of cells in $O_{M'}$.  Now we will construct
%the  matrix $DIST_{M'}$ of size $|I_{M'}|\times|O_{M'}|$ where  
%$DIST_{M'}[i,j]$ is the cost of optimal alignment from cell $I_i$ to cell $O_j$.
%Let $OUT[i,j]=I_i+DIST[i,j]$. Both $DITS$ and $OUT$ are totally monotone with
%convex condition 


\subsubsection{LZ78 factorization}

LZ78 is a compression algorithm that uses a dynamic dictionary, which is being built
while the sequence is compressed \cite{Lempel1976}. The way how it parses the sequence
can be used to accelerate dynamic programming \cite{Crochemore2002,Weimann2009}. 
LZ78 factorization divides
sequence $S$ into $k$ strings $S_0,\dots,S_{k-1}$, where $S_0S_2\dots S_{k-1}=S$ and
for every index  $i,0< i <k$ there is index $0\leq j<i$ such that $S_j$ is
prefix of $S_i$ of size $|S_i|-1$. We will call $S_j$ to be a
\firstUseOf{predecessor} of $S_i$.  We have guarantee that the number of strings
$k$ is in  $O(\frac{n}{\log n})$ where $n$ is the length of $S$
\cite{Lempel1976}. 

To align sequences $X$ and $Y$, we factorize them into sequences of strings
$\{X_i\}_{0\leq i < k_x}$ and $\{Y_i\}_{0\leq i<k_y}$.  Every pair $(X_i,Y_j)$
defines a rectangular submatrix $B_{i,j}$ of $M$.  All $B_{i,j}$ are
disjoint and all blocks cover matrix $M$. We say that block $B_{k,l}$ is
a predecessor of block $B_{i,j}$ if one of the following conditions is true:

%If we want to align sequences $S$ and $T$, we factorize $S$ and $T$
%into sequences of strings $\{S_i\}_{0\leq i < k_s}$ and $\{T_i\}_{0\leq i<
%k_t}$. Every pair $(S_i,T_j)$ defines rectangular block of matrix $M$.
%We say that $(S_k,T_l)$ is predecessor of $(S_i,T_j)$ if one of the following
%conditions is true:

\begin{itemize}
\item $X_k$ is predecessor of $X_i$ and $l=j$
\item $k=i$ and $Y_l$ is predecessor of $Y_j$
\item $X_k$ is predecessor of $X_i$ and $Y_l$ is predecessor of $Y_j$
\end{itemize}

Note that every block has $0,1$ or $3$ predecessors. Only block
$B_{0,0}$ has $0$ predecessors and only blocks $B_{i,0}$ and $B_{0,i}$
($i>0$) have only $1$ predecessor.  

%\todo{Mozno by nebolo odveci popisat aj tu strukturu, ak bude cas} NEBUDE
{\it Crochemore et al.} described a data structure that represents $DIST$ and $OUT$
matrices for  \nocite{Crochemore2002} individual blocks $B_{i,j}$. Structures
for $B_{i,j}$ can be
computed from $DISTS$ and $OUT$ matrices of theirs predecessors in time
$O(|X_i|+|Y_j|)$. Details of this algorithm can be found in
\cite{Crochemore2002}. Time complexity of this algorithm is proportional to the
sum of the sizes of all blocks, which is $O(n^2/\log n)$ where $n$ is the length
of the longer sequence.


\begin{comment}
\section{Algorithmic Improvements For HMMs}
\label{SECTION:ALGORITHMICIMPROVEMENTS}


In this section we review implementation details and several algorithmic
improvements of the Viterbi algorithm and the Posterior decoding for HMMs and
pHMMs. Mostly we will assume that HMMs does not have silent states.  Most of these
techniques are easily adjustable to HMMs with silent states. 

\subsection{Basic Implementations}

Implementation of the Viterbi algorithm and the Forward-Backward algorithm can be
done by a two-dimensional dynamic programming, similarly as with the sequence
alignment. Let $H$ be an HMM, $V[i,v]$ be  $n\times m$ matrix where $n$ is the
length of the sequence and $m$ is the number of states of the HMM. $V$ will be the
dynamic programming matrix for the Viterbi algorithm or the Forward algorithm.

Code for the Viterbi algorithm is following:
\begin{lstlisting}
Initialize D[0,*]
for i = 1...n-1:
  for v = 0...m-1:
    V[i,v] = max    [u=0...m-1] V[i-1,u]*[v,X[i]]*a[u,v]
    B[i,v] = argmax [u=0...m-1] V[i-1,u]*[v,X[i]]*a[u,v]
statePath[n-1] = argmax [u=0...m-1] V[n-1,u]
for i=n-1...1:
    statePath[i-1] = B[i,statePath[i]]
\end{lstlisting}

This algorithm runs in $O(nm^2)$ and requires $O(nm)$ memory. Note that values
of row $V[i,\cdot]$ and $B[i,\cdot]$ are computed from rows $V[i-1,\cdot]$ so if
we want just the probability of the most probable state path, we need to
remember only last two rows of $V$ (and we do not need $B$) so the memory
requirements will be $O(n+m)$. If we replace on line $4$ maximum with summation
and remove lines $5-8$, we will obtain the Forward algorithm. Lines $6-8$
implements back-tracing procedure, which will be omitted in the Forward
algorithm. The Backward algorithm is analogous to the Forward algorithm.

Note that if $a[u,v]$ is zero, then value on the right side of lines $4$ nd $5$
is zero. Therefore we have to iterate only for such $u$, that $u\to v$ is
transition in $H$. By this we will reduce time complexity to $O(n(m+t))$ where
$t$ is the number of transitions of $H$ not only for the Viterbi algorithm, but
also for the Forward algorithm and the Posterior decoding.

\end{comment}

\begin{comment}
\section{Non-Affine Gap Models} 


%\todo{Mozno by sa dalo dodat aj viac citacii} 
The reasons why affine gap models are
used in sequence alignment are that they are simple, easy to compute and give
reasonable results. While affine gap scoring works fine for short gaps, penalty
for long gaps is too high. Use of different gap models can improve quality of
reconstructed alignments \cite{Gill2004,Cartwright2009}.

Let $f(x)$ be the gap penalty for a gaps of length $X$.
In 
affine gap models, this function has the form $f(x)=g+dx$. For general gap function, we have to
reformulate equation \ref{ALIGN:ALGO:AFFINE} for computing the highest scoring
alignment in following way:
\begin{align}
M[i,j] &= \max
\begin{cases}
 M[i-1,j-1]+S(X_i,X_j)\\
 \max_{x\leq j}M[i,j-x]+f(x)\\
 \max_{x\leq i}M[i-x,j]+f(x)
\end{cases}, 0<x\leq i<n\label{ALIGN:ARBITRARYGAPEQUATION}
\end{align}
This algorithm has time complexity of $O(n^3)$, where $n$ is the length of the
longer sequence. Note that this was the original Needleman-Wunsch 
algorithm \cite{Needleman1970}, and later it was improved to $O(n^2)$ algorithm
\cite{Sankoff1972} for alignment with linear gap penalties.
Unlike previous recursive equations, in this algorithm $M[i,j]$ depends not
only on neighbouring cells, but also on all previous cells in the same row and column.
Therefore some memory-saving techniques like Hirschberg algorithm or checkpointing cannot be
used with general gap penalties.


\subsection{Convex/Concave Gap Functions}\label{SECTION:CONVEX}

Arbitrary gap penalties increased running time of the Needleman-Wunsch algorithm
by factor of $O(n)$. However if we place some
restrictions on the  gap penalty function, we can use faster algorithms. In this
section, we
show an algorithm that computes the optimal alignment in $O(nm\log n)$ time if
the gap
penalty is convex or concave. In some cases this algorithm can be improved to
$O(nm)$ time. This algorithm was discovered independently by Miller
{\it et al. (1988)} and Galil {\it et al. (1989)} \nocite{Miller1988,Galil1989}.

We will present a variant of $O(nm\log n)$ algorithm for convex gap functions which
we believe is easiest to explain . Algorithm for concave gap function is similar.
We consider only convex functions defined on natural numbers (gap length can be only
natural number). Our definition of convex function is similar to one used in
\cite{GusfieldBook}.


\begin{definition}
We assume that $f(x)$ is a  function defined on natural numbers. Function $f$
is convex if and only if 
\[f(x+1)-f(x)\geq f(x)-f(x-1)\]
for all $x\in\mathbb{N}$.
\end{definition}
\begin{note}
It is easy to show that if $f$ is convex then $f(x+d)-f(x) \leq f(x+e+d) -
f(x+e)$ for nonnegative $d$ and $e$.
\end{note}

We will improve dynamic programming that computes matrix $M$ using equation
\ref{ALIGN:ARBITRARYGAPEQUATION}. The increase in running time is caused by two
terms: $M[i,j-y]+f(y)$ and
$M[i-x,j]+f(x)$. In dynamic programming from section \ref{SECTION:NEEDLE} we  have
to add additional nested cycles that go through all values $x$ and $y$. We show
a data
structure that can compute $\max_{0\leq y < j}M[i,j-y]+f(y)$ in $O(\log n)$
amortized time.
This data structure can also handle the second problematic term, but we
will describe only the first one. First we change the variable in the
maximization so that we maximize

%$\max_{0\leq y < j}M[i,j-y]+f(y)$ is equivalent to
\begin{equation}
\max_{0\leq k < j}M[i,k]+f(j-k)\label{CONVEX:MAXFUNCTION}
\end{equation}
which is equivalent to the original expression from the equation
\ref{ALIGN:ARBITRARYGAPEQUATION}.  Our data structure is a list $L_j$ containing
values of 
$k$ that can become maximum for future values of $j$. We will try to keep this list as small as
possible. List $L_j$ will be a called \firstUseOf{candidate list} and members of $L_j$
will be called \firstUseOf{candidates}. Rank of candidate $k$ is the value
$G(k,j)=M[i,k]+f(j-k)$. To compute $M[i,j]$ we iterate over values of $k$ in
$L_j$ and find the one with the highest rank.
%$L_j$ is candidate list for computing $M[i,j]$. 
List $L_{j+1}$ will be computed from $L_j$ by adding $j$ and
removing some elements of $L_j$. 
%For simplicity, let $G(k,j) = M[i,k]+f(j-k)$.
%If $k\in L_j$, than \firstUseOf{rank} of $k$ is $G(k,j)$. 
Note that the rank of
candidates will change when we move from $L_j$ to $L_{j+1}$.

We will explain this algorithm in iterative way: we will be adding to algorithm
new features until it will have desired time complexity. 

%Let $L_j$ be the list of such $k$'s. We will call $L_j$ a \firstUseOf{candidate
%list} and members of $L_j$ \firstUseOf{candidates}. $L_j$ will be created from
%$L_{j-1}\cup\{j-1\}$ by removing some elements. Let $G(k,j) =
%M[i,k]+f(j-k)$.  \firstUseOf{Rank} of candidate $k$ is $G(k,j)$.  i
The trivial algorithm includes all values smaller than $j$ in $L_j$.
%We start with $L_j$ containing all $k$ that are smaller then $j$.  We can find
%element with maximal rank by computing rank for all candidates from candidate
%list $L_j$. 
This leads to $O(j)$ time for finding maximum and $O(1)$ for update
from $L_i$ to $L_{i+1}$
%candidate list (we have to add element $j$ to list $L_{j}$ to create
%list $L_{j+1}$). 
This is equivalent to the original Needleman-Wunsch algorithm
\cite{Needleman1970}. 
We will use the following lemma to
decrease size of candidate lists. This lemma and proof is  slightly modified
from 
\cite{GusfieldBook}.


\begin{lemma}\label{OneStrikeAndOut}
If $G(k,j)\leq G(k',j)$ for some $k<k'<j$, then
$G(k,j')\leq G(k',j')$ for all $j'\geq j$. 
\end{lemma}


\begin{proof}

Inequality $M[i,k]+f(j-k)\leq M[i,k']+f(j-k')$ implies that
$M[i,k]-M[i,k']\leq f(j-k')-f(j-k)$. From convexity of $f$ we have
$f(j-k')-f(j-k)\leq f(j'-k')-f(j'-k)$, and therefore
$M[i,k]-M[i,k']\geq f(j'-k')-f(j'-k)$ which 
is equivalent to
$M[i,k]+f(j'-k)\leq M[i,k']+f[(j'-k')$ or $G(k,j')\leq G(k',j')$.

\end{proof}

Using this lemma, we can improve our algorithm.
Once we find out that $G(k,j)\leq
G(k',j)$, for $k<k'$, we can remove $k$ from $L_{j'}, j'\geq j$ without
affecting the result
of the algorithm, because rank of $k$ will remain smaller than rank of $k'$ for
all $j'\geq j$.
When we move from $j$ to $j+1$, we therefore set $L_{j+1}=L_{j}\cup\{j\}$ and
then
we remove from $L_{j+1}$ that candidates that have lower or equal rank than the 
candidate following them in the list.
%when we move from
%$j-1$ to $j$ we add set $L_j=L_{j-1}\cup\{j-1\}$ and we compute $G(k,j)$ for
%all $k\in L_j$ and remove all such $k$ where there exists $k'\in L_j$ such that
%$G(k',j)\geq G(k,j)$. i
This can be done in $O(|L_{j}|)$ time. Candidates in the resulting list,
$L_{j+1}$ will have decreasing rank. Therefore the first item in
$L_{j+1}$ is the maximal one, so we can find it $O(1)$ time.  This change
does not improve the worst-case behaviour of our algorithm, but it might
significantly reduce the  size of $L_j$ in practice. 


Using this algorithm, $L_j$ will be always ordered by rank. However when we move
from $j$ to $j+1$, the ranks will change, and we have to remove some elements 
to restore this property. The algorithm can be improved if we can compute the column in
which two adjacent  candidates will brake the ordering.

%Nice property of $L_j$ is that it's members are ordered by rank. However, 
%when we move to column $j+1$, ranks of candidates will change and order might be
%broken and therefore we have to traverse through whole $L$ to fix that.
\begin{definition}
For $k<k'$, let $H(k,k')=l$ be the minimal $l$ such that $G(k,l)\leq G(k',l)$. If such $l$ does
not exists,  $H(k,k')=\infty$ 
\end{definition}

If the rank of $k$ is greater than the rank of $k'$ then $H(k,k')$ is the first
column $j'$
 where the rank of $k$ is less than or equal to the
rank of $k'$. Candidate $k$ will be removed in column $j'$, or possibly earlier.  Note that
such $l$ can be found using binary search (based on lemma \ref{OneStrikeAndOut})
in $O(\log n)$ time for any convex function. For some convex functions we can
find it analytically in constant time.


We can further decrease the size of the candidate list using the following
lemma.

\begin{lemma}\label{TimeLemma}
Let $k,k',k''\in L_j$ and $k<k'<k''$. If $H(k,k')\geq H(k',k'')$, element $k'$ can be
removed from $L_j$.
\end{lemma}

\begin{proof}
Since candidates in $L_j$ are ordered by rank, $G(k,j)>G(k',j)>G(k'',j)$.
From definition of $H$ we know that 
$G(k',j')\leq G(k'',k')$ for  all $j'\geq H(k',k'')$. Since
$H(k,k')\geq H(k',k'')$, then $G(k,j')\geq G(k',j')$ for all $j'<H(k,k'')$.
Therefore the rank
of $k'$ will never be maximum and $k'$ can be removed.

%and both are in $L_j$ then $k$ has higher rank that $k'$. Therefore
%it is not maximum. $k'$ have in $H(k',k'')$ lower rank and will be removed from
%candidate list (using lemma \ref{OneStrikeAndOut}). However, since $H(k,k')\geq
%H(k',k'')$ rank of $k'$ will never be maximum.
\end{proof}

Now we can formulate the final algorithm. After updating $L_j$ from $L_{j-1}$,
$L_j$ will satisfy the following invariant.
\begin{invariant}\label{LogGapInvariant}
All consecutive elements $k,k'$ in $L_j$ satisfy the following properties:
\begin{enumerate}
\item $k<k'$
\item $G(k,j)>G(k',j)$ (decreasing rank)
\item if $k'$ is not the last element of $L_j$ and $k''$ is the element
following $k'$
in $L$, then $H(k,k')> H(k',k'')$ (increasing ``time'' when consecutive
candidates will brake the ordering of ranks)
\end{enumerate}
\end{invariant}

Since the rank is decreasing, we can find the maximum in $O(1)$ time. Now we show how
to compute $L_{j+1}$ from $L_j$. Let $L$ be our working list,
$L[k]$ be its $k$-th element, and $l$ be the index of the last element of $L$. 
List $L_{j+1}$ can be computed by the following algorithm. 
\begin{enumerate}
\item $L=L_j$
\item $L_{j+1}=L\cup\{j\}$.
\item If $G(L[0],j+1)\leq G(L[1],j+1)$ then remove $L[0]$ from
list (lemma \ref{OneStrikeAndOut}).
\item While $H(L[{l-1}],L[{l}])\leq H(L[l-1],L[l-2])$ or $G(L[l],j+1)\geq
G([l-1],j+1)$, remove $L[l-1]$ from the list (lemma
\ref{TimeLemma} and lemma \ref{OneStrikeAndOut}).
\end{enumerate}

The first element is also the first element that can break order, so we have to
test it (step 3). Since we add candidate $j$, we have to remove from the end of
the list all candidates with smaller rank than $j$ and those that can be removed
based on lemma \ref{TimeLemma} and lemma \ref{OneStrikeAndOut}.  Since both
ranks and times are ordered, we need to remove candidates only from the end of
the list.

It is easy to see that if $L_j$ satisfies the invariant, $L_{j+1}$  satisfy it.
It is also clear that this algorithm will never remove any candidate that will
be the maximum later, which implies the correctness of the algorithm.

The algorithm described above has running time $O(l\log(n))$, but the amortized time
complexity for computing all columns of one row of matrix $M$ is $O(n\log n)$ since every candidate will
be added/removed to/from list at most once. If we can compute function $H$ in
constant time, the time complexity is $O(n)$.

Using this algorithm, we can compute the optimal alignment with convex gap functions in
$O(nm\log n)$ time and in $O(nm)$ time  if $H$ can be computed
in constant time.

There is also a more complicated $O(nm\alpha(n))$ algorithm for this problem
\cite{Klawe1990}, where $\alpha(n)$ is the inverse Ackermann function. This algorithm
is very complex, and we doubt that it is useful in practice. However, we are
not aware of any experimental comparison of these two algorithms.




%tu chcem definiciu convexnej gap funkcie
%algoritmus ktory je v case n^2 log(n)
%algoritmus, ktory bezi v case n^2 alpha(n)
%kubicky algoritmus v worstcase, ale prakticky konstantny (budem musiet najst 
%citaciu -- nepodarilo sa mi to najst. Zaujimave\ldots)

\end{comment}

\begin{comment}
\subsection{Using Sequence Repetition}

This techniques uses dictionary based encoding schemes to speedup calculation of
algorithms. We will show how to use this technique to Forward algorithm. Details
about how to use this technique to other algorithms can be found in
\cite{Lifshits2009}.


Idea of this speedup is that we reformulate Forward algorithm into sequence of
matrix multiplications.
Let $H=(\Sigma,V,I,e,a)$ be a HMM, $|V|=m$.  Let $M_x[u,v]=a_{u,v}e_{v,x},
u,v\in V,x\in \Sigma$ be the $m\times m$ matrix and $I_x[u]=I_ue_{u,x}$ be
$1\times m$ vector. Matrix $M_x[u,v]$ corresponds to the transition from the
state $u$ to the state $v$ with emission of $x$ from the state $v$.

\begin{lemma}\label{LEMA:MATRIXMULTI}
Let $X=X_1\dots X_n$ be a sequence, $H$ be a HMM and $M_x$ and $I_x$ defined as above and
$F[i,\cdot]$ be row vector from the Forward algorithm.
Then
\begin{align}
F[0,\cdot] &= I_{X_0}\\
F[i,\cdot] &= I_{X_0}\prod_{j=1}^i M_{X_j}
\end{align}
for all $i< n$.
\end{lemma}

\begin{proof}
We prove this lemma by induction.  If $|X|=1$ then
$F[0,v]=I_{X_0}[v]=I_{v}e_{v,X_0}$ for all $v\in V$.  Assume that for
$F[k,:] = I_{X_0}\prod_{j=1}^k M_{X_j}$ (note that the  product of zero matrices
is the identity matrix). We have 
\begin{align*}
\left(I_{X_0}\prod_{j=1}^{k+1} M_{X_j}\right)[v] &= 
\left(F[k,:] M_{X_{k+1}}\right)[v]\\ &= \sum_{u\in V} F[k,u]
M_{X_{k+1}}[u,v]\\ &= \sum_{u\in V} F[k,u] a_{u,v}e_{v,X_{k+1} } \\&= F[k+1,v] 
\end{align*}
for all $v$. 
\end{proof}

Consequence is that we can write the Forward algorithm as the sequence of
multiplications of $|\Sigma|$ different matrices. Since matrix multiplication
is associative, we can use repetitive patters to speedup the calculation. 
Let $x$ be a subsequence of $X$ that has $k>1$ non-overlapping occurrences in $X$.
We can compute $M_{x[0]}M_{x[1]}\dots M_{x[{|x|-1}]}$ once and use it several
times. Let $M_{x}=\prod_{i=0}^{|x|-1}M_{x[i]}$ for any string $x$.

By proper choosing of substrings $x$ we can speedup computation of the Viterbi
algorithm\footnote{For the Viterbi algorithm we need different matrix
multiplication}, the Forward algorithm  or the
Forward-Backward algorithm. Algorithms have following structure:
\begin{enumerate}
\item We choose the \firstUseOf{dictionary} $D$. $D$ is set of strings over
alphabet $\Sigma$. String is a
\firstUseOf{good} if it belongs to $D$.
\item We compute $M_x$ for all $x\in D$.
\item We split input sequence $X[1:]$ into the sequences of a good strings
$x_0,x_1,\dots,x_{k-1}$ such that $x_0x_1\dots x_{k-1}=X[1:]$.
%and $x_i\in D$ for
%all $0\leq i<k$
\item We compute $I^{X_0}\prod_{i=0}^{k-1}M_{x_i}$ in $O(kf(m))$ time where
$f(m)$ is time needed to compute multiplication of two matrices of size $m\times m$
\end{enumerate}

Lifshits {\it et al.} \cite{Lifshits2009} showed several ways how to choose set
$D$ and split $X$ into sequence of good words. One of them is to use LZ78 factorization
to split $X$ into $O(n/\log n)$ good words. Computation of $M_x$ for all $x\in
D$ can be done in $O(|D|m^3)$. For all $x\in D$, the matrix $M_x$ can be
computed from $M_{x[:|x|-1]}$ in $O(m^3)$
time with the following algorithm:
if $|x|=1$ then we compute $M_x$ trivially.
If $|x|>1$ then there is $x'\in D$ such that $x=x'a,a\in \Sigma$ and therefore $M_x =
M_{x'}M_{a}$. 

Since $|D|=O(n/log n)$, computing $M_x$ takes $\Omega(nm^3/log m)$
time if
we use $O(m^3)$ algorithm for computing the matrix multiplication. Computing the fourth
step of the algorithm takes also $\Omega(nm^3/\log n)$ time and therefore the overall time complexity
is $\Omega(nm^3/\log n)$ which is $\Omega(\log n/m)$ speedup over the
classical implementation of the Forward algorithm for the dense matrices.

It is possible to use different encoding methods. Speedup for the run length encoding 
is $\Omega(r/\log r)$, for the straight-line programs is $\Omega(r/m)$ and for the
byte pair encoding is $\Omega(r)$ where $r$ is the compression ratio under each
compression scheme. We will not discuss details of this implementations, they
can be found in \cite{Lifshits2009}.

To adapt this approach to the Viterbi algorithm we have to make two adjustments. 
We have to use the max-time matrix multiplication instead of matrix multiplication
\cite{Lifshits2009}. The max-time matrix multiplication is the  matrix multiplication
where summation is replaced with the maximization:
\[M_1\odot M_2 [u,v] = \max_{0\leq i\leq m-1}M_1[u,i]M_2[i,v] \]
where $M_1$ and $M_2$ are matrices of size $m\times m$. This matrix
multiplication is also associative \cite{Lifshits2009}.

The second adaptation is that we have to had the ability to reconstruct 
the most probable state path.
After computing $I^{X_0}\prod_{i=0}^{k-1}M_{x_i}$ we do the back-trace to obtain 
the partial state path $\pi'_0,\pi'_1,\dots \pi'_{k}$ which is the subsequence of the
most probable state path. Each $\pi'_i$ is a state corresponding to the last
symbol of $x_i$. We have to reconstruct state path between $\pi'_i$'s; for
every $x_i$ we have to reconstruct state path between $\pi'_{i-1}$ and $\pi'_i$.

All mentioned compression schemes has the property,
that for every good string $x$ there are also two good strings $x^1,x^2$ such
that $x=x^1x^2$. To recover state path, for every good string $x$ and $u,v\in V$
we have to pre-compute \[R_x[u,v] = \arg\max_{i\in V}M_{x^1}[u,i]M_{x^2}[i,v]\]

Let $x=x_i$. Then $R_{x}[\pi'_{i-1},\pi'_{i}]$ corresponds to the state in the
position of the last symbol of $x^1$. By recursive applying this rule to $x^1$
and $x^2$ we can reconstruct the most probable state path on $x$. By doing this
we can reconstruct the most probable state path with the time complexity $O(n)$.

Lifshits {\it et al.} also applied this  speedup  to the Posterior decoding and
the Baum-Welsch training. 

%Let $X$ be the most probable state path that uses 
%To adapt
%this approach to the Viterbi algorithm we have to use in matrix multiplication
%maximum instead of summation (such multiplication is also associative)
%\cite{Lifshits2009}.

\subsection{On-line Viterbi Algorithm}
\label{SECTION:ONLINEVITERBI}

The On-line Viterbi algorithm is different approach to reduce the memory complexity of
the Viterbi algorithm. In this approach we represent matrix $B$ in sparse way
and we keep only those parts of $B$ that can be necessary for the reconstruction of
the most probable state path \cite{Sramek2007}.
We  represent $B$ as the forest $G=(V',E)$ where
\begin{align*}
V' &= \left\{ (i,v)\mid 0\leq i< n, v\in V  \right\}\\
E &= \left\{ (i,v)\to (i-1,B[i,v])\mid 0<i<n,v\in V\right\}
\end{align*}
Edges in this forest are oriented from the children to parents. Roots of the
trees are vertices $(0,v),v\in V$.  The most probable state path that generated
$X[:i]$ and ending in vertex $v$ corresponds to the path from vertex $(i-1,v)$
to root of its tree. Let $S_{i,v}$ be the set of all vertices on the path from
the vertex
$(i,v)$ to the root of its tree. Let $S_i=\cup_{v\in V}S_{i,v}$.
After computation of $B[i,v]$, only the vertices in $S_i$ can be in the most
probable state path so others can be removed. State paths corresponding to
$S_{i,v}$ may overlap, so by removing not necessary vertices we can reduce the memory
requirements.

{\v S}r{\'a}mek {\it et al.} developed data structure that maintains $S_i$
without introducing significant overhead.  Let $G_i$ be the induced subgraph of $G$
consisting from the vertex set $S_i$. Leaf $(j,v)$ is \firstUseOf{unreachable} if
$j<i$.  $G_i$ does not contain unreachable leaf and therefore $G_i$ contains $m$
leaves $(i,v),v\in V$. Construction of $G_{i+1}$ is following.
After computing $B[i+1,\cdot]$ we add vertices $(i+1,v),v\in V$ and edges 
$(i+1,v)\to (i,B[i+1,v]),v\in V$. Vertices $(i+1,v),v\in V$ are the new leaves.
While in $G_{i+1}$ are some internal leaves, we remove them.
After removing all internal leaves, $G_{i+1}$ contains only vertices from
$S_{i+1}$.
To make $G_i$ smaller, we use the compressed representation of the trees. We compress
chains of vertices with one children into one edge (but we keep
corresponding state path of compressed chain). If we do so, every vertex will
have at most two children and therefore there will be at most $m-1$ internal
nodes. $G_i$ has therefore always $O(m)$ leaves and updating can be done in
$O(m)$ time. Therefore this technique does not alter the asymptotic running time
of the Viterbi algorithm. In practice this technique increase running time by
$5$\% \cite{Sramek2007}.

Improvement in the memory requirements can be significant. However, there exists
an HMM on which this technique will not improve memory requirements
\cite{Sramek2007}. In practice, {\v S}r{\'a}mek {\it et al.} reports polylogarithmic 
memory requirements for HMM used for gene finding.
They also proved that for two
state HMMs (with one exception), expected memory complexity of this algorithm is
$O(H\log n)$ where $H$ is a constant specific for an HMM. Estimation of the expected time
complexity for given HMM is still open problem.

Keibler {\it et al. (2007)} developed similar algorithm called the Treeterbi
algorithm. Both algorithms are similar, but the Treeterbi does not use the
compressed version of the graph. Both algorithms can be extended to the GHMM and
pHMM, the extensions of the Treeterbi algorithm can found in \cite{Keibler2007}.
\end{comment}
