\chapter{Conclusion}

In this thesis we studied the use of hidden Markov models for sequence
annotation and sequence alignment and provided both theoretical and practical
contributions.

Chapter \ref{CHAPTER:TWOSTAGE} considers two-stage decodings for sequence
annotation. In a two stage decoding we at first compute the guide and then find
an annotation optimizing gain function using the guide as an restriction. Guides
can be used to decrease running time, but we have shown experimentally that
guides can also improve the accuracy. Then we focus on the computational
problems of computing guides.  We gave the combinatorial proof that the most
probable set problem is NP-hard.  We also show that this problem is
fixed-parameter tractable when the fixed parameter is the number of states of
the HMM It is not clear, if the most probable set problem is in NP. Then we
show a reduction from 3-SAT to the most probable restriction problem proving
that this problem is NP-complete. Similarly to the most probable set problem,
this problem is also fixed-parameter tractable.  Finally, we focus on the most
probable footprint problem.  We showed a constant-sized HMM for which finding
the most probable footprint is NP-hard even if we use the identity function as
a labeling function. We also show how to alter the proof of the most probable
footprint to the proofs of the most probable set and the most probable
restriction. 

There are still some open problems left. The most probable footprint is a
special case of the most probable ball problem (with the border shift sum
distance, where $r$ is the radius) which was studied by Brown and Truszkowski
\cite{Brown2010}; we described it in Section \ref{SECTION:DISTANTMEASURES}.
This problem is NP-hard even for $r=0$ \cite{Brown2010}, if multiple states can
have same label. Additionally, if $r\geq n$, where $n$ is the length of the
input sequence, the most probable ball problem is equivalent to the most
probable footprint problem. Our result imply that the most probable ball
problem is NP-hard even if all states have unique label. It is still open
problem, if the most probable ball problem is NP-hard for $r<n$ and HMMs with
the identity function as an annotation function. 

From the practical point of view, two-stage algorithms  could be used
for pair hidden Markov models, or in different application domains; for example
the gene finding.

In Chapter \ref{CHAPTER:REP} we study the sequence alignment problem using pair
HMMs. We propose the new SFF model that incorporates tandem repeats, which are
prevalent in the genomic sequences. In addition to the new model, we explored
additional decoding criteria, the block Viterbi algorithm and the block
posterior decoding, which treat tandem repeats a blocks. On the simulated data,
the SFF model alone decreased the error rate by at least $15\%$  compared to
the standard three-state model with the Viterbi algorithm. The decrease in the
error rate was  concentrated near repetitive intervals. 

There are several possible directions for further research. For example our
model does not take into account dependencies between repetitions and therefore
tandem repeats in different sequences are independent (apart from using same
motif). We could model the evolution of repeats more realistically either in
the SFF model or in the decoding algorithm.  Similar improvement can be done
into the TTP model, which does not model the first repetition well and allows
use of the different motifs in homologous repeats. This problem could be solved
by modeling the first repetition as an pair profile HMM and other repetitions
would be modeled by the TANTAN model.  Other improvement can be incorporating
additional submodels modeling other features, for example gene structures. It
might be interesting to study the ways of incorporating different, but
overlapping features into the same model (for example the tandem repeats inside
of gene structures).  Finally, we can extend this method to align multiple
sequences. From the practical point of view, our software needs to be
optimized, so it can be used on genomic sequences of arbitrary length.

We studied the decoding algorithms in all chapters of this thesis.
Traditionally, the selection of proper decoding method is underestimated
problem and many applications simply choose the Viterbi algorithm. While it is
true that for many applications the Viterbi algorithm is optimal,  our
experience shows that choosing the proper decoding method can significantly
improve the accuracy of a predicted annotations or an alignments. We believe
that the selection of the decoding function should be a important part of
designing method that uses probabilistic model, because domain-specific
decoding algorithms can decrease some statistical biases and can be used as an
compensation for simplifications done during designing of the HMM. Finally,
there are many interesting computational problems that arises from the decoding
algorithms, from the most probable annotation problem to the optimization of
custom gain functions.

\todo{odstavec nadhladu}
\label{LastPage}
