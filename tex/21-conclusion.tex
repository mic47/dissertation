\chapter{Conclusion}

In this thesis we studied the use of the (pair) hidden Markov model for
sequence annotation (Chapter \ref{CHAPTER:TWOSTAGE}) and sequence alignment
(Chapter \ref{CHAPTER:REP}) and provide both theoretical and practical
contributions.

The Chapter \ref{CHAPTER:TWOSTAGE} considers two-stage decodings. In two stage
decoding we at first compute the guide and then find an annotation optimizing
gain function using guide as an restriction. Guides can be used to decrease
running time, but we have shown experiment where guides can be parts of the
decoding criteria and increase accuracy\todo{Ale iba ak naozaj spravime tie
experimenty}. Then we focus on the computational problems of computing guides.
We gave the combinatorial proof that the most probable set problem is NP-hard.
We also show that this problem is fixed-parameter tractable when the fixed
parameter is the number of states of HMM It is not clear, if the most probable
set problem is in NP, since deciding is the probability of the set is non-zero
is NP-complete. Then we show reduction from 3-SAT to the most probable
restriction problem proving that this problem is NP-complete. Similarly to the
most probable set problem, this problem is also fixed-parameter tractable.
Finally, we focus on the most probable footprint problem. We proved that this
problem is NP-complete even if a labeling function is the identity function
(without this constraint, the most probable annotation problem can be trivially
reduced to this problem). We showed the constant-sized HMM for which finding
the most probable footprint is NP-hard even if we use the identity function as
a labeling function. We also show how to alter the proof of the most probable
footprint to the proofs of the most probable set and the most probable
restriction. 

There are still some open problems left. The most probable footprint is a
special case of the most probable ball problem (with the border shift sum
distance, where $d$ is the threshold) which was studied by Brown and
Truszkowski \cite{Brown2010}, and we described it in Section
\ref{SECTION:DISTANTMEASURES}.  This problem is NP-hard even for $d=0$
\cite{Brown2010}, if multiple states can have same label. Additionally, if
$d\leq n$, where $n$ is the length of the input sequence, the most probable
ball problem is equivalent to the most probable footprint problem. Our result
imply that the most probable ball problem is NP-hard even if all states have
unique label. It is still open problem, if the most probable ball problem is
NP-hard for $d<n$ and HMMs with the identity function as an annotation
function. 

From the practical point of view, two-stage algorithms algorithms could by used
for pair hidden Markov models, or on different application domains; for example
the gene-finding.

In Chapter \ref{CHAPTER:TWOSTAGE} we study the sequence alignment problem. We
propose the new SFF model that incorporates tandem repeats, which are prevalent
in the genomic sequences. In addition to the new model, we explored additional
decoding criteria, the block Viterbi algorithm and the block posterior
decoding, which treats tandem repeats a blocks. On the simulated data, the SFF
model alone decreased the error rate by at least $15\%$ when comparing to
standard three-state model with the Viterbi algorithm. The decrease in the
error rate was was concentrated near repetitive intervals. 

There are several possible directions for further research. For example we can
use sunflower model equivalent to the sunflower model with the silent cycle,
leading into more symmetrical model. Additionally, our model does not take into
account dependencies between repetitions and therefore tandem repeats in
different sequences are independent (apart from using same motif). Same
improvement could be done into the TTP model, which is based on TANTAN model.
Other improvement can be incorporating additional submodels modeling other
features, for example gene structures. Finally, we can extend this method to
align multiple sequences. From the practical point of view, our software needs
to be optimized, so it can be used on genomic sequences of arbitrary length.

\begin{comment}
\begin{reformulate*}
Kapitola 3 -- ukazali sme ze two-stage vie pomoct v presnosti, 
daju sa skumat dalsie uplatnenia na ine domeny, a vhodne guide funkcie.

Ako by sa vysledky dali rozsirit na parove modely, vie mi to ze mam 2 pasky v niecom pomoct? kosntantne modely  apodobne? Mozu tam byt nejake aproximacne algoritmy s dobrym pomerom?, pripadne vedia guided
algorititmy pomoct aj pri parovych HMM? 

Kapitola 4: Ukazali sme taky a taky model a algoritmus, daval take a take vysledky  Dal by sa pouzit model s cyklom
stavov, bol by viac symetricky. Dali by sa pridat dalsie submodely (hladanie
genov). Da sa viac rozvinut TANTAN model, pripadne modelovat evoluciu
repeticii. Ma zmysel kombinovat (z praktickeho hladicka) dalsie sumbodely s
repeatovym, kedze su tie modely nezavisle.

\end{reformulate*}
\end{comment}
\label{LastPage}
