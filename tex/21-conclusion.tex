\chapter{Conclusion}

In this thesis we studied the use of the (pair) hidden Markov model for
sequence annotation (Chapter \ref{CHAPTER:TWOSTAGE}) and sequence alignment
(Chapter \ref{CHAPTER:REP}) and provide both theoretical and practical
contributions.

The Chapter \ref{CHAPTER:TWOSTAGE} considers two-stage decodings. In two stage
decoding we at first compute the guide and then find an annotation optimizing
gain function $f$ using guide as an restriction. Guides can be used to decrease
running time, but we have shown experiment where guides can be parts of the
decoding criteria and increase accuracy\todo{Ale iba ak naozaj spravime tie
experimenty}. Then we focus on the computational problems of computing guides.
We gave the combinatorial proof that the most probable set problem is NP-hard.
We also show that this problem is fixed-parameter tractable when the fixed
parameter is the number of states of HMM It is not clear, if the most probable
set problem is in NP, since deciding is the probability of the set is non-zero
is NP-complete. Then we show reduction from 3-SAT to the most probable
restriction problem proving that this problem is NP-complete. Similarly to the
most probable set problem, this problem is also fixed-parameter tractable.
Finally, we focus on the most probable footprint problem. We proved that this
problem is NP-complete even if a labeling function is the identity function
(without this constraint, the most probable annotation problem can be trivially
reduced to this problem). We showed the constant-sized HMM for which finding
the most probable footprint is NP-hard even if we use the identity function as
a labeling function. We also show how to alter the proof of the most probable
footprint to the proofs of the most probable set and the most probable
restriction. There are remaining questions regarding this problems\todo{Ake su
otvorene otazky}

In Chapter \ref{CHAPTER:TWOSTAGE} we study the sequence alignment problem.

\ref{CHAPTER:REP}
\begin{reformulate*}
Kapitola 3 -- ukazali sme ze two-stage vie pomoct v presnosti, 
daju sa skumat dalsie uplatnenia na ine domeny, a vhodne guide funkcie.

Ako by sa vysledky dali rozsirit na parove modely, vie mi to ze mam 2 pasky v niecom pomoct? kosntantne modely  apodobne? Mozu tam byt nejake aproximacne algoritmy s dobrym pomerom?, pripadne vedia guided
algorititmy pomoct aj pri parovych HMM? 

Kapitola 4: Ukazali sme taky a taky model a algoritmus, daval take a take vysledky  Dal by sa pouzit model s cyklom
stavov, bol by viac symetricky. Dali by sa pridat dalsie submodely (hladanie
genov). Da sa viac rozvinut TANTAN model, pripadne modelovat evoluciu
repeticii. Ma zmysel kombinovat (z praktickeho hladicka) dalsie sumbodely s
repeatovym, kedze su tie modely nezavisle.

\end{reformulate*}
\label{LastPage}
