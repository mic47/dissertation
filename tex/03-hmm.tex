\chapter{Hidden Markov Models}
\label{CHAPTER:HMM}

\abbreviation{Hidden Markov Models}{HMM} are graphical probabilistic models
commonly used to sequence annotation or sequence alignment. An HMM is
a probabilistic finite state machine that in every state emits one symbol. Later
we will also discuss variants of HMMs that emit more symbols or that emit symbols on multiple
tapes. In this chapter, we will describe basic
definitions and algorithms that are used with HMMs.

\section{Definitions}\label{SECTION:HMMDEF}
                       
%HMM
%Pravdepodobnost
%Posterior pravdepodobnost
%Anotacia
%Pravdepodobnost anotacie
%Footprint
HMMs are generative probabilistic models.
The generative process of an HMM starts in a random state $q$ sampled according
to the \firstUseOf{initial
distribution} $I$.  When an HMM is in some state $q$ it emits one symbol from
alphabet $\Sigma$ according distribution $e_q$ and moves to another state
according to transition distribution $a_q$. Note that emission and transition
distributions can be different for every state.  This process produces two
sequences: sequence of states $\pi=\pi_0\pi_1\pi_2\dots$ called
\firstUseOf{state path} and output sequence $X=X_0X_1X_2\dots$ over alphabet
$\Sigma$. In this work we will use only discrete versions of HMMs with finite state
space and alphabet.  


\begin{definition}
Any square matrix $M$ of size $n\times n$ is \firstUseOf{stochastic} if satisfy
following properties.
\begin{enumerate}
\item $\forall 0\leq i<n,0\leq j < m, 0\leq M[i,j]\leq 1$
\item $\forall 0\le i<n, \sum_{i=0}^{m-1}M[i,j]=1$
\end{enumerate}
\end{definition}

\begin{note}
Stochastic matrix consists from $n$ distributions over set of size $n$.
\end{note}

\begin{definition}\label{DEF:HMM}
A \abbreviation{Hidden Markov Model}{HMM} is a tuple $H=(\Sigma,V,I,e,a)$ where
$\Sigma=\{\sigma_0,\dots\sigma_{m-1}\}$ is a finite alphabet of size $m$,
$V=\{v_0,\dots,v_{k-1}\}$ is a finite set of state of size $k$, $I$ is
a distribution over $V$, $e$ is $k\times m$ matrix where each row contains
distribution over $\Sigma$ and $a$ is stochastic matrix of size $m\times m$.

We will index elements of $e$ and $a$ by subscripts, therefore $e_{u,v}$ is
element in $u$-th row and $v$-th column of $e$.  

%Therefore following conditions hold:
%\begin{enumerate}
%\item $\forall v\in V, I_v \geq 0$
%\item $\sum_{v\in V}I_v=1$
%\item $\forall u\in V,\forall \sigma\in\Sigma, e_{u,\sigma}\geq0$
%\item $\forall u\in V, \sum_{\sigma\in \Sigma}e_{u,\sigma}=1$
%\item $\forall u,v\in V, a_{u,v}\geq0$
%\item $\forall u\in V, \sum_{v\in V}a_{u,v}=1$
%\end{enumerate}
\end{definition}

\begin{example}\label{EXAMPLE:EXAMPLEHMM} Consider is HMM $H=(\Sigma,V,I,e,a)$ from figure
\ref{FIGURE:EXAMPLEHMM}.  Alphabet is $\Sigma=\{A,C,G,T\}$ and set of states is
$V=\{R,In,E\}$.  Transition and emission distributions are described in figure.
We can define initial distribution to $I_R=0.5, I_E=0.3$ and $I_{In}=0.2$.

%Let $H=(\Sigma,V,I,e,a)$ be an HMM with
%$\Sigma=\{A,C,G,T\}$, $V={I,G}$, $I=(0.2,0.8)$,
%$e_{I,A}=0.1,e_{I,C}=0.2,e_{I,G}=0.3,e_{I,T}=0.4, e_{G,x}=0.25, x\in \Sigma$ and 
%$a_{I,I}=0.9,a_{I,G}=0.1,a_{G,G}=0.95$ and $a_{G,I}=0.05$.
\end{example}

%The conditions $1-6$ ensure that everything is correct probabilistic
%distribution.
\begin{figure}
\begin{center}
\includegraphics{../figures/exampleHMM.pdf}
\end{center}
\caption[Example of simple Hidden Markov Model]{
Example of HMM have $3$ states
that emits symbols from alphabet of size four.  Circles represents states, arc
from state $u$ to $v$ represents probability $a_{u,v}$. Missing arc from state
$In$ to $R$ means that $a_{In,R}=0$.  There are also arcs between states and
fourtouples. Everu fourtuple represents emmision distribution of it's state.
}\label{FIGURE:EXAMPLEHMM} 
\end{figure}


\begin{definition}\label{DEF:STATEPATH}
Let $H=(\Sigma,V,I,e,a)$ be an HMM. We say that there is \firstUseOf{transition}
form state $u$ to state $v$ if $a_{u,v}>0$. We will write transition from $u$ to
$v$ as $u\to v$ and $T=\{u\to v\mid a_{u,v}>0\}$ is set of all transitions of
$H$.

\firstUseOf{State path} $\pi=\pi_0\pi_1\dots\pi_{n-1}$ is a sequence of
states. We say that state path $\pi$ is \firstUseOf{admissible} if $I_{\pi_0}>0$
and  $\pi_{i-1}\to\pi_i\in T$ for all $1\leq i < n$. Otherwise $\pi$ is
\firstUseOf{inadmissible}.
\end{definition}

\begin{example}
Consider HMM from figure \ref{FIGURE:EXAMPLEHMM}. Set of transitions is
$T=\{R\to R,R\to In, R\to E,In\to In, In\to E, E\to E, E\to In, E\to R\}$.
State path $\pi_1=RRRRInER$ is admissible state path and $\pi_2=RRRInREER$ is inadmissible
state path, because contain transition with zero probability ($In\to R$).
\end{example}



%Formally, Hidden Markov Model is defined by it's finite state space
%$Q=\{q_0,q_2,\dots, q_{k-1}\}$ of size $k$, initial probabilistic distribution
%$I$ over state space, finite alphabet $\Sigma = (\sigma_0, \sigma_1, \dots,
%\sigma_{m-1})$ of
%size $m$, emission and transition distribution over $\Sigma$ and $Q$
%respectively defined for every state independently. We denote emission
%distribution of state $q$ by $e_q$ and transition distribution by $a_{q}$. 

Since Hidden Markov Models are generative probabilistic models, we have to
define probability distribution of HMM's products. In following definitions we
will define probability distribution of pairs (state path, sequence) and
probability distribution of sequences generated by HMM.

\begin{definition}
Let $H=(\Sigma,V,I,e,a)$ be an HMM and $X=X_0X_1\dots X_{n-1}$ be a sequence over
alphabet $\Sigma$ of length $n$. Let $\pi$ be a state path of length $n$. Then the
probability that state path $\pi$ generated sequence $X$ is 

\[\prob{X,\pi\mid H}=
I_{\pi_0}e_{\pi_0,X_0}\prod_{i=1}^{|X|-1}a_{\pi_{i-1},\pi_i}e_{\pi_i,X_i}\]

If length of the sequence $X$ is not equal to length of the state path $\pi$,
then
probability that $\pi$ generates $X$ is zero.
\end{definition}



\begin{example}
Consider HMM $H$ from example \ref{EXAMPLE:EXAMPLEHMM}. Let state path be $\pi=RRInInE$ and
generated sequence be $X=ACGTT$. Then the probability that $H$ generates $\pi$
and $X$ is 
$\prob{ X,\pi\mid H } = 0.5 \cdot 0.35 \cdot 0.99 \cdot 0.25 \cdot 0.005 \cdot 0.25 \cdot 0.95 \cdot 0.05 \cdot 0.05 \cdot 0.4 =
0.5143359375  \cdot  10^{-7}$
\end{example}

\begin{definition}
Let $H=(\Sigma,V,I,e,a)$ be a HMM and $X=X_0X_1\dots,X_{n-1}$ be sequence over
alphabet $\Sigma$ of length $n$. The probability that $X$ was generated by the
model $H$ is 
\[\Pr\left(X\mid H\right)=\sum_{\pi\in V^n}\Pr\left(X,\pi\mid H\right)\]
\end{definition}


%The probability of emission of sequence $X$ with state path $\pi$
%is \[\Pr\left(X,\pi\mid H\right) =
%I(\pi_0)e_{\pi_0}\left(X_0\right)\prod_{i=1}^{|X|-1}e_{\pi_i}a_{\pi_{i-1}}\left(\pi_i\right)\]
%Generally, there are many ways how to generate every sequence.  Therefore every
%HMM $H$ defines probability distribution \[\Pr\left(X\mid H\right) =
%\sum_{\pi,|\pi|=|X|}\Pr\left(X,\pi\mid H\right)\].  Since in every step HMM
%generates one symbol from $\Sigma$ and one state, state path has same length as
%generated sequence.

\begin{note}
In our  definition of an HMM, the sum of the probabilities of all sequences of
length $n$ is $1$. Later we will discuss concept of final states. With final
states, the sum of the probabilities of all sequences (of all lengths) is one.

%HMMs can have several final states.
%If final states are present, once HMM reach that state generation of a sequence
%stops. In this chapter we will ignore final states. \todo{Budeme ich vobec
%niekde potrebovat? Final states mozno a zdo poznamky za definiciouu
%pravdepodobnosti sevkvecie (ak ich
%bude treba)}
\end{note}

\todo{Tento text chcem dat asi trosku neskor, Az ked to budem potrebovat}
We will abuse the notation and by $e_q(\cdot)$ we mean row vector
$\left(e_{q,b_1},\dots,e_{q,b_{\sigma}}\right)$. Similarly,
$a_q(\cdot)=\left(a_{q,q_1},\dots,a_{q,q_k}\right)$ is row vector that contain
transition probabilities from the state $q$.  $a_q(\cdot)$ have therefore
dimension $1\times k$.
%By $A$ we will denote the $|Q|\times |Q|$ matrix
%consisting from vectors $a_{q_1},a_{q_2},\dots,a_{q_k}$.  Therefore $A[q_i,q_j]$
%contains probability of transition from state $q_i$ to $q_j$.

Usually there are three main problems associated with
HMMs\cite{}:
\begin{enumerate}
\item Given sequence $X$ and model $H$. What is the probability that $X$ was
generated by model $H$?
\item Given sequence $X$, model $H$ and assumption that $X$ was generated by 
$H$, what is the best explanation of $X$? By explanation is usually meant state
path that generated $X$. We call process of computing explanation from $X$ and
$H$ \firstUseOf{decoding}.
\item Given training data $D$ (usually sequences with ``explanations'') and
topology of model (set of states and transitions), what are the parameters
(initial, transition and emission distributions) that explains train
data?
\end{enumerate} 
In following sections we will discuss several algorithms
that are trying to solve problems describes above.



\section{Forward Algorithm}
The Forward algorithm \cite{Durbin1998} computes for a given sequence $X$ of
length $n$ the probability $\Pr\left(X\mid H\right)$ that sequence was
generated by the model (it solves first problem of HMM). The algorithm is based on dynamic programming. It fills
matrix $F$ of size $n\times m$ where $F[i,v]$ is the probability that $H$
generated $X[:i+1]$ with state path that ends in state $v$ and $m$ is the number
of states of $H$. Values $F[i,v]$ are also called \firstUseOf{forward
variables}. $F[i,v]$ can be computed by following formula.

\begin{align}
F[0,v] &= I_ve_{v,X_0}, v\in V\\
F[i,v] &= \sum_{u\in V}F[i-1,u] \cdot a_{u,v} \cdot e_{v,X_i}, v\in V,0< i < n
\end{align}
%We call values $F[i,v]$ forward variables. Value $F[i,v]$ is the
%probability, that $X[:i+1]$ was generated by state path that ends with state
%$v$. 
The probability that $H$ generated $X$ is 
 \[\Pr\left(X\mid H\right) = \sum_{v\in V} F[|X|-1,v]\]

Using the recurrence equations above, we can compute the probability of $X$ in
$O(nm^2)$ time and $O(m)$ memory dynamic programming where $n$ is the length
of the input sequence and $m$ is the number of states of $H$. If transition
matrix is sparse, then this algorithm can be implemented in $O(n(m+t))$ time
where $t$ is number of transitions.  Forward variables are also used in other
algorithms. 
%Storing all of the forward variables requires soring $O(nm)$ numbers.

\section{Sequence A nnotation}

%definujeme annotation, footprint, set of colors

HMMs can be used for sequence annotation. By sequence annotation we mean
\todo{Predpokladam ze uz som v prvej kapitole vysvetlil co je gen a tak}
assigning ``labels'' to parts of the input sequence according to their meaning.
Now we give few examples.

\paragraph{Gen e finding:} in living organisms parts of DNA sequences are
translated into proteins. Those parts of sequences which were used during
translation to proteins (we include also introns) are called genes. In gene
finding we want to find which parts of the input sequence are genes and which
are not. 

%For example in the gene-finding domain, we want to assign gene labels to those
%parts of the sequence that encode proteins. To annotate transmembrane proteins,
%we want to know, which parts of the sequence are on which  side of the membrane.
%In recombination prediction we want to predict which parts of the sequence
%belong to which subtype of an organism.  \todo{rozviest priklady, alebo zrusit}

\paragraph{Transmembrane proteins:} Some proteins in cells goes from one side of
membrane to other side of membrane (usually they pass membrane several times).
Aim of transmembrane proteins is to predict which parts of the protein sequence
is on which part of membrane and which parts are in membrane.

\paragraph{Recombination prediction:} Some organisms, for example HIV virus
, evolves rapidly and they have been classified into several subtypes.
Moreover, some viruses are mosaic combination of viruses from different
subtypes. For example beginning and end of RNA sequence of virus can be from 
one subtype and the middle of a sequence is from different subtype. In
recombination prediction we want to annotate RNA sequence of a virus by subtypes
from which that part of sequence originates.
\paragraph{} 
In general, we have a finite set of labels $C=\{c_0,c_1,\dots,c_{l-1}\}$ and we
want to assign one label to every symbol of the input sequence. We do it by
assigning one label to every state of HMM.
%in a way, that states of HMM that
%encodes same meaning have same color. After that we try 
To predict the annotation of the input sequence $X$, we can for example find the
most probable state path that can generated $X$ and we annotate each symbol of
the sequence $X$ according to the label assigned to state that generated that
symbol. We can formalized this in the following way.

\todo{Aby nebol bordel medzi label a color}

\begin{definition}\label{DEFINITION:ANNOTATION}
Let $H=(\Sigma,V,I,e,a)$ be HMM and $C=\{c_0,c_1,\dots,c_{l-1}\}$ be the finite
sets of labels (or colors). Then \firstUseOf{coloring funcion} 
$\lambda: V^*\to C^*$ is function that satisfies following properties:
\begin{enumerate}
\item $\lambda(v)\in C$ for all $v\in V$.
\item $\lambda(xy) = \lambda(x)\lambda(y)$ for all $x,y\in V^*$.
\end{enumerate}

Let $X$ be sequence generated by state path $\pi$. Then annotation
$\Lambda$ of sequence $X$ is $\Lambda = \lambda(\pi)$.
\end{definition}

In sequence annotation, we do not know the state path that generated a given
sequence. Our goal is to reconstruct the state path or at least to give a good
approximation of the correct annotation. Note that in general several state
paths can have the same annotation.

%Since many states of the HMM can have assigned same label, there may be may be
%many state path with same annotation.

\begin{definition}
Let $H$ be an $HMM$, $X$ be a sequence of length $n$, $\Lambda$ be an annotation of sequence
$X$. The probability of annotation $\Lambda$ given sequence $X$ is 
\begin{equation}
\Pr\left(\Lambda\mid X,H\right)=\sum_{\pi \in V^n,\lambda(\pi) =
\Lambda}\Pr\left(\pi\mid X,H \right)\label{DEF:ANNOTATION:PROBABILITY}
\end{equation}
\end{definition}

Note that $\prob{\pi\mid X,H}=\frac{\prob{\pi,X\mid H}}{\prob{X\mid
H}}$

\begin{example}\label{EXAMPLE:ANNOTATION}
Let $H$ be HMM from example \ref{EXAMPLE:EXAMPLEHMM}. Let $C=\{I,G\}$ and
$\lambda(R)=I$ and $\lambda(In)=\lambda(E)=G$.  Then annotation of sequence
$X=AACT$, which was generated by state path $\pi=RInEE$, is $\lambda(RInEE) =
IGGG$. 

We will use following interpretation of the model $H$. We can imagine $H$ as very simple gene
predictor\footnote{Not very realistic}. $R$ represents intergenic regions, state $In$
represents introns and $E$ represents exons. Annotation symbol $I$ represents
intergenic region and annotation symbol $G$ represents regions that are genes
(we recall that genes consists from introns and exons).

Using this interpretation, sequence $AACT$ contains gene $ACT$. There are
several explanations of this ``fact'': if $AACT$ was generated by state path
$REEE$ then subsequence $EEE$ is exone; if it was generated by state path $REIE$
then sequence contain two exons and one intron in the middle. There are $2^3$
different state paths $\pi$ with $\lambda(\pi)=IGGG$.  All of those state path
support ``fact'' that $IGGG$ is annotation of $X$.

\end{example}


%\begin{example}
%Consider again HMM from example \ref{EXAMPLE:EXAMPLEHMM} and annotation function
%from example \ref{EXAMPLE:ANNOTATION}.
%\end{example}

One natural question is, given sequence $X$ what is the best annotation of $X$.
One measure of quality of an annotation is it's probability. The more probable
is annotation, the more likely $X$ was generated with state path with such
annotation. We can formulate this in following problem.

\begin{definition}
Given HMM $H$ and sequence $X$, \abbreviation{the most probable annotation
problem}{MPA} is the problem of finding an annotation of $X$ that maximizes
the probability $\prob{\Lambda\mid X,H}$.
\end{definition}

\begin{theorem}
Most probable annotation problem is NP-hard.
\end{theorem}
This theorem was proved in 2002 by Lyngsø {\it et al.} and can be found in
\cite{Lyngso2002}. Their proof was done by conversion to maximal clique problem.
For input graph with $n$ vertices they construct HMM with $O(n^2)$ states and
sequence of length $n$ which most probable annotation could be converted into
maximal clique of input graph. 

\begin{theorem}
There exists an HMM such that it is HP-hard to find the most probable annotation
to given sequence $X$.
\end{theorem}
The proof of this theorem can be found in \cite{Brejova2007mpa}. This paper also
contain polynomial algorithm (Extended Viterbi Algorithm) that finds most
probable annotation for special classes of HMMs. 



\section{The Viterbi Algorithm}
%\todo{Viterbi variable ma rovnake oznacenie ako mnozina stavov. Treba to
%upravit} -- Brona tvrdi ze netreba
The Viterbi algorithm  is probably the most frequently used
\correction{decoding}{Zaviedli sme to len vagne} algorithm for Hidden Markov
models \cite{Durbin1998}.
The Viterbi algorithm answer a straightforward question: given the sequence
$X=X_0X_1\dots X_{n-1}$, what
is the most-likely state path $\pi$ that generates $X$? Formally, Viterbi
algorithm finds a state path maximizing $\Pr\left( \pi\mid X,H \right)$. Since
\[\Pr\left(\pi\mid X,H\right) = \frac{\Pr\left(\pi,X,\mid
H\right)}{\Pr\left(X\mid H\right)}\] and quantity $\Pr\left(X\mid H\right)$ is
fixed,
this is same as finding $\pi$ maximizing $\Pr\left(X,\pi\mid H\right)$. 

The Viterbi algorithm is very similar to the  Forward algorithm. It starts with computing
Viterbi variables $V[i,v]$. Variable $V[i,v]$ stores the probability of the most probable 
state path that generated $X[:i+1]$ and ends in state $v$. The algorithm
also computes back-links $B[i,v]$ that contain the previous state in the most
probable state path that generated $X[:i+1]$ and ends in state $v$. We can
compute these values by the following equations:
\begin{align}
V[0,v] &= I_{v}e_{v,X_0}, v\in V\\
V[i,v] &= \max_{u\in V} V[i-1,u]a_{u,v}e_{v,X_i}, v\in V,0<i<n\\
B[i,v] &= \arg\max_{u\in V} V[i-1,u]a_{u,v}e_{v,X_i}, v\in V,0<i<n
\end{align}
\begin{note}
Values $B[0,v],v\in V$ are not needed in the algorithm.

Note the similarity of the Viterbi algorithm and the Forward algorithm.
We can obtain the Viterbi equations from the forward equations by replacing
summation with
maximization.
\end{note}


Variable $V[n-1,v]$ contains the probability of the most probable state path
that generated $X$ and end in state $v$. Therefore the state $v_{\max} =
\arg\max_{v\in V}V[n-1,v]$ is the last state of the most probable state path.
Variable $B[n-1,v_{\max}]$ contains the previous state of the most probable
state path. By traversing back through back-links  we can reconstruct the most
probable state path.

Time complexity of the Viterbi algorithm is $O(nm^2)$ where $m$ is the number of
states. Similarly to the Forward algorithm, if transition matrix is sparse then 
the Viterbi algorithm can be implemented in $O(n(m+t))$ time where $t$ is the
number of transitions of HMM.

%\begin{example}
%Example of Viterbi Table with back-links
%\end{example}

We can use the Viterbi algorithm to annotate sequence $X$ by finding the most
probable state path $\pi$ and then computing $\lambda(\pi)$. If the  coloring
function $\lambda$ is the identity function, this will find the most probable
annotation, but in general, Viterbi algorithm not even a good approximation of
the most probable annotation as shown in the following example.

\begin{figure}
\begin{center}
\includegraphics{../figures/multiplePathProblemHMM.pdf}
\end{center}
\caption[Hidden Markov Model with multiple path problem.]{Example of HMM with
multiple path problem. States $w_1,w_2,b$ emits symbol $0$ with probability $1$
and state $e$ emits symbol $1$ with probability $1$. Initial distribution is set
to $I_{w_1}=I_{w_2}=\frac14, I_{b}=0.5$
and $I_e=0$. Annotations of states are represented by colors of states
($\lambda(w_1)=\lambda(w_2)$ and $\lambda(b)=\lambda(e)$. }\label{FIGURE:BADVITERBIEXAMPLE}
\end{figure}

\begin{example}
Consider HMM from figure \ref{FIGURE:BADVITERBIEXAMPLE}. Take sequence $X=0^n1$.
Since state $e$ is only one that can emit $1$, every state path with non-zero
probability ends in state $e$. If state path starts in state $b$ then state
path has form $b^ne$. Annotation of such state path is $\Lambda_1=\lambda(b)^{n+1}$
and probability of annotation $\Lambda_1$ (and also state path $b^ne$) is $p^n(1-p)$.
If state path starts in one of the white states, then state path
has form $w'_0w'_1\dots w'_{n-1}e$ where $w'_i$ is either $w_1$ or $w_2$.
There are $2^n$ such state paths and each of them have probability $0.5\cdot
0.25^n$ and annotation $\Lambda_2=\lambda(w_1)\lambda(e)$. Probability of
annotation $\Lambda_2$ is therefore $0.5^{n+1}$.
If $n$ is sufficiently high and $\frac14<p<\frac12$ then most probable state
path is $b^ne$ which corresponds to annotation $\Lambda_1$. However, the most
probable annotation is $\Lambda_2$ and it's probability is exponentially higher
then probability of $\Lambda_1$. Therefore the Viterbi algorithm in to even a
good approximation for the most probable annotation problem.

We say that HMM has \firstUseOf{multiple path problem} if it has annotation that
corresponds to more than one state path.
%$\Lambda_2i$
%is greater then probability if $\Lambda_1$ but the most probable state path is 
%the $b^ne$ and i
\end{example}


%\section{Backward Algorithm}
%Backward algorithm is backward version of the Froward algorithm. ???
\section{Forward-Backward Algorithm and Posterior Decoding}
\firstUseOf{Posterior Decoding} is another commonly used decoding method. In
contrast to the Viterbi algorithm, posterior decoding assign a label individually to
every symbol of input sequence and does not care about the overall structure of
reconstructed state path. 

Given sequence $X$, posterior decoding finds state path $\pi$ with the following
property:
\[\forall 0\leq i< n, \pi_i=\arg\max_{v\in V}\Pr\left(\pi_i=v\mid X,H\right) \]
where \[\Pr\left(\pi_i=v\mid X,H\right) = \sum_{\pi\in V^n,\pi_i=v}\Pr\left(\pi\mid X,H\right)\]

Values $\Pr\left(\pi_i=v\mid X,H\right)$ can be computed using Forward-Backward
algorithm. This algorithm computes $\Pr\left(\pi_i=v\mid X,H\right)$ for every
combination of position $i$ and state $v$. Then for every position chooses the
state that maximizes posterior probability.

Now we derive formula for efficient computation of  $\Pr\left(\pi_i=v\mid
X,H\right)$.
\begin{align}
\Pr\left(\pi_i=v,X\mid H\right) &=  
%\sum_{\pi\in V^n,\pi_i=v}\Pr\left(\pi\mid X,H\right)
%					\label{PosteriorDer1} \\
%				&=& 
				\sum_{\pi\in
				V^n,\pi_i=v}I_{\pi_0}e_{\pi_0,X_0}\prod_{j=1}^{n-1}e_{\pi_j,X_j}a_{\pi_{j-1},\pi_j}
					\label{PosteriorDer2}\\
%				&= \sum_{\pi\in V^n,\pi_i=v}I_{\pi_0}e_{\pi_0,X_0}
%				\left( 
%					\prod_{j=1}^{i-1} e_{\pi_j,X_j}a_{\pi_{j-1},\pi_j}
%				\right)
%				a_{\pi_{i-1},\pi_i}e_{\pi_i,X_i}
%				\left(  
%					\prod_{j=i+1}^{n-1} e_{\pi_j,X_j}a_{\pi_{j-1},\pi_j}
%				\right)
%					\label{PosteriorDer3}\\
				&= \sum_{\pi\in V^n,\pi_i=v}I_{\pi_0}e_{\pi_0,X_0}
				\left( 
					\prod_{j=1}^{i-1} e_{\pi_j,X_j}a_{\pi_{j-1},\pi_j}
				\right)
				a_{\pi_{i-1},v}e_{v,X_i}
				\left(  
					\prod_{j=i+1}^{n-1} e_{\pi_j,X_j}a_{\pi_{j-1},\pi_j}
				\right)
					\label{PosteriorDer4}\\
				&= 
				\left(
					\sum_{\pi\in V^{i+1},\pi_i=v}I_{\pi_0}e_{\pi_0,X_0}
					\left(
						\prod_{j=1}^{i} e_{\pi_j,X_j}a_{\pi_{j-1},\pi_j}
					\right)
				\right)
				\left( 
					\sum_{\pi\in V^{n-i},\pi_0=v}
					\prod_{j=1}^{n-i-1} e_{\pi_j,X_{j+i}}a_{\pi_{j-1},\pi_j}
				\right)
					\label{PosteriorDer5}\\
				&= F[i,v]
				\left( 
					\sum_{\pi\in V^{n-i},\pi_0=v}
					\prod_{j=1}^{n-i-1} e_{\pi_j,X_{i+j}}a_{\pi_{j-1},\pi_j}
				\right)
					\label{PosteriorDer6}
\end{align}
%Equations \ref{PosteriorDer2} and \ref{PosteriorDer3} is from definition of
%posterior probability. In \ref{PosteriorDer4} we just replace $\pi_i$ with $v$.
%Next equality we have from the fact that the left part of the formula
%\ref{PosteriorDer4} uses only left part of the state path and right part of the
%formula uses only right part of the state path. Last equation follows from
%definition of $F[i,v]$.
The right part of the formula \ref{PosteriorDer6} has
similar structure as $F[i,v]$, but it uses last part of the state path and
sequence and also it lack initial distribution. We can compute 
this formula using  the Backward algorithm, which is very similar to forward
algorithm. Let $B[i,v]$ be the right part of formula \ref{PosteriorDer6}.
\begin{align}
B[i,v]&=
\sum_{\pi\in V^{n-i},\pi_0=v}
	\prod_{j=1}^{n-i-1}
		e_{\pi_j,X_{i+j}}a_{\pi_{j-1},\pi_j}\\
% &= 
% \sum_{u\in V}
% 	e_{u,X{n+j-1}}a_{v,u}
%	\sum_{\pi\in V^{n-i},pi_0=v,\pi_1=u}
%		\prod_{j=2}^{n-i-1}
%			e_{\pi_j,X_{n+j-1}}a_{\pi_{j-1},\pi_j}\\
 &= 
 \sum_{u\in V}
 	e_{u,X_{I+1}}A_{V,u}
	\sum_{\pi\in V^{n-i-1},\pi_0=u}
		\prod_{j=1}^{n-i-2}
			e_{\pi_j,X_{i+j+1}}a_{\pi_{j-1},\pi_j}\\
 &= 
 \sum_{u\in V}
 	e_{\pi_j,X_{i+1}}a_{v,u}B[i+1,u]
\end{align}
If we set $B[n-1,v]$ to $1$, we have a recurrence to compute values
$B[i,v]$. This recurrence is very similar to recurrence $F[i,b]$ Recall that
\[F[i,v] = \sum_{u\in V}F[i-1,u] a_{u,v} e_{v,X_i}\]
$B[i,v]$ depends on next positions in sequence $X$,
while $F[i,v]$ depends on previous positions in sequence $X$. Another
difference is that $B[i,v]$ does include emission of $X[i]$ while $F[i,v]$ does.

In addition, it is possible to compute probability
of $X$ being generated by the model with backward algorithm.

\begin{align}
\prob{X\mid H} &= 
	\sum_{\pi\in V^n}
		e_{\pi_0,X_0}\prod_{i=1}^{n-1}e_{\pi_i,X_i}a_{\pi_{i-1},\pi_i}\\
	&=
	\sum_{u\in V}e_{u,X_0}
	\sum_{\pi\in V^n,\pi_0=u}
		\prod_{i=1}^{n-1}e_{\pi_i,X_i}a_{\pi_{i-1},\pi_i}\\
	&=
	\sum_{u\in V}e_{u,X_0}B[0,u]
\end{align}

Posterior probabilities can be computed by
\[\Pr\left(\pi_i=v\mid X,H\right) = \frac{F[i,v]\cdot B[i,v]}{\Pr\left( X\mid H
 \right)}\]


We can compute these values in $O(nm^2)$ time and $O(nm)$ memory. Again, in case
of sparse transition matrix, this algorithm can be altered to run in $O(n(m+t))$
time where $t$ is the number of transitions. We will
discuss possible algorithmic improvements of this algorithm later.

One drawback of an  posterior decoding is that it can reconstruct inadmissible
state path. One example is given below.

\begin{figure}
\begin{center}
\includegraphics{../figures/posteriorInadmissibleStatePath.pdf}
\end{center}
\caption[Hidden Markov Model on which posterior decoding reconstructs
inadmissible state path]{
Example of HMM on which posterior decoding reconstructs inadmissible state path. 
All unlabeled transitions are even ($0.5$). States $w_1,w_2$ and $v$ emits $0$
with probability $1$ and state $e$ emits $1$ with probability $1$.
Initial distribution is set to $I_{w_1}=I_{w_2}=0.3, I_{v}=0.4, I_{e}=0$.
}\label{FIGURE:INADMISSIBLESTATEPATH}
\end{figure}

\todo{Je mozne, ze by sa TU mohlo spomenut a maximalizacii ocakavaneho poctu
spravne predikovanych stavov}

\begin{example}
Consider HMM from figure \ref{FIGURE:INADMISSIBLESTATEPATH}. Let $X=001$. There
are three state paths that have non-zero probability:
$\pi_1=w_1w_2e,\pi_2=w_2w_1e$ and $\pi_3=vve$. Their probabilities are
$\prob{\pi_1\mid X,H}=\prob{\pi_2\mid X,H}=0.3, \prob{\pi_3\mid X,H}=0.4$.
Posterior probabilities are in table \ref{TABLE:INADMISSIBLESTATEPATH}.
As we can see, maximal posterior probability for the first position have state
$v$. For the second position it is state $w_2$ and for the third it is state
$e$. Therefore PD will reconstruct state path $vw_2e$. However, this probability
is inadmissible.

\begin{table}
\begin{center}
\begin{tabular}{|l|c|c|c|}
\hline
State/Position & $X_0$ & $X_1$ & $X_2$ \\\hline
$w_1$ & $0.3$ & $0$ & $0$ \\\hline
$w_2$ & $0.3$ & $0.6$ & $0$ \\\hline
$v$   & $0.4$ & $0.4$ & $0$\\\hline
$e$  & $0$ & $0$ & $1$ \\\hline
\end{tabular}
\end{center}
\caption[Example of posterior probabilities.]{Posterior probabilities for
sequence $X$ and HMM $H$ from figure \ref{FIGURE:INADMISSIBLESTATEPATH}.
}\label{TABLE:INADMISSIBLESTATEPATH}
\end{table}

\end{example}


\section{Training} 

\bigskip
{\large\bf Tato cast este nie je napisana. Lepsie je skocit na dalsiu sekciu --
chcem najskor zapracovat pripomienky}
\bigskip

Training is process of estimating parameters of probabilistic models. In this
section we will describe several approaches how to estimate transition and
emission distributions. We will describe several approaches to estimate
parameters of HMM. We will consider only estimating emission and transition
probabilities, not topology of a model.

We will use \firstUseOf{maximal likelihood approach}. Let $H_{\theta}$ be an HMM
where $\theta$ is vector containing emission and transition probabilities and
$D$ be training data (sequences, possibly with state paths).  We
will try to find $\theta$ that maximizes likelihood of the data.

\[L(H_\theta\mid D)=\prob{D\mid H_\theta}\]

\todo{definovat $\sim$}

Where date $D$ is set of pairs $(X,\pi)$ where $X$ is sequence and $\pi\in
V\cup\{*\}$. $*$ means missing information in state path. Probability that data
were generated is
\[
\prob{D\mid H_\theta} = \prod_{(X,\pi')\in D}\prob{X,\pi'\mid H_\theta}
\]
where
\[\prob{X,\pi'\mid H_\theta} = \sum_{\pi\sim\pi'}\prob{X,\pi\mid H_\theta}\]

In following paragraph we briefly discuss several methods of estimation of
parameters.

\paragraph{Fully labeled data:} Consider situation when $D$ consists of pairs
$(X,\pi)$ where $\pi$ is full state path (without symbol $*$). In such situation
we use frequencies of occurred events as the parameters of the model.
Let $A_{u,v}$ be the number of transitions from $u$ to $v$ in $D$.
Let $E_{u,x}$ be the number times when state $u$ emitted $x$ in training data
$D$.
Then 
\begin{align*}
&a_{u,v}=\frac{A_{u,v}}{\sum_{w\in V}A_{u,w}}
&e_{u,x}=\frac{E_{u,x}}{\sum_{y\in\Sigma}E_{y,x}}
\end{align*}
for all $u,v\in V, x\in\Sigma$. $e$ and $a$ maximizes the likelihood
\cite{Durbin1998}. If case of insufficient data some events that in reality have
nonzero probability may not occur and therefore it's probability will be
estimated to zero. To avoid this behavior we can use pseudo-counts
\cite{Durbin1998}. We artificially add constant $k_x$ to all events $x$ that we expect
to have non-zero probability.

\paragraph{The Baum-Welsch training:} Baum-Welsch training is instance of more
general expectation maximization algorithm. It is iterative algorithm that
computes expectations of missing data and use such expectations to compute 
new model. This algorithm has guarantee that it will converge to local maxima.
Expectations are computed with variant of Forward-Backward algorithm.
\paragraph{The Viterbi training: }

Viterbi training is similar to Baum-Welsch training, but instead of computing
expectations for missing data by Forward-Backward algorithm, we compute the most
probable state path with the Viterbi training and estimate new model with by
from frequencies of events. Advantage of this approach is that in practice this
training is faster than Baum-Welsch training. However, the Viterbi training and
Baum-Welsch training does not have to converge to same parameters
\cite{Durbin1998}.

In practice we can use the Viterbi training for the estimation of the reasonably
good starting estimation for Baum-Welsch training. One such application can be
found in \cite{FEAST2011}.

%\subsection{Other Training Methods}
%There are several other approaches how to similar
\section{Variants of Hidden Markov Models}

In this section we describe several variants of hidden Markov Models.
Some of these extension have more expressive power but mostly they are
introduced for simplifying the models.

\subsection{Silent states}

One common variant of HMMs are HMMs with silent states. Silent state are states
that does not emit any symbol. One of the consequences is  that a state path can
be longer than the emitted sequence. However, the number of non-silent states in
the state path has to be equal to the sequence length. Silent states does not
add any expressing power to HMMs, but in some cases they allow to reduce the
number of transitions by a factor $m$. This can be used to decrease number of
parameters and speed up algorithms. 

\begin{definition}
Hidden Markov Model with silent states is a:w
tuple $H=(\Sigma,V,Q,I,e,a)$
where $\Sigma,V,I,a$ are defined as in definition \ref{DEF:HMM}. $Q\subseteq V$ is set of
silent states. $e$ must satisfy following conditions:
\begin{enumerate}
\item $\forall u\in V\backslash Q,\forall \sigma\in\Sigma, e_{u,\sigma}\geq0$
\item $\forall u\in Q,\forall \sigma\in\Sigma, e_{u,\sigma}=0$
\item $\forall u\in V\backslash Q, \sum_{\sigma\in \Sigma}e_{u,\sigma}=1$
\end{enumerate}
\end{definition}

\begin{definition}
Transitions and state path are defined as in definition \ref{DEF:STATEPATH}. 

Let $\pi$ be a state path. \firstUseOf{Non-silent state path} $\pi^Q$ is
the maximal subsequence of $\pi$ that consists of non-silent states.
\end{definition}

\begin{note}
Non-silent state path $\pi^Q$ can by obtained from a state path $\pi$ by removing all silent
states.
\end{note}

\begin{definition}
Let $H=(\Sigma,V,Q,I,e,a)$ be a HMM with silent states and $X=X_0X_1\dots
X_{n-1}$ be a sequence over
alphabet $\Sigma$ of length $n$. Let $\pi$ be a state path for which $\pi^Q$ has
length $n$. Then the probability that state path generated sequence $X$ is 

\[\Pr\left(X,\pi\mid H\right) =
I_{\pi_0}\left(\prod_{i=1}^{|\pi|-1}a_{\pi_{i-1},\pi_i}\right)\left(\prod_{i=0}^{|X|-1}e_{\pi^Q_i,X_i}\right)\]

If length of the sequence $X$ is not equal to the length of the non-silent state
path $\pi^Q$, then the
probability that $\pi$ generates $X$ is zero.
\end{definition}

\begin{example}
Example of HMM with silent states that reduces the number of transitions by factor
of $\Theta(m)$.

Consider following HMM $H=(\Sigma,V,I,e,a)$ with $m$ states such that every row
of $a$ is uniform distribution over $V$. In other words, there is transition
between any two pairs of states from $V$, that is exactly $n^2$
transitions\footnote{$u\to u$ is also transition}.

We create a new HMM $H'$ by removing all transitions from $H$, adding one silent
state $s$ and adding transitions
from state $s$ to all states from $V$ with probability $\frac1m$  and from every
state $V$ to $s$ with probability $1$. This new HMM
defines the same distribution of sequences, but has one more state and only $2m$
transitions. 

The Viterbi algorithm, the Forward algorithm or Posterior decoding can be
implemented for $H'$ in $O(nm)$ time while these algorithm will have time
complexity $O(nm^2)$ for $H$.
\end{example}

\todo{Dodaj ze silent states sa daju aj urvat a ze tam nesmie byt cyklus}


\subsection{Start and Final state}

Sometimes it is useful to have special start state and special final state. We
will describe these two states separately, since they affect models in different
ways. 

Start states can be used instead of initial distribution. If state $s$ is start
state, then $I_v=1$. Therefore generative process always start in start state.
We can model initial distribution by silent start state $s$ with $a_{s,v}=I_v$ for all $v$.

Final states affects distribution of model. HMM defined in section
\ref{SECTION:HMMDEF} define distribution over sequences of same length. HMM with
final states defines distribution over sequences of all length.  We have to make
few changes to definition of HMM to have final states.
We add set of final states $F\subseteq V$. Transitions from final states are not
defined (or set to zero). Emission distribution might be defined (if not, final
states are silent). Every state path has to end with state that is final and
final state can be only in the end of a state path.

With final states, HMM stops generating a sequence once it reaches some final
state. Therefore sum of the probabilities of all sequences is $1$. Recall
that sum of probabilities of all sequences with length $n$ is $1$ for HMM
without final states.

Final states slightly affects algorithms. For example in the Forward algorithm we do
not have to change recurrences, only the final summation: probability of the sequence is $\sum_{q\in F}F[n-1,q]$. 
Similarly, in the Viterbi algorithm
we have to find $\max_{q\in F} V[n-1,q]$ not $\max_{q\ni V} V[n-1,q]$. 
The Backward changed by setting $B[n-1,q]$ to zero for all $q\notin F$.

\subsection{High Order HMMs}

Sequence $X$ and state path $\pi$ generated by the HMM can by considered 
as sequences of random variables
$X_0,X_1,\dots, X_{n-1}$ and $\pi_0,\pi_1,\dots,\pi_{n-1}$.
Random variables associated with the state path have the Markov
property\cite{Levin2006}, which means that $\pi_i$ depends only on $\pi_{i-1}$ or
more precisely
$\prob{\pi_i\mid\pi_0,\dots,\pi_{i-1}}=\prob{\pi_i\mid\pi_{i-1}}$. Similarly,
$X_i$ depends only on $\pi_i$, that is
$\prob{X_I\mid\pi_0,\dots,\pi_i,X_0,\dots,X_{i-1}}=\prob{X_i\mid\pi_i}$.
However,
sometimes the  ability to look back more than just one state or symbol is useful.

Therefore we can extend transition and emission probabilities to depend on
several previous states/emissions. It is possible to depend on whole previous
sequence, however it increase running time of decoding and training algorithms.
If we depend only on all previous emissions, time complexity of such algorithm
will increase by factor of $n$. If we depend on all previous state paths then
time complexity of those algorithms can increase exponentially.  Therefore
practical solutions that are using high-order HMM depends only on small number
of previous states/emissions \cite{Brejova2005,dalsie}.

We will briefly discuss $k$-th order HMM where emissions are dependent on
previous $k$ emissions. Specifically, $X_i$ depends only on $\pi_i$ and
$X_{i-k},\dots,X_{i-1}$. Emission distribution is therefore parametrized be
state and previous emissions, which is string of length at most $k$ (it can be
shorter in the beginning of the sequence). $e_{u,x,a}$ is probability that
state $u$ emits $a$ under the condition that $x$ is suffix of already emitted
sequence. Let $X$ be an sequence and $\pi$ be an state path. Then definition
of probability that $X$ and $\pi$ were generated by the model changes to

\[
\prob{X,\pi\mid H} = 
I_{\pi_0}e_{\pi,\varepsilon,X_0}\prod_{i=1}^{n-1}
e_{\pi_i,X[i-\min\{k,i\}:i],X_i}a_{\pi_{i-1},\pi_i}
\]
where $\varepsilon$ is empty string. Other definitions will not change. We can
use all algorithms that we have described above but we have incorporate into
them these new emission distributions.


\subsection{Generalized HMMs}

Consider a state $v$ with a self-transition, i.e. a state for which $e_{v,v}>0$.
The number of steps the model remains in this state is distributed gemetrically.
In particular, the probability that we will leave state $v$ after exactly $k$ steps is
$e_{v,v}^{k-1}(1-e_{v,v})$. For some
applications in speech processing and genefinding is this behaviour not
appropriate\cite{}.

A \abbreviation{generalized HMM}{GHMM} (or hidden semi-Markov model)
has with every state $v$ associated a duration distribution $d_v(\cdot)$.  When
a GHMM enters state $v$, it first samples length $l$ according
to distribution $d_v$.  Afterwards it generates string $x$ of length $l$.  To
specify probability $e_{v,x}$, each symbol of generated string $x$ is usually
generated independently, which means that $e_{v,x}=\prod_{i=0}^{|x|-1}e_{v,x}$.
Output of a GHMM are three sequences: state path $\pi=\pi_0\pi_1\dots\pi_{l-1}$,
\firstUseOf{duration sequence} $D=D_0D_1\dots D_{l-1}$ and sequence
$X=X_0X_1\dots X_{n-1}$.  State path and duration sequence has same
length and \[\sum_{i=0}^{|D|-1}D_i = |X|\].

Manipulation with GHMMs is  more complicated and technical. If one state can
generate strings with various length then in general from state space we are
not able to uniquely assign to every state $v$ in $\pi$ the symbols from
sequence which were generated by $v$. To do so, we need also duration sequence.
Let $D^i = \sum_{j=0}^{i}D_j$ and $D^{-1}=0$.
The probability that $\pi$ with $D$ generated sequence $X$ is
\begin{equation}
\Pr(X,D,\pi\mid H) = 
\left(
\prod_{i=0}^{|\pi|-1}
d_{\pi_i}(D_i)e_{\pi_i,X[D^{i-1}:D^i]}
\right)
\prod_{i=1}^{|\pi|-1}
a_{\pi_{i-1},\pi_i}
\end{equation}
Similarly as with regular HMMs,  probability of the observed
sequence is
\begin{equation}
\Pr\left(X\mid H\right) = \sum_{\pi,D}\Pr\left(X,\pi,D\mid X\right)
\end{equation}
Computing this value can be done by a variant of the Forward algorithm. More
details \todo{skontroluj si tu casovu zlozitost, ci to nejde nahodou lepsie}
in \cite{}. Maximizing likelihood of $\pi$ and $D$ can be found by variant of
the Viterbi algorithm, also found in \cite{}. However, time complexity of those
algorithm on GHMM is higher. The Viterbi algorithm and the Forward algorithm
run in $O(n^2m^2)$ time \cite{}.

\section{Pair Hidden Markov Models}\label{SECTION:PAIRHMM}
In this section we will describe hidden Markov models that generate output on
two tapes, resulting in two emitted sequences. These models are used in
bioinformatics to study relations between different sequences.

%\subsection{Pair Hidden Markov Models}

%Unlike previous variant of HMMs, \abbreviation{pair hidden Markov models}{pHMM}
%are not equivalent to standard HMMs defined in section \ref{SECTION:HMMDEF}.
%pHMM generates pair of sequences.i
Every state can in one step generate one symbol in each sequences, or one symbol
in one of the sequences or no symbols at all
Formally, every state generates a pair of strings $(a,b)$, where $a$
and $b$ are of length at most one.
Moreover, all pairs with generated with nonzero probability by one state have
the same length.  Formal definition of pair HMM is given below.

Aim of this type of model is to describe measures of similarities of sequences.
In terms of alignments, symbols generated by same state are considered
homologous (are in same column of an alignment). Symbols that are generated by a
state that generate in only one sequence are aligned to a gap. Therefore we can
use pHMM to define probabilistic scoring schemes for alignments.

\todo{Co znamena to par}

\begin{definition}
A \abbreviation{pair hidden Markov model}{pHMM} is a tuple $H=(\Sigma,V,I,d,e,a)$, where $\Sigma$ is finite
alphabet, $V$ is a finite set of states, $I$ is an initial distribution and $a$ is
a transition matrix, all defined as
in definition \ref{DEF:HMM}. $d^x_v$ and $d^y_v$ is state duration of state $v$
in sequence $x$ and $y$ respectively. For all $v\in V$,
$d^x_v\in \{0,1\}$ and $d^y_v\in \{0,1\}$.
Emission probability matrix $e$ is
a $|V|\times\left(|\Sigma\cup\{\varepsilon\}|\right)^2$ matrix with following
properties:
\begin{enumerate}
\item
$\forall v\in V\forall \sigma_1,\sigma_2\in\Sigma\cup\{\varepsilon\}:
0\leq e_{v,(\sigma_1,\sigma_2)}\leq 1$

\item 
$\forall v\in V:
\sum {\sigma_1,\sigma_2\in\Sigma\cup\{\varepsilon\}}e_{v,(\sigma_1,\sigma_2)} = 1$

\item For all states $v$ if $e_{(v,\sigma_1,\sigma_2)}>0$ then
$d^x_v=|\sigma_1|$ and $d^y_v=|\sigma_2|$
%$e_{v,(\sigma'_1,\sigma'_2)}>0$ then $|\sigma_1|=|\sigma'_1|$ and  
%$|\sigma_2|=|\sigma'_2|$.
\end{enumerate}

\end{definition}

A state path is defined same for HMMs with silent states. State duration
ensures
that there is only one way how to assign which symbol from each sequences was
generated by which state. 

However, if given only two sequences $X$ and $Y$ and no state path, we are not
able to tell which symbols of the two sequences were generated together. Before
we continue, we will define cumulative duration times which will make
expressions simpler.

\begin{definition}
Let $\pi=\pi_0\dots\pi_l$ be a state path. Then \firstUseOf{cumulative duration
times} are
$D^x_i(\pi)=\sum_{j=0}{i}d^x_{\pi_j}$ and $D^y_i(\pi)=\sum_{j=0}^{i}=d^y_{\pi_j}$.
Additionally, $D^x_{-1}(\pi)=D^y_{-1}(\pi)=0$. If it will be clear from context
which state path we are using, we will write $D^x_i$ and $D^y_i$ instead of
$D^x_i(\pi)$ and $D^y_i(\pi)$.
\end{definition}

Given sequences $X,Y$ and state path $\pi$ that generated them, we 
can tell which symbols were generated by which states. Since every state $v$
generate exactly $d^x_v$ and $d^y_v$ symbols in $X$ and $Y$ respectively,
state $\pi_i$ generated pair $(X[D^x_{i-1}:D^x_{i}],Y[D^y_{i-1}:D^y_{i}])$
(states $\pi_0,\pi_1,\dots\pi_{i-1}$ generated first $D^x_{i-1}$ symbols in $X$
and first $D^y_{i-1}$ symbols in $Y$). 


\begin{example}
In figure \ref{FIGURE:EXAMPLEPAIRHMM} is pair hidden Markov models modeling
sequence alignment with affine gap model. State $M$ is called match state and
generates pair of aligned residues and correspond to matrix $M$ from the sequence
alignment algorithm. States $I_X$ and $I_Y$ represents gaps,
generates residues in only one sequence and corresponds to matrices $M_X$ and
$M_Y$ from the sequence alignment algorithm. \todo{pridaj linku}
\begin{figure}
\begin{center}
\includegraphics{../figures/simplePairHMM.pdf}
\end{center}
\caption[Example of pair hidden Markov model.]{
State $M$ generates aligned pairs of residues residues, $I_X$ ($I_Y$) generates
symbol only in the first (second) sequence.
}\label{FIGURE:EXAMPLEPAIRHMM}
\end{figure}
\end{example}

\begin{definition}
Let $H=(\Sigma,V,I,d,e,a)$ be  a pHMM, $X$ and $Y$ be arbitrary sequences over
$\Sigma$ and $\pi$ be a state path. The probability that sequences $X$ and $Y$
were generated by model $H$  with state path $\pi$ is
\begin{equation}
\Pr\left(X,Y,\pi\mid H\right)=
I_{\pi_0}
\left(
	\prod_{i=1}^{|\pi|}a_{\pi_{i-1},\pi_i}
\right)
\left(
	\prod_{i=0}^{|\pi|}e_{\pi_i,(X[D^x_{i-1}:D^x_{i}],Y[D^y_{i-1}:D^y_{i}])}
\right)
\end{equation}
If $D^x_{|\pi|-1}\not=|X|$ or $D^y_{|\pi|-1}\not=|Y|$ then
$\Pr\left(X,Y,\pi\mid H\right)=0$. 
\end{definition}

Similarly as for HMM we can define probability that sequences $X$ and $Y$ were
generated by the model $H$.

\begin{definition}
Let $H=(\Sigma,V,I,e,a)$ be a pHMM and  $X$ and $Y$ be arbitrary sequences over
$\Sigma$. Then probability that sequences $X$ and $Y$ were generated together by
model $H$ is 
\begin{equation}
\Pr\left(X,Y\mid H\right)=\sum_{\pi}\Pr\left(X,Y,\pi\mid H\right)
\end{equation}
\end{definition}


\subsection{Viterbi algorithm for pair HMM}

Algorithms operating over  pHMM are similar to those fo rregular HMMs, but in
general they have higher computational complexity because over model states with
sequence alignment.
In this section, we describe two-dimensional version of the Viterbi algorithm,
other algorithms are analogous.

The Viterbi algorithm for HMMs computes variables $V[i,v]$ and $B[i,v]$. Every
variable is parametrized by a position in the sequence and a state. For
two-dimensional version, we will add position in the second sequence.

Let $V[i,j,v]$ be the probability of the most probable state path that generated
$X[:i+1]$ and $Y[:j+1]$ and ended in state $v$. Clearly, $\max_{v\in
V}V[|X|-1,|Y|-1,v]$ is the probability of the most probable state path that
generated $X$ and $Y$. Let $B[i,j,v]$ be the last but one state of the most
probable state path that generated $X[:i+1]$ and $Y[:j+1]$ and ended in state
$v$. To make it easier we expect that all states but one are not silent -- the emit
symbol in at least one sequence. The one silent state $s$ will have $I_s=1$.
The two-dimensional Viterbi algorithm given bellow can be altered to work with
silent states by method given in section \ref{KDE}. Let $n=|X|$ and $m=|Y|$.

\todo{skontroluj ci sedia indexy a tak}
\begin{align}
V[-1,-1,s] &= 1\\
V[-2,i,v] &= V[j,-2,v] = 0, \forall v\in V,-1 \leq i < n, -1\leq j < m\\
V[i,j,v] &= \max_{u\in
V}V[i-d^x_{v},j-d^y_v,u]a_{u,v}e_{v,(X[i:i+d^x_v],Y[j:j+d^y_v])}\label{EQUATION:2DVITERBIF}\\
%V[-1,j,v] &= V[i,0,v] = 0 \\
%V[0,0,v] &= I_{v}e_{v,(?,?)} \\
B[i,j,v] &= \arg\max_{u\in
V}V[i-d^x_{v},j-d^y_v,u]a_{u,v}e_{v,(X[i:i+d^x_v],Y[j:j+d^y_v])}\label{EQUATION:2DVITERBIB}
\end{align}

In equations \ref{EQUATION:2DVITERBIF} and \ref{EQUATION:2DVITERBIB} boundaries for $i$ and $j$ are $
-1\leq i< n,-1\leq j< m$ and $i$ or $j$ is greater than $-1$.


Finding the last state $v$ of the most probable state path and back-tracing from
$B[n-1,m-1,v]$ we can reconstruct the most probable state path. Time
complexity of this algorithm is $O(nm|V|^2)$ (or $O(nm(|V|+t)$ where $t$
is the number of transitions of $H$) and memory requirements are
$O(nm|V|)$. However, we can use various tricks to decrease memory
requirements of such algorithms, as shown in section \ref{TODO}.


\subsection{Generalized Pair HMMs}


A \abbreviation{generalized pair HMM}{GpHMM} (or pair hidden semi-Markov
process) are combination of $pHMM$ and $GHMM$. Every state generates two
duration lengths $d^x$ and $d^y$ from some joint distribution associated with
the current state $d_v(d^x,d^y)$ and after that it generates two strings $x'$
and $y'$ with lengths $d^x$ and $d^y$ according the joint distribution
$e_{v,(x',y')}$. This probability distribution can by defined for example by
pair hidden Markov model.  As with GHMMs and unlike pHMMs, the state path is not
sufficient to determine which parts of the sequences were generated by which
state, we also need two sequences of durations.

Drawback of GpHMM is increased computational complexity. For example time
complexity of the Viterbi algorithm is $O()$\cite{}. \todo{POZRI CLANOK KDE TO
JE} GpHMM were successfully used for gene-finding
\cite{Meyer2002,SLAM2003,Alexanderson2004,Majoros2005}. More detail are given in
chapter \ref{CHAPTER:PAIRHMM}.

%\begin{definition}
%GpHMM is tuple $H=(\Sigma,V,I,e,a)$ where $\Sigma$ is finite alphabet, $V$ is 
%\end{definition}




\section{Other Decoding Methods}
\todo{MPA moze mat podobne problemy ako Viterbi. Ake?}
In the previous sections we have described two decoding algorithm: Viterbi algorithm
that finds the most probable state path  and the Posterior decoding that for
every symbol of the sequence assigns the state that generated such symbol with
maximum probability. 
%In following section we show that the Posterior decoding is
%equivalent to maximizing the expected number of correctly predicted states in
%state path.  
We have shown that in some cases these algorithm recover bad
annotation. However, maximizing the most probable annotation is NP-hard and
therefore it is not tractable.

In this section we will describe highest expected gain decoding framework,
describe already discussed algorithm within this framework and show several
other decoding methods developed in recent years.

\subsection{Highest Expected Gain}

\label{SECTION:HEG}

In this section we will describe a framework for studying decoding algorithm in
a more systematic way. This framework was introduced introduced originally for
conditional random fields \cite{Gross2007}.  To use this framework, we need to
define a gain function, which will express similarity (gain) between two
annotations or state paths. The higher the gain, the more similar those two
sequences are. Gain function is domain specific and can penalize differences
between state paths that are most important in a particular application.  Gain
function is not a similarity in the mathematical sense; it does not even have to
be symmetric.

Our goal is to find an annotation that is as similar as possible to the correct
annotation. Problem is that we do not know the correct annotation. Our only
assumption is that data came from the model and therefore we will treat the
correct annotation as a random variable, with probability distribution defined
by the HMM and the observed sequence. We will seek for annotation that maximizes
the highest expected gain.

\begin{definition}
Let $H$ be an HMM and $L$ be the set of all annotations. Any function
$f:L\times L\to \mathbb{R}$ is a \firstUseOf{annotation gain function}.

Let $\Pi$ be the set of all state paths. Any function $f:\Pi\times
\Pi\to\mathbb{R}$ is a \firstUseOf{path gain functions}.

\end{definition}

\begin{note}
We will use term gain function instead of annotation/path gain function if it is
clear from the context.

Machine learning literature often uses a related term of loss function
\cite{}. Lower loss mean more similar annotations, that is higher gain. Instead
of maximizing expected gain we can therefore equivalently minimize expected
loss.
\end{note}

\begin{definition}
Let $H$ be an HMM, $f$ be a gain function, $X$ be a sequence generated by $H$ and
$\Lambda$ be an annotation of $X$. Then the \firstUseOf{expected gain} of annotation
$\Lambda$ is 
\begin{equation}
E_{\Lambda_X\mid X,H}[f(\Lambda_X,\Lambda)] =
\sum_{\Lambda_X}f(\Lambda_X,\Lambda)\Pr\left(\Lambda_X\mid X,H\right)
\end{equation}
Let $\pi$ be a state path. Then the expected gain of state path $\pi$ is 
\begin{equation}
E_{\pi_X\mid X,H}[f(\pi_X,\pi)] =
\sum_{\pi_X}f(\pi_X,\pi)\Pr\left(\pi_X\mid X,H\right)
\end{equation}

\end{definition}


Once we have HMM $H$, gain function $f$ and observed sequence $X$,
we are trying to find the annotation/state path maximizing the expected gain. 
\begin{equation}
\Lambda = \arg\max_{\Lambda}E_{\Lambda_X\mid
X,H}\left[f\left(\Lambda_x,\Lambda\right)\right]
\end{equation}

We can express the classical decoding algorithms within this framework to show
its universality. We will define two gain functions: $f_A$ which corresponds to
the  
Viterbi algorithm and the most probable annotation problem and $f_p$ which
corresponds to 
posterior decoding.

The gain function $f_A$ is simply identity function (definition for state paths
is analogous).
\begin{equation}
f_A(\Lambda,\Lambda') = \begin{cases}
1 & \text{if $\Lambda = \Lambda'$ }\\
0 & \text{if $\Lambda \not=\Lambda'$}
\end{cases}
\end{equation}
For this gain function $E_{\Lambda_X}[f_A(\Lambda_X,\Lambda)]=\prob{\Lambda\mid
X H}$ and this
maximizing expected gain is 
equivalent to the most probable annotation problem. Similarly if we define gain
as the identity function over state paths, we obtain the most probable state
path problem, which can by solved by the Viterbi algorithm.

Therefore for given sequence finding annotation with highest expected gain is
NP-hard if gain function is part of the input. However, for specific gain
functions we can maximize expected gain in polynomial time.

The gain function $f_P$ compares the two annotations of the same length position
by position and assigns score $1$ to every position where they are equal.
\begin{equation}
f_P(\Lambda,\Lambda') = 
\begin{cases}
0 & \text{if $|\Lambda|\not=|\Lambda'|$}\\
\sum_{i=0}^{|\Lambda|-1}\begin{cases}
1 & \text{if $\Lambda_i=\Lambda'_i$}\\
0 & \text{otherwise}
\end{cases}
\end{cases}
\end{equation}

Similarly, we can define gain function $f_P$ for state paths. In such case
maximizing expected gain is equivalent to posterior decoding. Highest expected
gain framework give us another interpretation of posterior decoding. Let
$\Lambda_X$  have same length as $\Lambda$. From linearity of the expectation we
have
\[E_{\Lambda_X}[f_P(\Lambda_X,\Lambda)] =
\sum_{i=0}^{|\Lambda|-1}E_{\Lambda_X[i]}[f_P(\Lambda_X[i],\Lambda[i])]\]
We say that $i$th label of $\Lambda$ is correctly predicted if $f_P(\Lambda_X[i],\Lambda[i])=1$. Therefore  
by maximizing $f_P$ we search for an annotation/state path that maximizes the
expected number of correctly predicted labels/states.

\subsection{Maximum Boundary Accuracy Decoding}

\abbreviation{Maximum boundary accuracy decoding}{MBAD} is used in gene-finder
CONTRAST \cite{Gross2007}. It
was proposed for \abbreviation{conditional random fields}{CRF}, but since CRF
are similar to HMM, we will define them in terms of HMMs.

This decoding method tries to maximize weighted difference between the expected
number of true-positive and false-positive coding region boundaries.

\begin{definition}
Let $\Lambda=\Lambda_0\Lambda_1\dots\Lambda_{n-1}$ be an annotation. A boundary of
annotation $\Lambda$ is every position $i$ where $\Lambda_{i-1}\not=\Lambda_i$. 
\end{definition}

Maximum boundary accuracy decoding has one parameter $\gamma$. Let $B_{\Lambda'}$ be the
set of all boundaries in $\Lambda'$. Then MBAD maximizes following function:
\begin{equation}
f(\Lambda,\Lambda')=\sum_{i\in B_{\Lambda'}}g(\Lambda,\Lambda',i)
\end{equation}
where 
\begin{equation}
g(\Lambda,\Lambda',i)=
\begin{cases}
1 & \text{if $\Lambda_{i-1}=\Lambda'_{i-1}$ and $\Lambda_{i}=\Lambda'_{i}$}\\
-\gamma& \text{otherwise}
\end{cases}
\end{equation}

From linearity of the expectation we know that
\[E_{\Lambda}[f(\Lambda,\Lambda)']=\sum_{i\in
B_{\Lambda'}}E_{\Lambda}[g(\Lambda,\Lambda',i)]\] We call
$E_{\Lambda}[g(\Lambda,\Lambda',i)]$ the expected gain of boundary $i$.


Algorithm for finding annotation that maximized expected gain with function $f$
is following. At first we compute the $P^i_{c_1,c_2}$, the probability that correct annotation
has boundary between colors $c_1$ and $c_2$ is at position $i$. This can be computed by variant posterior decoding
since 
\[P_{c_1,c_2}^i=\prob{\Lambda_i=c_1,\Lambda_{i+1}=c_2\mid X,H}\]  
This
can be computed by variant of Forward-backward algorithm.  The expected gain of
boundary at position $i$ between colors $c_1,c_2$  is 
\[P^i_{c_1,c_2}-\gamma (1-P^i_{c_1,c_2})\]
We denote this value as  $B^i_{c_1,c_2}$.

After computing values $B^i_{c_1,c_2}$ we construct graph $G=(V,E)$ where
\begin{align*}
V&=\{v^i_{c_1,c_2}\mid 0\leq i<|X|,c_1,c_2\in C\}\cup\{s\}\\
E&=\{(v^i_{c_1,c_2},v^j_{c_2,c_3})\mid 0\leq i<j< |X|, c_1,c_2,c_3\in C
\}\cup\{(s,v^i_{c_1,c_2}\mid 0\leq i< |X|, c_1,c_2\in C\} 
\end{align*}
where $C$ is the set
of labels. Weight of edge $(u,v^i_{c_1,c_2})$ is $B^i_{c_1,c_2}$ for all $u\in
V$. There is one to one correspondence between annotations of $X$ and paths in
$G$ starting in $s$. Moreover weight of every path is expected gain of
corresponding annotation. $G$ is acyclic and therefore we can find such path in
polynomial time. This can be implemented in $O(|X||C|^2+|X|m^2)$ time and memory
where $m$ is the number of states of HMM $H$. More detail can be found in
\cite{Nanasi2010mgr}.

Intuition behind gain function $f$ is following. Like with the Posterior
decoding, we want to maximize the number of correctly predicted boundaries (the
Posterior decoding maximizes the number of correctly predicted state). The
difference is in the $\gamma$. If $\gamma=0$ then almost every possible boundary
have positive gain and therefore the reconstructed annotation will contain many
false-positive boundaries with very small expected gain. Positive $\gamma$ cause
that boundaries with small posterior probability will have negative expected
gain and therefore it is less likely that they appear in optimal annotation.


\subsection{Highest Expected Reward Decoding}

\abbreviation{Highest Expected Reward Decoding}{HERD} is extension of maximum
boundary accuracy decoding. We have developed this decoding for prediction of
recombination of HIV virus.  Further details can be found in \cite{Nanasi2010}
and in my Master thesis \cite{Nanasi2010mgr}.


Genome of some viruses (for example HIV or HCV virus) can be divided into
several subtypes. Moreover, it is possible that virus is mozaic
recombination of viruses from different subtypes (we call this virus
recombinant). In the problem of recombination detection we try to decide if
given sequence $X$ is recombinant sequence. If $X$ is recombinant then we want to find
original subtypes of every part of a sequence $X$. Recombinations can be modeled
by jumping HMMs \cite{Schultz2006} which are HMM with topology specific to this domain.
In this application is 
%In the problem of recombination detection in virus genome In the problem of
%recombination detection in virus RNA is usually jumping HMM \cite{}. 
hard to find exact recombination point and annotations with slightly shifted
boundaries are very similar. Therefore we have defined the following gain
function.

\begin{figure}
\begin{center}
\includegraphics[width=10cm]{../figures/HERDbuddy.pdf}
\end{center}
\caption[Highest Expected Reward Decoding explanation]{
Triangles corresponds to boundaries. Vertical lines corresponds to the windows
of size $2W$ or smaller to avoid overlaping. Windows around boundaries
represents regions where we search for corresponding boundaries in the correct
annotation.
The first and the fourth boundary are correctly predicted because in the correct
annotation is boundary between same colors within distance $W=2$.
%correctly predicted (they have penalty $-\gamma$). Because there is no such
%boundary in the correct
%annotation within window.
This picture was taken from \cite{Nanasi2010mgr}.
}\label{FIGURE:HERDBUDDY}
\end{figure}

We say that boundary on position $i$ is correctly predicted, if it satisfy
following conditions.
\begin{enumerate}
\item There is boundary between same labels in the correct annotation on position $j$ and $j$ is
within distance $W$ from $i$ ($|i-j|\leq W$).
\item There is no other boundary between $i$ and $j$. 
\end{enumerate}
Boundaries are illustrated in figure \ref{FIGURE:HERDBUDDY}. 
%In addition, the
%beginning and the end of the sequence are considered to be boundary too (with
%special start and end label).

Let $x$ be the number of correctly
predicted boundaries and $y$ be the number of other boundaries in the proposed
annotation. Then $f(\Lambda,\Lambda')=x-\gamma y$ where $\gamma$ is defined
constant. If $W=1$ then this is equivalent to Maximum Boundary Accuracy
Decoding. 

Optimizing this criteria can be done in $O(|X|W|C||T| + n|C|^2W^2)$ time  and
$O(\sqrt{|X|}|C||V|+W|C||V|+n|C|^2)$ memory. Experimental evaluation,
optimization algorithm and implementation details can be found in
\cite{Nanasi2010mgr}.

\subsection{Distance Measures on Annotations}

Another approach to solve similar problem was proposed by {\it Brown,
Truszkowski} in \cite{Brown2010}. Originally their implementation was aimed at
prediction of boundaries in transmebrane proteins \cite{Brown2010}, but later
they successfully adapted their algorithm to jumping HMMs \cite{Truszkowski2011}.
Their approach is trying to solve same problem as HERD: the exact boundaries of
an annotations are hard to find and grouping similar annotation is useful.

\begin{definition}
Let $d$ be any distance measure defined on annotations. Ball of radius $r$
around annotation $\Lambda$ is 
\begin{equation*}
B_d(\Lambda,r) = \{\Lambda'\mid d(\Lambda,\Lambda')\leq r\}
\end{equation*}
\end{definition}

\begin{definition}
Let $\Lambda=\Lambda_0\Lambda_1\dots\Lambda_{n-1}$ be an annotations. Footprint
of $\Lambda$ is maximal subsequence of $\Lambda$ that does not contain two
identical 
consecutive labels.
\end{definition}

\begin{definition}
Let $b_i(\Lambda)$ be $i$-th boundary of $\Lambda$ and $b(\Lambda)$ be the
number of boundaries in $\Lambda$.
Border shift distance $d_{b}$ is 
\begin{equation*}
d_{b}(\Lambda,\Lambda') = \begin{cases}
\infty & \text{if $\Lambda$ and $\Lambda'$ have different footprint}\\
\max_{i=0}^{b(\Lambda)-1} d_i(\Lambda)-d_i(\Lambda') & \text{otherwise}
\end{cases}
\end{equation*}
and border shift sum distance $d_s$ is 
\begin{equation*}
d_{s}(\Lambda,\Lambda') = \begin{cases}
\infty & \text{if $\Lambda$ and $\Lambda'$ have different footprint}\\
\sum_{i=0}^{b(\Lambda)-1} d_i(\Lambda)-d_i(\Lambda') & \text{otherwise}
\end{cases}
\end{equation*}

\end{definition}

{\it Brown,
Truszkowski} maximize following function
\begin{equation*}
f_d(\Lambda,\Lambda') = 
\begin{cases}
1 & {\text if }\Lambda\in B_d(\Lambda',r)\\
0 & \text{otherwise}
\end{cases}
\end{equation*}

As distance $d$, they have considered Hamming distance, Border shift distance
and Border shift sum distance \cite{Brown2010}.  If we set $r=0$ then $f_d$ is
same as $f_A$ and therefore finding annotation that maximize $f_d$ is NP-hard.

Maximizing $f_d$ is NP-hard but finding annotation with footprint $F$ and
maximal expected gain can be done in polynomial time \cite{Brown2010}. Therefore
{\it Brown and Truszkowski} used following heuristic algorithm: At first they
sample the state path to get footprint with hight probability

Gain function $f_d$ is similar to $f_A$: the annotation is considered correct if
it is same/very similar to th correct one. However, MBAD and HERD and the
Posterior decoding do not care about overall structure of annotation, they
construct annotation from highly probable ``features''.




%\subsection{Hybridizing Viterbi and Posterior Decoding}

%Toto tu asi nebude. Rad by som to tu mal, ale nie je cas a vobec to nie je
%dolezite. Mozno do dizertacky to napisem.

%There are approaches to combine Viterbi and Posterior decoding. It is
%interesting that 

\bigskip
{\large\bf Podtialto su zapracovane pripomienky}
\bigskip


\section{Algorithmic Improvements}
\label{SECTION:ALGORITHMICIMPROVEMENTS}

In this section we review implementation details and several algorithmic
improvements of the Viterbi algorithm and the Posterior decoding for HMMs and
pHMMs. Mostly we will assume that HMMs does not have silent states most of these
techniques are easily adjustable to HMMs with silent states. 

\subsection{Basic Implementations}

Implementation of the Viterbi algorithm and Forward-Backward algorithm can be
done by two-dimensional dynamic programming, similarly as with the sequence
alignment. Let $H$ be an HMM, $V[i,v]$ be matrix $n\times m$ where $n$ is the
length of the sequence and $m$ is the number of states of HMM. $V$ will be the
dynamic programming matrix for the Viterbi algorithm or the Forward algorithm.

Code for the Viterbi algorithm is following:
\begin{lstlisting}
Initialize D[0,*]
for i = 1...n-1:
  for v = 0...m-1:
    V[i,v] = max    [u=0...m-1] V[i-1,u]*[v,X[i]]*a[u,v]
    B[i,v] = argmax [u=0...m-1] V[i-1,u]*[v,X[i]]*a[u,v]
statePath[n-1] = argmax [u=0...m-1] V[n-1,u]
for i=n-1...1:
    statePath[i-1] = B[i,statePath[i]]
\end{lstlisting}

This algorithm runs in $O(nm^2)$ and $O(nm)$ memory. Note that values of row
$V[i,\cdot]$ and $B[i,\cdot]$ are computed from rows $V[i-1,\cdot]$ so if we
want just the probability of the most probable state path, we need to remember
only last two rows of $V$ (and we do not need $B$) so the memory requirements
will be $O(n+m)$. If we replace on line $4$ maximum with summation and remove
lines $5-8$, we will
obtain the Forward algorithm. Lines $6-8$ implements back-tracing procedure.

Note that if $a[u,v]$ is zero, then value on the right side of lines $4$ nd $5$
is zero. Therefore we have to iterate only for such $u$, that $u\to v$ is
transition in $H$. By this we will reduce time complexity to $O(n(m+t))$ where
$t$ is the number of transitions of $H$ not only for the Viterbi algorithm, but
also for the Forward algorithm and the Posterior decoding.


\subsection{Chceckpointing}
\label{SECTION:HMMCHECKPOINTING}
Chceckpointing can be used to decrease memory complexity of the Viterbi
algorithm with backtracing procedure and the Posterior decoding to $O(\sqrt n
m)$.  Similarly, this trick can be used to reduce memory complexity of algorithm
for pHMMs to $O(nm\sqrt n )$ where $n$ is the length of the longer sequences and
$m$ is the number of states of HMM.

\subsubsection{The Viterbi Algotrithm}
While back-tracing, we need access to values of matrix $B$ in decreasing order.
We need only matrix $B$ and last row of $F$. If we have $i$-th row of $F$ we can
recompute submatrix $B[i+1:,\cdot]$. Therefore we will remember every $k$-th row
of $V$ from which we will recompute blocks of matrix $B$. $B$ will be split into
blocks
$B_0=B[1:k+1,\cdot],B_1=[k+1:2k+2,\cdot],\dots,B_i=B[ik+1:{i+1}k+1,\cdot],\dots$
We will keep in memory exactly one block of $B$. If we need row that is in block
$B_i$ but $B_i$ is not in memory, we discard current block and compute block
$B_i$ from $V[ik,\cdot]$. Since we need rows of $B$ in decreasing order, we
recompute every block exactly once. If $k=\theta(\sqrt n)$ then we need
$O(n+m\sqrt n)$ memory to find most probable state path.

For two dimensional version we keep every $k$-th matrix $V[ik,\cdot,\cdot]$ and
algorithm is analogous to two-dimensional version.

\subsubsection{The Posterior Decoding}

In the Posterior decoding need to interlace computations of $F[i,v]$ and
$B[i,v]$ to compute $F[i,v]\cdot B[i,v]$ for all $0\leq i<n,v\in V$. We will
compute $F$ with the chceckpointing algorithm (we compute every $k$-th row of
$F$ and keep in memory only one block).  $B[i,v]$ will be computed row by row
with the $O(m)$ version of Backward algorithm. When we compute $B[i,v]$ we
compute $F[i,v]$ by checkpointing algorithm (if $i$ is in current block then
return $F[i,v]$ from current block. Otherwise discard current block and
recompute block in which $i$ belongs). With this approach we can compute
posterior values in $O(m\sqrt n)$ memory. We recompute every row at most once
and therefore time complexity will not change.


%Checkpointing approach for posterior decoding is more similar to the
%sequence-alignment approach. We want to compute $F[i,v]\times B[i,v]$ for all
%$0\leq i< n$ and $v\in V$. We compute every $k$-th row of $F$ by forward
%algorithm that remembers only last two rows. After that we interlace computation
%of $B[i,v]$ for $i$ from $n-1$ to $0$ with $F[i,v]$ which we compute with
%remembered block and checkpoints. For computation of $B$ we need only $O(m)$
%memory since we have to remember only two rows. For computation of $F$ we need
%$O(\sqrt{n}m)$ memory of $k=\theta(\sqrt n)$. We will recompute every block at
%most once and therefore time complexity is still $O(n(m+t))$ and memory
%complexity is $O(n+m\sqrt n)$.
%
%\subsection{The Hirschberg Algorithm}

%It is possible to use Hirschberg algorithm to r

%We can use the Hirschberg trick to reduce memory complexity of the Viterbi
%algorithm for pair hidden Markov models. We hav 


%Similarly as with the sequence alignment we can decrease the memory requirement
%of the Viterbi algorithm for pHMM to $O(nm)$ where $n$ is the length of the  


\subsection{Using Sequence Repetition}

This techniques uses dictionary based encoding schemes to speedup calculation of
algorithms. We will show how to use this technique to Forward algorithm. Details
about how to use this technique to other algorithms can be found in
\cite{Lifshits2009}.


Idea of this speedup is that we reformulate Forward algorithm into sequence of
matrix multiplication.
Let $H=(\Sigma,V,I,e,a)$ be a HMM, $|V|=m$.  Let $M_x[u,v]=a_{u,v}e_{v,x},
u,v\in V,x\in \Sigma$ be $m\times m$ matrix and $I^x_u=I_ue_{u,x}$ be
$1\times m$ vector. Matrix multiplication

\begin{lemma}\label{LEMA:MATRIXMULTI}
Let $X=X_1\dots X_n$ be a sequence, $H$ a HMM and $M_x$ and $I^x$ defined as above and
$F[i,\cdot]$ be row vector from forward
Then
\begin{align}
F[0,\cdot] &= I^{X_0}\\
F[i,\cdot] &= I^{X_0}\prod_{j=1}^i M_{X_j}
\end{align}
for all $i< n$.
\end{lemma}

\begin{proof}
We prove this lemma by induction.  If $|X|=1$ then
$F[0,v]=I^[X_0]=I_{v}e_{v,X[0]}$ for all $v\in V$.  Assume that for
$F[k,\cdot] = I^{X_0}\prod_{j=1}^k M_{X_j}$ (we assume that product of zero matrices
is identity matrix). We have 
\begin{align*}
\left(I^{X_0}\prod_{j=1}^{k+1} M_{X_j}\right)[v] &= 
\left(F[k,\cdot] M_{X_{k+1}}\right)[v]\\ &= \sum_{u\in V} F[k,u]
M_{X_{k+1}}[u,v]\\ &= \sum_{u\in V} F[k,u] a_{u,v}e_{v,X_{k+1} } \\&= F[k+1,v] 
\end{align*}
for all $v$. 
\end{proof}

\begin{note}
In original paper \cite{Lifshits2009} they use little different definition of
HMM so they do not have problem with first $I$

\end{note}

Consequence is that we can write forward algorithm as the sequence of
multiplications of $|\Sigma|+1$ different matrices. Since matrix multiplication
is associative, we can use repetitive patters to speedup calculation. 
Let $x$ be subsequence of $X$ that has $k>1$ non-overlaping occurrences in $X$.
We can compute $M_{x_0}M_{x_1}\dots M_{x_{|x|-1}}$ once and use it several
times. Let $M_{x}=\prod_{i=0}^{|x|-1}M_{x[i]}$ for any string $x$.

By proper choosing of substrings $x$ we can speedup computation of Viterbi or
Forward-Backward algorithm. Algorithm has following structure:
\begin{enumerate}
\item We choose the \firstUseOf{dictionary} $D$. $D$ is set of string over alphabet $\Sigma$. String is
\firstUseOf{good} if it belongs to $D$.
\item We compute $M_x$ for all $x\in D$.
\item We split input sequence $X[1:]$ into sequence of good words
$x_0,x_1,\dots,x_{k-1}$ such that $x_0x_1\dots x_{k-1}=X[1:]$.
%and $x_i\in D$ for
%all $0\leq i<k$
\item We compute $I^{X_0}\prod_{i=0}^{k-1}M_{x_i}$ in $O(kf(m))$ time where
$f(m)$ is time needed to compute multiplication of two matrices of size $m\times m$
\end{enumerate}

Lifshits {\it et al.} \cite{Lifshits2009} showed several ways how to choose set
$D$ and split $X$ into sequence of good words. One is to use LZ78 factorization
to split $X$ into $O(n/\log n)$ good words. Computation of $M_x$ for all $x\in
D$ can be done in $O(|D|m^3)$ since $M_x,x\in D$ can be computed in $O(m^3)$
time with following algorithm: if $|x|=1$ then $M_x$ is already computed.
If $|x|>1$ then there is $x'\in D$ such that $x=x'a,a\in \Sigma$ and therefore $M_x =
M_{x'}M_{a}$. Since $|D|=O(n/log n)$, computing $M_x$ takes $\Omega(nm^3/log m)$ if
we use $O(m^3)$ algorithm for computing matrix multiplication. Computing fourth
step of algorithm takes also $\Omega(nm^3/\log n)$ time and therefore overall time complexity
is $\Omega(nm^3/\log n)$, which is $\Omega(\log n/m)$ speedup.

It is possible to use different encoding methods. Speedup for the run length encoding 
is $\Omega(r/\log r)$, for the straight-line programs $\Omega(r/m)$ and for the
byte pair encoding $\Omega(r)$ where $r$ is the compression ratio under each
compression scheme. We will not discuss details of this implementations, they
can be found in \cite{Lifshits2009}.

To adapt this approach to the Viterbi algorithm we have to make two adjustments. 
We have to use max-time matrix multiplication instead of matrix multiplication
\cite{Lifshits2009}. Max-time matrix multiplication is matrix multiplication
where summation is replaced with maximization:
\[M_1\odot M_2 [u,v] = \max_{0\leq i\leq m-1}M_1[u,i]M_2[i,v] \]
where $M_1$ and $M_2$ are matrices of size $m\times m$. This matrix
multiplication is also associative.

The second adaptation is that we have to had ability to reconstruct 
the most probable state path.i
After computing $I^{X_0}\prod_{i=0}^{k-1}M_{x_i}$ we do back-trace to obtain 
partial state path $\pi'_0,\pi'_1,\dots \pi'_{k}$ which is subsequence of the
most probable state path. Each $\pi'_i$ is state corresponding to the last
sequence of $x_i$. We have to reconstruct state path between $\pi'_i$s; for
every $x_i$ we have to reconstruct state path between $\pi'_{i-1}$ and $\pi'_i$.

All mentioned compression schemes has property,
that for every good string $x$ there are also two good strings $x^1,x^2$ such
that $x=x^1x^2$. To recover state path for every good string $x$ and $u,v\in V$
we have to pre-compute \[R_x[u,v] = \arg\max_{i\in V}M_{x^1}[u,i]M_{x^2}[i,v]\]

Let $x=x_i$. Then $R_{x}[\pi'_{i-1},\pi'_{i}]$ corresponds to the state in the
position of the last symbol of $x^1$. By recursive applying this rule to $x^1$
and $x^2$ we can reconstruct the most probable state path on $x$. By doing this
we can reconstruct it the most probable state path in $O(n)$ time.

Lifshits {\it et al.} applied this  speedup to the Posterior decoding and the
Baum-Welsch training.

%Let $X$ be the most probable state path that uses 
%To adapt
%this approach to the Viterbi algorithm we have to use in matrix multiplication
%maximum instead of summation (such multiplication is also associative)
%\cite{Lifshits2009}.

\subsection{On-line Viterbi Algotithm}
\label{SECTION:ONLINEVITERBI}

On-line Viterbi algorithm is different approach to reduce memory complexity of
Viterbi algorithm. In this approach we will discard parts of matrix $B$ that 
are not necessary for reconstruction of the most probable state path.
We can represent $B$ as a forest $G=(V',E)$ where
\begin{align*}
V' &= \left\{ (i,v)\mid 0\leq i< n, v\in V  \right\}\\
E &= \left\{ (i,v)\to (i-1,B[i,v])\mid 0<i<n,v\in V\right\}
\end{align*}
Edges in this forest are oriented from the children to parents. Roots of the
trees are vertices $(0,v),v\in V$.  Most probable state path that generated
$X[:i]$ and ending in vertex $v$ corresponds to the path from vertex $(i-1,v)$
to root of it's tree. Let $S_{i,v}$ be the set of vertices path from vertex
$(i,v)$ to the root of it's tree and $S_i=\Cup_{v\in V}S_{i,v}$.
After computation of $B[i,v]$ only vertices in $S_i$ can lie on the most
probable state path so other can be removed. State paths corresponding to
$S_{i,v}$ may be overlap, so by removing others we can reduce memory
footprint.

{\v S}r{\'a}mek {\it et al.} developed data structure that maintains $S_i$
without introducing significant overhead.  Let $G_i$ be induced subgraph of $G$
consisting from vertex set $S_i$. Leaf $(j,v)$ is \firstUseOf{unreachable} if
$j<i$.  $G_i$ does not contain unreachable leaf and therefore $G_i$ contains $m$
leaves $(i,v),v\in V$. Construction of $G_{i+1}$ is following.
After computing $B[i+1,\cdot]$ we add vertices $(i+1,v),v\in V$ and edges 
$(i+1,v)\to (i,B[i+1,v]),v\in V$. Vertices $(i+1,v),v\in V$ are new leaves.
While in $G_{i+1}$ are some internal leaves, we remove them.
After removing all internal leaves, $G_{i+1}$ contains only vertices from $S_i$.
To make $G_i$ smaller, we use compress representation of trees. We compress
chains of vertices with one children into one edge (but we keep
corresponding state path of compressed chain). If we do so, every vertex will
have at most two children and therefore there will be at most $m-1$ internal
nodes. $G_i$ has therefore always $O(m)$ leaves and updating can be done in
$O(m)$ time so this technique does not alter asymptotic running time of the
Viterbi algorithm. In practice this technique increase running time by $5$\%
\cite{Sramek2007}.

Improvement in memory requirements can be significant. Even when there exists
HMMs on which this technique will not improve memory requirements
\cite{Sramek2007}, {\v S}r{\'a}mek {\it et al.} reports polylogarithmic 
memory requirements for HMM used for gene finding.
They also proved that for symmetric two
state HMMs (with one exception), expected memory complexity of this algorithm is
$O(H\log n)$ where $H$ is constant specific for HMM. Estimation of expected time
complexity for given HMM is still open problem.

TODO spomienka o treeterbi
