\chapter{Hidden Markov Models}

\abbreviation{Hidden Markov Models}{HMM} are graphical probabilistic model
commonly used to sequence annotation or sequence annotation. HMM is
probabilistic finite state machine that in every state emit one symbol. There
are variants of HMMs that emits more symbols or which emits symbols on multiple
tracks, we will discuss them later. In this chapter we will describe basic
definitions and algorithms that are used with HMMs.

\section{Definitions}\label{SECTION:HMMDEF}
                       
%HMM
%Pravdepodobnost
%Posterior pravdepodobnost
%Anotacia
%Pravdepodobnost anotacie
%Footprint
HMMs are generative probabilistic models (formal definition is given below). Initially HMM is in random state $q$
according to \firstUseOf{initial distribution} $I$.  When HMM is in arbitrary
state $q$ it emits one symbol from some alphabet $\Sigma$ according distribution
$e_q$ and move to another state according transition distribution $a_q$. Note
that emission and transition distribution can be different for every state.
This process produce two sequences: sequence of states
$\pi=\pi_0\pi_1\pi_2\dots$ called \firstUseOf{state path} and output sequence
$X=X_0X_1X_2\dots$ over alphabet $\Sigma$. In this work we will use only
discrete versions of HMMs and state space and alphabet will be always finite.
HMMs can have several final states. If final states are present, once HMM reach
that state generation of a sequence stops. In this chapter we will ignore final
states. \todo{Budeme ich vobec niekde potrebovat?}

\begin{definition}\label{DEF:HMM}
A tuple $H=(\Sigma,V,I,e,a)$ is \abbreviation{Hidden Markov Model}{HMM} where
$\Sigma=\{\sigma_0,\dots\sigma_{m-1}\}$ is finite alphabet of size $m$,
$V=\{v_0,\dots,v_{k-1}\}$ is finite set of state of size $k$. $I$ is
distribution over $V$, $e$ is $k\times m$ matrix where each row contains
distribution over $\Sigma$ and $a$ is stochastic matrix of size $m\times m$.

We will write members of $e$ and $a$ as subscript, therefore $e_{u,v}$ is
element from $e$ in $u$-th row and $v$-th column.  

Therefore following conditions holds:
\begin{enumerate}
\item $\forall v\in V, I_v \geq 0$
\item $\sum_{v\in V}I_v=1$
\item $\forall u\in V,\forall \sigma\in\Sigma, e_{u,\sigma}\geq0$
\item $\forall u\in V, \sum_{\sigma\in \Sigma}e_{u,\sigma}=1$
\item $\forall u,v\in V, a_{u,v}\geq0$
\item $\forall u\in V, \sum_{v\in V}a_{u,v}=1$
\end{enumerate}
\end{definition}

The conditions $1-6$ just ensure, that everything is correct probabilistic
distribution.

\begin{example}
\end{example}

\begin{definition}\label{DEF:STATEPATH}
Let $H=(\Sigma,V,I,e,a)$ be an HMM.
We say that there is \firstUseOf{transition} from state $u$ to state $v$ if
$a_{u,v}>0$. We will transition from $u$ to $v$ as $u\to v$ and $T=\{u\to v\mid
a_{u,v}>0\}$ is the set of all transitions of $H$.

\firstUseOf{State path} $\pi=\pi_0\pi_1\dots\pi_{n-1}$ is the sequence of
states. We say that state path $\pi$ is \firstUseOf{admissible} if $I_{\pi_0}>0$
and for all $1\leq i < n$ $\pi_{i-1}\to\pi_i\in T$. Otherwise $\pi$ is
\firstUseOf{inadmissible}.
\end{definition}

\begin{example}
\end{example}

%Formally, Hidden Markov Model is defined by it's finite state space
%$Q=\{q_0,q_2,\dots, q_{k-1}\}$ of size $k$, initial probabilistic distribution
%$I$ over state space, finite alphabet $\Sigma = (\sigma_0, \sigma_1, \dots,
%\sigma_{m-1})$ of
%size $m$, emission and transition distribution over $\Sigma$ and $Q$
%respectively defined for every state independently. We denote emission
%distribution of state $q$ by $e_q$ and transition distribution by $a_{q}$. 

\begin{definition}
Let $H=(\Sigma,V,I,e,a)$ be a HMM and $X=X_0X_1\dots X_{n-1}$ be sequence over
alphabet $\Sigma$ of length $n$. Let $\pi$ be state path of length $n$. Then the
probability that state path generated sequence $X$ is 

\[\Pr\left(X,\pi\mid H\right) =
I_{\pi_0}e_{\pi_0,X_0}\prod_{i=1}^{|X|-1}e_{\pi_i}a_{\pi_{i-1},\pi_i}\]

If length of the sequence $X$ is not equal to length of the state path $\pi$,
then
probability that $\pi$ generates $X$ is zero.
\end{definition}

\begin{example}
Example of HMM
\end{example}

\begin{definition}
Let $H=(\Sigma,V,I,e,a)$ be a HMM and $X=X_0X_1\dots,X_{n-1}$ be sequence over
alphabet $\Sigma$ of length $n$. The probability that $X$ was generated by the
model $H$ is 
\[\Pr\left(X\mid H\right)=\sum_{\pi\in V^n}\Pr\left(X,\pi\mid H\right)\]
\end{definition}

%The probability of emission of sequence $X$ with state path $\pi$
%is \[\Pr\left(X,\pi\mid H\right) =
%I(\pi_0)e_{\pi_0}\left(X_0\right)\prod_{i=1}^{|X|-1}e_{\pi_i}a_{\pi_{i-1}}\left(\pi_i\right)\]
%Generally, there are many ways how to generate every sequence.  Therefore every
%HMM $H$ defines probability distribution \[\Pr\left(X\mid H\right) =
%\sum_{\pi,|\pi|=|X|}\Pr\left(X,\pi\mid H\right)\].  Since in every step HMM
%generates one symbol from $\Sigma$ and one state, state path has same length as
%generated sequence.

\begin{note}
In this definition of HMM, the sum the probabilities of all sequences of length
$n$ is $1$.
\end{note}

\todo{Tento text chcem dat asi trosku neskor, Az ked to budem potrebovat}
We will abuse the notation and by $e_q(\cdot)$ we mean row vector
$\left(e_q(b_1),\dots,e_q(b_{\sigma})\right)$. Similarly,
$a_q(\cdot)=\left(a_q(q_1),\dots,a_q(q_k)\right)$ is row vector that contain
transition probabilities from the state $q$.  $a_q(\cdot)$ have therefore
dimension $1\times k$. By $A$ we will denote the $|Q|\times |Q|$ matrix
consisting from vectors $a_{q_1},a_{q_2},\dots,a_{q_k}$.  Therefore $A[q_i,q_j]$
contains probability of transition from state $q_i$ to $q_j$.

\section{Forward Algorithm}
Forward algorithm computes for given sequence $X$ of length $n$ the quantity $\Pr\left(X\mid
H\right)$, the probability that sequence
was generated by the model. 
\begin{eqnarray}
F[0,v] &=& I_ve_{v,X_0}, v\in V\\
F[i,v] &=& \sum_{u\in V}F[i-1,u]*a_{u,v}*e_{v,X_i}, v\in V,0< i < n
\end{eqnarray}
We call values $V[i,v]$ forward variables. Value $V[i,v]$ is the
probability, that $X[:i+1]$ was generated by state path that ends with state
$v$. Therefore \[\Pr\left(X\mid H\right) = \sum_{v\in V} V[|X|-1,v]\]

Using recursive equations above, we can compute the probability of the $X$ in
$O(nm^2)$ time and $O(m)$ memory by dynamic programming. Forward variables are
also used in other algorithm. Storing all of the forward variables requires
soring $O(nm)$ numbers.

\section{Sequence Annotation}

%definujeme annotation, footprint, set of colors

One way of using HMMs is to use them to sequence annotation. By sequence
annotation we mean assigning ``labels'' to parts of the input sequence according
to their meaning. For example in gene-finding domain, we want to assign gene
labels to those parts of the sequence, that encodes proteins. In domain of
annotation of transmembrane proteins, we want to know, which parts of the
sequence are on which  side of the membrane. In recombination prediction we want
to predict which parts of the sequence belong to which subtype of organism.

In practice, we have finite set of labels $C=\{c_0,c_1,\dots,c_{l-1}\}$ and we
want to assign one label to every symbol of input sequence. We do it by
assigning one label to every state of HMM in a way, that states of HMM that
encodes same meaning have same color. After that we try to predict the
annotation of the sequence, for example by finding the most probable state path
that can generated models and we annotate each symbol of the sequence according
to the label assigned to state that generated that symbol. We can formalized
this in following way.

\todo{Aby nebol bordel medzi label a color}

\begin{definition}
Let $H=(\Sigma,V,I,e,a)$ be HMM and $C=\{c_0,c_1,\dots,c_{l-1}\}$ be the finite
sets of labels (or colors). Then \firstUseOf{coloring funcion} 
$\lambda: V^*\to C^*$ is function, that satisfies following properties:
\begin{enumerate}
\item $\lambda(v)\in C$ for all $v\in V$.
\item $\lambda(xy) = \lambda(x)\lambda(y)$ for all $x,y\in V^*$.
\end{enumerate}

Let $X$ be sequence generated by state path $\pi$. Then annotation
$\Lambda$ of sequence $X$ is $\Lambda = \lambda(\pi)$.
\end{definition}

One problem in sequence annotation is, that usually we do not know the state
path that generated sequence. Therefore we have to use methods, that somehow
reconstruct the state path or at least give good approximation of the correct
annotation. 

\begin{example}
Example of coloring function.
\end{example}

%Since many states of the HMM can have assigned same label, there may be may be
%many state path with same annotation.

\begin{definition}
Let $H$ be an $HMM$, $X$ be a sequence $\Lambda$ be a annotation of sequence
$X$. The probability of annotation $\Lambda$ given sequence $X$ if length $n$ is 
\[\Pr\left(\Lambda\mid X,H\right)=\sum_{\pi \in V^n,\lambda(\pi) =
\Lambda}\Pr\left(\pi\mid X,H \right)\]
\end{definition}

\begin{example}
Example of annotation and its probability.
\end{example}

\section{Viterbi Algorithm}
\todo{Viterbi variable ma rovnake oznacenie ako mnozina stavov. Treba to
upravit}
Viterbi algorithm is probably the most used decoding algorithm for Hidden Markov
models
Viterbi algorithm answer straightforward question: given the sequence
$X=X_0X_1\dots X_n$, what
is the most-likely state path $\pi$ that generates $X$? Formally, Viterbi
algorithm finds a state path maximizing $\Pr\left( \pi\mid X,H \right)$. Since
\[\Pr\left(\pi\mid X,H\right) = \frac{\Pr\left(\pi,X,\mid
H\right)}{\Pr\left(X\mid H\right)}\] and quantity $\Pr\left(X\mid H\right)$ is fixed
this is same as finding $\pi$ maximizing $\Pr\left(X,\pi\mid H\right)$. 

Viterbi algorithm is very similar to Forward algorithm. It starts with computing
Viterbi variables $V[i,v]$. $V[i,v]$ stores the probability of the most probable 
state path that generated $X[:i+1]$ and ends in state $v$. Viterbi algorithm
also computes back-links $B[i,v]$, which contains the previous state in the most
probable state path that generated $X[:i+1]$ and ends in state $v$. We can
compute these values by following equations:
\begin{eqnarray}
V[0,v] &=& I_{v}e_{v,X_0}, v\in V\\
V[i,v] &=& \max_{u\in V} V[i-1,u]a_{u,v}e_{v,X_i}, v\in V,0<i<n\\
B[i,v] &=& \arg\max_{u\in V} V[i-1,u]a_{u,v}e_{v,X_i}, v\in V,0<i<n
\end{eqnarray}
\begin{note}
We do not specify how $B[0,v],v\in V$ is defined since we will not
need it to be defined in our computation.
\end{note}
\begin{note}
If we compare Viterbi algorithm with the Forward algorithm, we see that
Viterbi equations are forward equations if we replace summation with
maximization.
\end{note}

Variable $V[n-1,v]$ contains the probability of the most probable state path
that generated $X$ and end in state $v$. Therefore the state $v_{\max} =
\arg\max_{v\in V}V[n-1,v]$ is the last state of the most probable state path.
Variable $B[n-1,v_{\max}]$ contains the previous state of the most probable
state path. By traversing back through back-links  we can reconstruct the most
probable state path.

\begin{example}
Example of Viterbi Table with back-links
\end{example}

We can use Viterbi algorithm to annotate sequence $X$ by finding the most
probable state path $\pi$ and then computing $\lambda(\pi)$. In coloring
function $\lambda$ is identity function, this will find the most probable
annotation, but in general, Viterbi algorithm not even good approximation of the
most probable annotation.

\begin{example}
Why Viterbi is not good enough
\end{example}

\section{Backward Algorithm}
Backward algorithm is backward version of the Froward algorithm. ???
\section{Forward-Backward Algorithm and Posterior Decoding}
\firstUseOf{Posterior Decoding} is another commonly used decoding method. In
contrast to Viterbi algorithm, posterior decoding assign labels individually to
every symbol of input sequence and does not care about overall structure of
reconstructed state path. 

Given sequence $X$, posterior decoding find state path $\pi$ with following
property:
\[\forall 0\leq i< n, \pi_i=\arg\max_{v\in V}\Pr\left(\pi_i=v\mid X,H\right) \]
where \[\Pr\left(\pi_i=v\mid X,H\right) = \sum_{\pi\in V^n,\pi_i=v}\Pr\left(\pi\mid X,H\right)\]

Values $\Pr\left(\pi_i=v\mid X,H\right)$ can be computed with Forward-Backward
algorithm. This algorithm computes $\Pr\left(\pi_i=v\mid X,H\right)$ for every
combination of position $i$ and state $v$. Then for every position chooses state
that maximizes posterior probability.

Now we derive formula for effective computation of  $\Pr\left(\pi_i=v\mid
X,H\right)$.
\begin{eqnarray}
\Pr\left(\pi_i=v,X\mid H\right) &=&  
%\sum_{\pi\in V^n,\pi_i=v}\Pr\left(\pi\mid X,H\right)
%					\label{PosteriorDer1} \\
%				&=& 
				\sum_{\pi\in
				V^n,\pi_i=v}I_{\pi_0}e_{\pi_0,X_0}\prod_{j=1}^{|X|-1}e_{\pi_j}a_{\pi_{j-1},\pi_j}
					\label{PosteriorDer2}\\
				&=& \sum_{\pi\in V^n,\pi_i=v}I_{\pi_0}e_{\pi_0,X_0}
				\left( 
					\prod_{j=1}^{i-1} e_{\pi_j}a_{\pi_{j-1},\pi_j}
				\right)
				a_{\pi_{i-1},\pi_i}e_{\pi_i,X_i}
				\left(  
					\prod_{j=i+1}^{|X|-1} e_{\pi_j}a_{\pi_{j-1},\pi_j}
				\right)
					\label{PosteriorDer3}\\
				&=& \sum_{\pi\in V^n,\pi_i=v}I_{\pi_0}e_{\pi_0,X_0}
				\left( 
					\prod_{j=1}^{i-1} e_{\pi_j}a_{\pi_{j-1},\pi_j}
				\right)
				a_{\pi_{i-1},v}e_{v,X_i}
				\left(  
					\prod_{j=i+1}^{|X|-1} e_{\pi_j}a_{\pi_{j-1},\pi_j}
				\right)
					\label{PosteriorDer4}\\
				&=& 
				\left(
					\sum_{\pi\in V^{i+1},\pi_i=v}I_{\pi_0}e_{\pi_0,X_0}
					\left(
						\prod_{j=1}^{i} e_{\pi_j}a_{\pi_{j-1},\pi_j}
					\right)
				\right)
				\left( 
					\sum_{\pi\in V^{|X|-i},\pi_0=v}
					\prod_{j=1}^{|X|-i-1} e_{\pi_j}a_{\pi_{j-1},\pi_j}
				\right)
					\label{PosteriorDer5}\\
				&=& F[i,v]
				\left( 
					\sum_{\pi\in V^{|X|-i},\pi_0=v}
					\prod_{j=1}^{|X|-i-1} e_{\pi_j}a_{\pi_{j-1},\pi_j}
				\right)
					\label{PosteriorDer6}
\end{eqnarray}
\todo{$e_{x}$ chyba jeden parameter.}
Equations \ref{PosteriorDer2} and \ref{PosteriorDer3} is from definition of
posterior probability. In \ref{PosteriorDer4} we just replace $\pi_i$ with $v$.
Next equality we have from the fact that the left part of the formula
\ref{PosteriorDer4} uses only left part of the state path and right part of the
formula uses only right part of the state path. Last equation follows from
definition of $F[i,v]$. Right part of the formula \ref{PosteriorDer6} has
similar structure than $F[i,v]$, but it uses last part of the state path and
sequence and also is lacks of using initial distribution. We can compute 
this formula using Backward algorithm, which is very similar to forward
algorithm. Let $B[i,v]$ be the right part of formula \ref{PosteriorDer6}.
\begin{eqnarray}
B[i,v]&=&
\sum_{\pi\in V^{|X|-i},\pi_0=v}
	\prod_{j=1}^{|X|-i-1}
		e_{\pi_j,X_{|X|+j-1}}a_{\pi_{j-1},\pi_j}\\
 &=& 
 \sum_{u\in V}
 	e_{u,X{|X|+j-1}}a_{v,u}
	\sum_{\pi\in V^{|X|-i},pi_0=v,\pi_1=u}
		\prod_{j=2}^{|X|-i-1}
			e_{\pi_j,X_{|X|+j-1}}a_{\pi_{j-1},\pi_j}\\
 &=& 
 \sum_{u\in V}
 	e_{u,X{|X|+j-1}}a_{v,u}
	\sum_{\pi\in V^{|X|-i-1},pi_0=u}
		\prod_{j=1}^{|X|-i-2}
			e_{\pi_j,X_{|X|+j-2}}a_{\pi_{j-1},\pi_j}\\
 &=& 
 \sum_{u\in V}
 	e_{\pi_j,X{|X|+j-1}}a_{\pi_0,\pi_1}B[i+1,u]
\end{eqnarray}
If we set $B[n,v]$ to $1$, we have recursive formula, to compute values
$B[i,v]$. This formula differs from Forward algorithm by going
backwards\footnote{Therefore it is called Backward algorithm.}.


Therefore \[\Pr\left(\pi_i=v\mid X,H\right) = \frac{F[i,v]\cdot
B[i,v]}{\Pr\left( X\mid H \right)}\]

We can compute this values in $O(nm^2)$ time and $O(nm)$ memory. By
check-pointing technique we can achieve $O(m^2\sqrt{n})$ memory requirements
with only constant slowdown.

As we can seem time complexity and memory requirement of posterior decoding is
same as for Viterbi algorithm, but in practice Forward-Backward algorithm is
slower and requires more memory.

Since posterior decoding does not care about overall structure of state path, it
can reconstruct inadmissible state path. One example is given below.
\begin{example}
Bad example
\end{example}

\section{Training} 

Training is process of estimating parameters of probabilistic models. In this
section we will describe several approaches how to estimate transition and
emission distributions. We will describe several approaches to do this thing

\subsection{Baum-Welsch algorithm}
\subsection{Viterbi Training}

Viterbi training is same as Baum-Welsch train

\subsection{Other Training Methods}
There are several other approaches how to similar
\section{Variants of Hidden Markov Models}

Hidden Markov models described in section \ref{SECTION:HMMDEF} are the Hidden
Markov models. In this section we describe several other variants of HMMs, some
of them with same expressing power and some not.

\subsection{Silent states}

One common variant of HMMs are HMMs with silent states. Silent state are states
that does not emit any symbol. One of the consequences is, that state path can
be longer then sequence. However, the number of non-silent states in state path
have to be equal to sequence length. Silent states does not add any expressing
power to HMMs, but in some cases they allow to reduce the number of transitions
by factor $m$.

\begin{definition}
Formally, Hidden Markov Model with silent states is tuple $H=(\Sigma,V,Q,I,e,a)$
where $\Sigma,V,I,a$ are defined as in definition \ref{DEF:HMM}. $Q\subseteq V$ is set of
silent states. $e$ must satisfy following conditions:
\begin{enumerate}
\item $\forall u\in V\backslash Q,\forall \sigma\in\Sigma, e_{u,\sigma}\geq0$
\item $\forall u\in Q,\forall \sigma\in\Sigma, e_{u,\sigma}=0$
\item $\forall u\in V\backslash Q, \sum_{\sigma\in \Sigma}e_{u,\sigma}=1$
\end{enumerate}
\end{definition}

\begin{definition}
Transitions and state path are defined as in definition \ref{DEF:STATEPATH}. 

Let $\pi$ be an state path. \firstUseOf{Non-silent state path} $\pi^Q$ is
maximal subsequence of $\pi$ that consists of non-silent states.
\end{definition}

\begin{definition}
Let $H=(\Sigma,V,Q,I,e,a)$ be a HMM and $X=X_0X_1\dots X_{n-1}$ be sequence over
alphabet $\Sigma$ of length $n$. Let $\pi$ be state path for which $\pi^Q$ have
length $n$. Then the probability that state path generated sequence $X$ is 

\[\Pr\left(X,\pi\mid H\right) =
I_{\pi_0}\left(\prod_{i=1}^{|\pi|-1}a_{\pi_{i-1},\pi_i}\right)\left(\prod_{i=0}^{|X|-1}e_{\pi^Q_i,X_i}\right)\]

If length of the sequence $X$ is not equal to length of the non-silent state
path $\pi^Q$, then
probability that $\pi$ generates $X$ is zero.
\end{definition}

\begin{example}
Example of HMM with silent states that reduces the number of edges by factor
$m$.

Consider following HMM: $H=(\Sigma,V,I,e,a)$ where
$|V|=m$, $I$ is arbitrary distribution over $V$, $e$ is also arbitrary and in
every row of $a$ is uniform distribution over $V$. In other words, there is
transition between any two pairs of states from $V$, that is exactly $n^2$
transitions\footnote{$u\to u$ is also transition}.

If remove all those transitions, add one silent state $s$ and add transitions
from state $s$ to all states from $V$ with probability $\frac1m$  and we add
transition from every state $V$ to $s$ with probability $1$. This new HMM
defines same distribution of sequences, but have one more state and only $2m$
transitions. 
\end{example}



\subsection{Start and Final state}

Sometimes it is useful to have special start state and special final state. We
will describe both states independently, since they affect models in different
ways. 

Having special start state $v$ is equivalent to $I_v=1$. 

HMM defined in section \ref{SECTION:HMMDEF} lacks conditions 
We can extend definition of HMMs to one or several final states.  


Final states affects distribution of model. HMM defined in section
\ref{SECTION:HMMDEF} define distribution over sequences of same length. HMM with
final states defines distribution over sequences of all length. 

%\subsection{High Order HMMs}
%
%If we look on the generated sequence as on the sequence of random variables
%$X_0,X_1,\dots, X_{n-1}$ and if we look on the state path also as on the
%sequence of random variables, $\pi_0,\pi_1\dots,\pi_{n-1}$, we observe that
%$X_i$ depends only on $\pi_i$ and $\pi_i$ depends only on $\pi_{i-1}$ (We say
%that random variables associated with state path have Markov
%property\cite{}). However, sometimes ability to look back more then just one
%state/symbol is useful.

\subsection{Generalized HMMs}

If we start observing then length of stay in one stay, we quickly end in 
with the conclusion that this is according geometric distribution.
The probability we will leave state $v$ after exactly $n$ steps is
$e_{v,v}^{n-1}(1-e_{v,v}$ which is geometric distribution. For some
applications in speech processing and ??? is this behaviour no appropriate.

\abbreviation{Generalized HMM}{GHMM} have with every state $v$ associated
duration distribution $d_v(\cdot)$.  When GHMM is in state $v$, it decides how
long string it will generate according $d_v$. Let it be $l$. Afterwards it
generates string $x\in\Sigma^l$ with probability $e_{v,x}$. Usually each symbol
of generated string is generated independently and then
$e_{v,x}=\prod_{i=0}^{|x|-1}e_{v,x}$. Output of GHMM is three sequences: state
path $\pi=\pi_0\pi_1\dots\pi_{l-1}$, \firstUseOf{duration sequence}
$D=D_0D_1\dots D_{l-1}$ and sequence $X=X_0X_1\dots X_{n-1}$.  State path and
duration sequence has to have same length and \[\sum_{i=0}^{|D|-1}D_i = |X|\].

Manipulation with GHMM is  more complicated and technical. If one state can
generate strings with various length then in general from state space we are
not able to uniquely assign to every state $v$ in $\pi$ the symbols from
sequence which were generated by $v$. To do so, we need also duration sequence.
Let $D^i = \sum_{i=0}^{i}D_i$ and $D^{-1}=0$.
The probability that $\pi$ with $D$ generated sequence $X$ is
\begin{equation}
\Pr(X,D,\pi\mid H) = 
\left(
\prod_{i=0}^{|\pi|-1}
d_{\pi_i}(D_i)e_{\pi_i,X[D^{i-1}:D_i]}
\right)
\prod_{i=1}^{|\pi|-1}
a_{\pi_{i-1},\pi_i}
\end{equation}
if $|\pi|=|D|$ and $|X|=\sum_{i=0}^{|D-1}D_i$. Otherwise $\Pr\left(X,D,\pi\mid
H\right)$ is zero. Similarly as with regular HMMs,  probability of the observed
sequence is
\begin{equation}
\Pr\left(X\mid H\right) = \sum_{\pi,D}\Pr\left(X,\pi,D\mid X\right)
\end{equation}
Computing this value can be done by variant of Forward algorithm. More details
in \cite{}. Maximizing likelihood of $\pi$ and $D$ can be found by variant of
Viterbi algorithm, also found in $\pi$.

%Generalized HMMs, or HMMs with explicit state duration density 

%Unlike hight order HMMs, emission and transition distribution depends only on
%current state. In \abbreviation{generalized hidden Markov model}{GHMM} every
%state can emit any finite string over alphabet $\Sigma$ according some
%distribution. This distribution can be, and usually is, different for every
%state. 

%Formally, each 
\section{Hidden Markov Models with Multiple Outputs}

\subsection{Pair Hidden Markov Models}

Unlike previous variant of HMMs, \abbreviation{pair hidden Markov models}{pHMM}
are not equivalent to standard HMMs defined in section \ref{SECTION:HMMDEF}.
pHMM generates pair of sequences. Every state generate none or one symbol in
both sequences, one symbol in only one sequence or zero symbols in both
sequences. Formally, every state generates pair of strings $(a,b)$, where $|a|$
and $|b$ are of length at most one. Moreover, once state generate some pair with
nonzero probability, it can generate only pairs with same length. Formal
definition of pair HMM is given below.

\begin{definition}
Pair hidden Markov model is tuple $H=(\Sigma,V,I,e,a)$, where $\Sigma$ is finite
alphabet, $V$ is final set of states and 
\end{definition}

\subsection{Generalized Pair HMMs}


\abbreviation{Generalized pair HMM}{GpHMM} are combination of $pHMM$ and $GHMM$.
Every state can emit 
like pHMM generates in every state
pair of string but GpHMM like GHMM do not have restriction on the length of the
generated strings. Only restriction is, that generated string has to be final.
Since GpHMM generalization of pHMM, we will define only GpHMM.

\begin{definition}
GpHMM is tuple $H=(\Sigma,V,I,e,a)$ where $\Sigma$ is finite alphabet, $V$ is 
\end{definition}

\section{Other Decoding Methods}

\subsection{Most Probable Annotation}

\subsection{Highest Expected Reward Decoding}

\subsection{Hybridizing Viterbi and Posterior Decoding}

\subsection{Distance Measures}

\subsection{One other}

\section{Algorithmic improvements}
%balls, herd, hybridizing viterbi












