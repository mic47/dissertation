\chapter{PhD project}
Poznamka pre seba -- toto chcem asi kratsie ako som to zacal pisat. Celkovo by
to mohlo mat tak par stran

\section{Pairwise sequence alignment}

\section{Hidden Markov Models}

\subsection{Measures for decoding algorithms}

%Predpoklady co som uz vysvetlil:
% HMM, Viterbi, Posterior, most probable annotation, balls, herd, -- gain
% functions

In section \ref{SECTION:HEG} we have discussed
the highest expected gain framework. This framework generalizes decoding methods
for HMMs. Recall that decoding method is the method of obtaining an annotation
of a sequence from the sequence and the model. At first we briefly review the
highest expected reward framework.

Hidden Markov models defined probability distribution over state paths $\pi$
given sequence $X$ and model $H$: $\prob{\pi\mid X,H}$. Given annotation
function $\lambda$, HMMs also defines probability distribution of annotations
$\Lambda=\lambda(\pi)$ given sequence $X$: $\prob{\Lambda\mid X,H}$.  The state
path (or an annotation) that generated a sequence $X$ is hidden.
We will denote by
the \firstUseOf{correct state path} of sequence $X$ the state path that
generated $X$ and the \firstUseOf{correct annotation} is annotation of the
correct state path. Our goal is to find the correct annotation/state path of
given sequence $X$ or at least the good approximation of it.

This is usually done by finding the most probable state path. However,
in case of HMMs that have multiple path problems, there might be many 
Event the most probable annotation does not have to give us useful approximation
of the correct annotation.

We have shown earlier that the most probable annotation can be different from 
the most probable state path, but searching for the most probable annotation is
NP-hard \cite{Lyngso2002,Brejova2007mpa}. In some cases the most probable
annotation does not have to be the thing we should look for.

\todo{Nesedi poradie argumentov, skontroluj to v celej minimovke!}
Since the correct annotation/state path is unknown, we treat correct annotation
as a random variable with distribution $\prob{\Lambda\mid X,H}$. We define gain
function $f:\bar{\Lambda}\times\bar{\Lambda}\to \mathbb{R}$.
$f(\Lambda_1,\Lambda_2)$ is the ``measure'' of an approximation of $\Lambda_2$
by $\Lambda_1$. This function is not mathematical measure, it can be arbitrary
function. Function $f$ is application specific, it should reflect the features
that are we interested in.

After defining function $f$, we seek the for the annotation $\Lambda$ with the highest expected gain:
\[\Lambda = \arg\max_{\Lambda} 
E_{\Lambda_X\mid X,H}[f(\Lambda_X,\Lambda)] =
\sum_{\Lambda_X}f(\Lambda_X,\Lambda)\Pr\left(\Lambda_X\mid X,H\right)
\]
We have shown that problem of finding the annotation with the highest expected
gain given sequence $X$ and annotation function $f$ is NP-hard, but some gain
functions can be decoded in polynomial time (for example gain functions for the
Viterbi algorithm or gain function for  the Posterior decoding).

There is a big gap between the time complexity of the Viterbi algorithm and the
NP-hard problems. While decoding some functions gan be 

We want to study gain functions (or measures of decoding from HMM) 

Highest expected gain can be also extended to pair hidden Markov models. In such
case we need to properly define annotations for the pHMM. In general, pHMM are
used to annotation and comparison of two sequences at once: In case of
comparative gene finding, pHMM finds genes in both sequences and alignment was
not main issue. In case of actual alignments, we were not so much interested in
the annotation of the sequences, but only in relation between the symbols of the
sequences. From state path (for pHMM) we can uniquely reconstruct alignment, so
we could define annotations as for the HMMs. However, it could create following
problem:

Let $H$ be simple pHMM from figure \ref{FIGURE:SIMPLEPHMM} (simple three state
pHMM for sequence alignment). Let $C=\{M,I\}$ be the set of labels and 
$\lambda$ be annotation function, such that $\lambda(M)=M, \lambda(I_X)=I$ and
$\lambda(I_Y)=I$. In such case it is not possible to assign annotation symbols
to input sequences $X$ and $Y$ and even it is not possible to assign unique
annotation. For example let $X=Y=ACGT$ and $\Lambda = IIMMII$. Then
following $6$ alignments are possible:
\begin{verbatim}
Annotation:    IIMMII    IIMMII    IIMMII    IIMMII    IIMMII    IIMMII
X:             --ACGT    ACGT--    A-CGT-    A-CG-T    -ACGT-    -ACG-T
Y:             ACGT--    --ACGT    -ACG-T    -ACGT-    A-CG-T    A-CGT-
\end{verbatim}
Note that last four  alignments are equivalent -- they only differ in the order
of indels in sequences. All symbols from both $X$ and $Y$ can be aligned to gap
or to some symbol in other sequence. All symbols from both sequences can be
annotated by $I$ or $M$. Therefore such annotation does not give us much
information (but still we can choose the most probable explanation from all
candidate alignments/annotations but it miss the sense).
For this reason the definition of annotation function for the pHMM
should be following:

\todo{ujednotit annotation function, labeling function a coloring function}
\begin{definition}
Let $H=(\Sigma,V,I,d,e,a)$ be an HMM and $C=\{c_0,c_1,\dots,c_{l-1}\}$ be the
finite set of labels (or colors). The \firstUseOf{coloring function}
$\lambda:V^*\to C^*$ is function that satisfies following properties: 
\begin{enumerate}
\item $\lambda(v)\in C$ for all $v\in V$.
\item $\lambda(xy) = \lambda(x)\lambda(y)$ for all $x,y\in V^*$.
\item If $\lambda(u)=\lambda(v)$ then $d^x_u=d^x_v$ and $d^y_u=d^y_v$ for all
$u,v\in V$.
\end{enumerate}
Let $X$ and $Y$ be sequences that were generated by state path $\pi$. Then
\firstUseOf{pairwise annotation} $\Lambda$ of sequences $X$ and $Y$ is 
$\Lambda=\lambda(\pi)$.
\end{definition}

\todo{mozno to nie je najstastnejsia definicia}
Note that this definition is almost same as the definition
\ref{DEFINITION:ANNOTATION} with exception of using pHMMs and with added 
third condition on coloring function. This condition ensures that with
annotation we will not lose the information from state duration $d$ and we will
be able to derive unique alignment from annotation and uniquely assign
annotations to symbols of input sequences.

Similarly as with the HMMs, sequences $X$ and $Y$ and pHMM $H$ define
probability distribution of pairwise annotations $\Lambda$
\[\prob{\Lambda\mid X,Y,H} = \sum_{\pi,\lambda(\pi)=\Lambda}\prob{\pi\mid
X,Y,H}\]

Similarly we can define gain function as any function $f:C^*\times C^*\to
\mathbb{R}$. In highest expected gain problem for pHMM we search for
the pairwise annotation $\Lambda$ of input sequences $X$ and $Y$ that maximizes
expected gain:
\[\Lambda = \arg\max_{\Lambda} E_{\Lambda'\mid X,Y,H}[f(\Lambda',\Lambda)]
=  \sum_{\Lambda'}f(\Lambda',\Lambda)\prob{\Lambda'\mid X,Y,H}\]

Similarly as with HMM, finding the most probable pairwise annotation, which is equivalent
to finding the most probable state path if $\lambda$ is identity function,
is equivalent to fining pairwise annotation with highest expected gain with
function
\[f_{ID}(\Lambda',\Lambda) = \begin{cases}
1 & \text{if $\Lambda'=\Lambda$}\\
0 & \text{otherwise}
\end{cases}\]
In this case, the expected gain of an pairwise annotation $\Lambda$ is
equivalent to the probability of $\Lambda$ and therefore by maximizing the
expected gain we also maximize the probability $\prob{\Lambda\mid X,Y,H}$.

By extending this concept we want to develop gain functions that will


%Once we have an pairwise annotation, we can define the highest expected gain
%problem for
%pair hidden Markov models. 
%Skumanie jednorozmernych dekodovacich funckii

%Skumanie dvojrozmernych dekodovacich funkcii


%Traditionally used algorithm for decoding \abbreviation{Hidden Markov
%Models}{HMMs} is Viterbi \todo{Aj tak si myslim, ze cele toto by som asi mal
%presunut o kapitoly vyssie}
%algorithm. \abbreviation{Viterbi algorithm}{VA} finds the most probable state path, which is
%state path $\pi$ that maximizes $\Pr\left(\pi \mid X\right)$. Other commonly
%used criteria, the \abbreviation{Posterior Decoding}{PD}, is maximizing the sum of the posterior
%probability for every state in state sequence independently. Therefore PD finds
%the state path $\pi$ that maximizes $\sum_{i=1}^{|\pi|} \Pr\left(\pi_i \mid
%X\right)$ \todo{Trosku upravit oznacenia vo vzorcoch}. Both VA and PD can be
%computed in effectively in $O(NM)$ time and memory, where $N$ is the
%length of the sequence and $M$ is the size of the model (number of states plus
%number of transitions). There are also couple of tricks how to lower the memory
%requirements of such algorithms or even exploit repetition of the sequences to
%decrease running time.
%
%In some practical HMMs some states have same role (i.e. they encode same thing)
%\todo{chcelo by to predtym poriadne zaviest co su anotacie a tak -- este v
%prehlade}
%but for some reasons can not be merged into same states. Therefore does not have
%to distinguish between state paths, that differs only in those states. In
%general, we can assign to each state annotation symbol (or "color") and we want
%to account into probability of some state path $\pi$ probabilities of all state
%spaces, that has same sequence of annotation symbols. Therefore we are
%maximizing following criteria
%\[\sum_{\pi'\textrm{ has same annotation as }\pi}\Pr\left( \pi' \mid X \right)\]
%
%However, maximizing such criteria is NP-hard \cite{} and even if we fix the
%HMM \cite{Brejova2005}. Since maximizing this criteria is not tractable, we have
%to use some heuristics that can possibly find only local maxima. It possible to
%find HMM, for which VA finds annotation that has exponentially smaller
%probability than the most probable annotation. Similarly, since PD maximizes
%each state in state path independently, it can reconstruct state path/annotation
%that has zero probability.
%
%We can generalize all those into framework in which we will define gain function
%and where we are trying to find annotation that has highest possible gain with
%respect to correct annotation. Since we do not know the correct annotation, we
%are treating the correct annotation as an random variable with distribution
%$\Pr\left(\cdot\mid X\right)$. We can express all decoding methods that are
%described above in terms of gain functions and therefore finding the annotation
%with the highest expected gain (if gain function is computable function) is
%NP-hard. But some HMM's and gain functions we can find annotation with the
%maximum expected gain in polynomial time. 

%-- proste ze chceme sa venovat takym veciam, ze ktore funkcie sa daju pocitat
%efentivne a ktore nie. A podobne.

%Gain functions described above are not the only ones that can be in practice
%used. There are several ??

%+ extend those measures to pair-hidden Markov Models


%nieco o mierach -- vo vseobecnosti niektore miery su NP-tazke, 
%vo vseobecnosti to moze byt az nerozhodnutelne -- redukcia na PKP, alebo nieco
%take.
%
%Rozne miery su vhodne pre rozne situacie -- niekedy chceme len lokalne
%vlastnosti,  niekedy globalne, optimalizujeme rozne funkcie, nie vsetky
%vlastnosti su zachytene modelom (otazka: preco potom radsej nechceme zmenit
%model?) 

\subsection{Improving existing algorithms}

Recently it has been shown, that traditional algorithms for working with HMMs
can be improved by various techniques. In section \ref{} we discussed several
approaches how to improve this algorithms by checkpointing \cite{}, compression
\cite{} or ``garbage collection'' \cite{}. There is still space for improvements
and analysis of improvements of basic algorithm for HMMs.


For example we can reduce the memory
footprint of the Viterbi algorithm by discarding 

Similar approach is used in Treeterbi algorithm. There is also variant of this
technique, that improve memory footprint of k-best algorithm. This approaches
works in practice, but 

However analysis of such algorithm consists of 

+ extend to pHMM

%* Vyut niektore teoreticke vlastnosti na zlepsenie algoritmov -- pamatova
%narocnost,

%Ako sa daju vylepsiet terajsie algoritmy -- cez mixing time -- napriklad
%ak vypocitame mixing time pre model -- definovat sa na napriklad tromi sposobmi
%-- najhorsia sekvencia, nahodna (uniformna) alebo nahodna z modelu.
%Neda sa to robit normalne, lebo nevieme ako vyzera sekvencia. Ale vieme robit
%napriklad normalizovany forward -- teda ako velmi sa budu lisit tieto sekvencie
%aj zafixujeme sekvenciu -- alebo predpokladame ze je nahodna.

%Definovane rovnako ako mixing time -- rozdiel distribucii --

%Takto vieme spravit trenovanie rychlejsie -- jednak ho vieme 
%lahsie distribuovat, dvak vieme znizit pamatovu narocnost. Vieme nieco spravit
%napriklad pre Viterbiho? Da sa urcit nieco z modelu? 



\label{LastPage}
