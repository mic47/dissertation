\chapter{PhD project}
%Poznamka pre seba -- toto chcem asi kratsie ako som to zacal pisat. Celkovo by
%o mohlo mat tak par stran

We want to study decoding algorithms for hidden Markov models and pair hidden
Markov models (and their variants) and their application in comparative analysis
of biological sequences. This can be divided into three subjects
\begin{itemize}
\item We want to study different decoding algorithms (expressed using the
highest expected gain framework) for HMMs and pHMMs.

\item We want to study effective implementations of existing algorithm for HMMs
and pHMM, analyze their expected computational complexity and propose new
improvements.

\item Study application of the pHMM to improve existing pairwise alignment
tools. This includes developing new decoding algorithms and pHMMs that take into
account structures of sequences.
\end{itemize}
In following sections we describe problems we want to solve and propose several
approach how to solve those problems.


\section{Hidden Markov Models}

\subsection{Measures for decoding algorithms}

%Predpoklady co som uz vysvetlil:
% HMM, Viterbi, Posterior, most probable annotation, balls, herd, -- gain
% functions

In section \ref{SECTION:HEG} we have discussed the highest expected gain
framework. This framework generalizes decoding methods for HMMs. Recall that
decoding method is the method of obtaining an annotation of a sequence from the
sequence and the model. At first we briefly review the highest expected reward
framework.

Hidden Markov models defined probability distribution over state paths $\pi$
given sequence $X$ and model $H$: $\prob{\pi\mid X,H}$. Given annotation
function $\lambda$, HMMs also defines probability distribution of annotations
$\Lambda=\lambda(\pi)$ given sequence $X$: $\prob{\Lambda\mid X,H}$.  The state
path (or an annotation) that generated a sequence $X$ is hidden.
We will denote by
the \firstUseOf{correct state path} of sequence $X$ the state path that
generated $X$ and the \firstUseOf{correct annotation} is annotation of the
correct state path. Our goal is to find the correct annotation/state path of
given sequence $X$ or at least the good approximation of it.

This is usually done by finding the most probable state path and taking it's
annotation. However, in case of HMMs that have multiple path problem, 
the most probable state path does not have to correspond to the most probable
annotation. Additionally, even the most probable annotation does not have to
give us useful approximation of the correct annotation. Therefore there is need
for developing new decoding algorithms tailored for specific application
domains.

%We have shown earlier that the most probable annotation can be different from 
%the most probable state path, but searching for the most probable annotation is
%NP-hard \cite{Lyngso2002,Brejova2007mpa}. In some cases the most probable
%annotation does not have to be the thing we should look for.

\todo{Nesedi poradie argumentov, skontroluj to v celej minimovke!}
Since the correct annotation/state path is unknown, we treat correct annotation
as a random variable with distribution $\prob{\Lambda\mid X,H}$. We define gain
function $f:\bar{\Lambda}\times\bar{\Lambda}\to \mathbb{R}$.
$f(\Lambda_1,\Lambda_2)$ is the ``measure'' of an approximation of $\Lambda_2$
by $\Lambda_1$. This function is not mathematical measure, it can be arbitrary
function. Function $f$ is application specific, it should reflect the features
that are we interested in.

After defining function $f$, we seek the for the annotation $\Lambda$ with the highest expected gain:
\[\Lambda = \arg\max_{\Lambda} 
E_{\Lambda_X\mid X,H}[f(\Lambda_X,\Lambda)] =
\sum_{\Lambda_X}f(\Lambda_X,\Lambda)\Pr\left(\Lambda_X\mid X,H\right)
\]
We have shown that problem of finding the annotation with the highest expected
gain given sequence $X$ and annotation function $f$ is NP-hard. However, some gain
functions can be decoded in polynomial time (for example gain functions for the
Viterbi algorithm or gain function for  the Posterior decoding).

%There is a big gap between the time complexity of the Viterbi algorithm and the
%NP-hard problems. While decoding some functions gan be 
We want to continue studying gain functions for the virus recombination
domain, we want focus on extension of the highest expected gain framework to
pair hidden Markov models and develop gain functions (and decoding algorithms
that maximize such functions) tailored to the pairwise alignment problem. 

On possible approach how to extend the highest expected gain framework to pHMMs
is following.  We need to properly define annotations for the pHMM. In general,
pHMM are used to annotation and comparison of two sequences at once. In case of
comparative gene finding, pHMM finds genes in both sequences and actual
alignment was not main issue. In case of actual alignments, we were not so much
interested in the annotation of the sequences, but only in relation between the
symbols of the sequences. State path of pHMM uniquely assigns annotations to the
input sequences and also defines unique alignment.  Therefore we could define
annotations in same way as for HMMs. If we do so, it will create following
problem.

Let $H$ be simple pHMM from figure \ref{FIGURE:SIMPLEPHMM} (simple three state
pHMM for sequence alignment). Let $C=\{M,I\}$ be the set of labels and 
$\lambda$ be annotation function, such that $\lambda(M)=M, \lambda(I_X)=I$ and
$\lambda(I_Y)=I$.   For example let $X=Y=ACGT$ and $\Lambda = IIMMII$.
In this case it is not possible to assign annotation symbols
to input sequences $X$ and $Y$ and even it is not possible to assign unique
annotation.
Then
following $6$ alignments are possible:
\begin{verbatim}
Annotation:    IIMMII    IIMMII    IIMMII    IIMMII    IIMMII    IIMMII
X:             --ACGT    ACGT--    A-CGT-    A-CG-T    -ACGT-    -ACG-T
Y:             ACGT--    --ACGT    -ACG-T    -ACGT-    A-CG-T    A-CGT-
\end{verbatim}
And following three annotations of $X$ and $Y$ are possible: $MMII,IMMI,IIMM$.

Note that last four  alignments are equivalent -- they only differ in the order
of indels in sequences. All symbols from both $X$ and $Y$ can be aligned to gap
or to some symbol in other sequence. All symbols from both sequences can be
annotated by $I$ or $M$. Therefore such annotation does not give us much
information (but still we can choose the most probable explanation from all
candidate alignments/annotations but it miss the sense).
For this reason the definition of annotation function for the pHMM
should be following:

\todo{ujednotit annotation function, labeling function a coloring function}
\begin{definition}
Let $H=(\Sigma,V,I,d,e,a)$ be an HMM and $C=\{c_0,c_1,\dots,c_{l-1}\}$ be the
finite set of labels (or colors). The \firstUseOf{coloring function}
$\lambda:V^*\to C^*$ is function that satisfies following properties: 
\begin{enumerate}
\item $\lambda(v)\in C$ for all $v\in V$.
\item $\lambda(xy) = \lambda(x)\lambda(y)$ for all $x,y\in V^*$.
\item If $\lambda(u)=\lambda(v)$ then $d^x_u=d^x_v$ and $d^y_u=d^y_v$ for all
$u,v\in V$.
\end{enumerate}
Let $X$ and $Y$ be sequences that were generated by state path $\pi$. Then
\firstUseOf{pairwise annotation} $\Lambda$ of sequences $X$ and $Y$ is 
$\Lambda=\lambda(\pi)$.
\end{definition}

\todo{mozno to nie je najstastnejsia definicia}
Note that this definition is almost same as the definition
\ref{DEFINITION:ANNOTATION} (definition of an annotation for HMMs) with exception of using pHMMs and with added 
third condition on coloring function. This condition ensures that 
annotation keep information about state duration $d$ and we will
be able to derive unique alignment from annotation and uniquely assign
colors to symbols of input sequences.

Similarly as with the HMMs, sequences $X$ and $Y$ and pHMM $H$ define
probability distribution of pairwise annotations $\Lambda$
\[\prob{\Lambda\mid X,Y,H} = \sum_{\pi,\lambda(\pi)=\Lambda}\prob{\pi\mid
X,Y,H}\]

Similarly we can define gain function as any function $f:C^*\times C^*\to
\mathbb{R}$. In highest expected gain problem for pHMM we search for
the pairwise annotation $\Lambda$ of input sequences $X$ and $Y$ that maximizes
expected gain:
\[\Lambda = \arg\max_{\Lambda} E_{\Lambda'\mid X,Y,H}[f(\Lambda',\Lambda)]
=  \sum_{\Lambda'}f(\Lambda',\Lambda)\prob{\Lambda'\mid X,Y,H}\]

Similarly as with HMM, finding the most probable pairwise annotation, which is equivalent
to finding the most probable state path if $\lambda$ is identity function,
is equivalent to fining pairwise annotation with highest expected gain with
function
\[f_{ID}(\Lambda',\Lambda) = \begin{cases}
1 & \text{if $\Lambda'=\Lambda$}\\
0 & \text{otherwise}
\end{cases}\]
In this case, the expected gain of an pairwise annotation $\Lambda$ is
equivalent to the probability of $\Lambda$ and therefore by maximizing the
expected gain we also maximize the probability $\prob{\Lambda\mid X,Y,H}$.

By extending this concept we want to develop gain functions that will


%Once we have an pairwise annotation, we can define the highest expected gain
%problem for
%pair hidden Markov models. 
%Skumanie jednorozmernych dekodovacich funckii

%Skumanie dvojrozmernych dekodovacich funkcii



\subsection{Improving and Analysing Existing Algorithms}

It has been shown that traditional algorithms for working with HMMs can be
improved by various techniques. In section \ref{} we discussed several
approaches how to improve this algorithms by checkpointing \cite{}, compression
\cite{} or by discarding computed values that are found unnecessary \cite{}.
There is space improvements and analysis of improvements of basic algorithm for
HMMs.

In chapter \ref{} we have discuss the  On-line Viterbi algorithm. Instead of
matrix of backlinks $B$ it stores compressed forest that composed only from such 
pairs (state,position) that could be in the most probable state path.
Original paper \cite{} contains analysis of this algorithm for the two-state
symmetric HMM. With such HMM, the expected memory requirements is $O(h\log n)$
where $h$ is a constant specific for HMMs. Šrámek {\it et al.} conducted
experimental evaluation of the expected memory complexity with conjecture that
expected memory complexity of such algorithm is polylogarithmic in the length of
the sequence. However, analytical result for general HMMs are still missing.

Our goal is to characterize classes of HMMs for which  the expected memory
complexity of the On-line algorithm will be logarithmic or polylogarithmic in
the length of the input sequence and linear in the size of the state space
under the assumptions that input sequence is sampled from the model or that
symbols of the sequence were sampled independently from certain (for example
uniform) distribution.

%Da sa studovat aj worstcase, ale to nepisem

Our goal is to characterize classes of HMMs, on which average memory complexity
of the On-line Viterbi algorithm is logarithmic or polylogarithmic. Average
memory complexity is of the On-line Viterbi algorithm is computed either on the
random sequence or if the sequence comes from the model. Interesting will be
also to study worst case complexity for certain classes of HMMs.

One possible approach how to solve this problem is to search ``synchronizing
sequences''. Sequence $s\in\Sigma^*$ is synchronizing sequence if there exists
state $v$ and position $p$ in the sequence $s$, such that for all sequences
$x\in \Sigma^*$ the most probable state path $\pi$ of the sequence $xs$
$\pi_{p+|x|}=v$. Intuitively, synchronizing sequence force the Viterbi algorithm
to pass through certain state and therefore for any sequence $y\in \Sigma^*$,
the most probable state paths $\pi_{xsy}$ of sequence $xsy$ has same prefix
$\pi_{xsy}[:|x|]$. Therefore after passing sequence $s$ the On-line Viterbi
algorithm will print (and discard) the most probable state path for prefix $x$
of $xs$. 

If HMM has synchronizing sequences (HMM has synchronizing sequence if there is
non-zero probability that HMM will generate sequence that contains synchronizing
sequence starting in any state) and $f(n)$ is the expected maximal distance
between two synchronizing strings in sequence of length $n$ (sampled from
distribution $D$), then the On-line Viterbi algorithm will have expected memory
complexity $O(f(n)m)$ where $m$ is the number of states under the assumption
that input sequence was sampled from distribution $D$.


\subsubsection{Mixing Times}

We also propose following approach how to improve memory complexity of the
Posterior decoding. From HMM $H$ we compute mixing time
$t_{mix}(\epsilon)$ (we describe mixing time later).  We divide input sequence
$X$ into set of $l$ overlapping subsequences
$x_i=X[it_{mix}(\epsilon):(i+2+k)t_{mix}(\epsilon)]$ where $k$ is at least $1$
After division we run posterior decodings in all sequences
independently, obtaining state paths $\pi_i$ ($\pi_i$ was decoded from $x_i$). If $m$ is the number of states of $H$ then this will
take $O(nm^2)$ time since the total length of sequences $x_i$ is less than $3n$.
We construct the optimal state path $\pi$ by concatenation of the subsequences
of $\pi$ that are not in overlaps:
\[\pi = \pi_0[:(1+k)T] \pi_1[T:(k+1)T] \pi_2[T:(k+1)T] \dots
\pi_{l-2}[T:(k+1)T] \pi_{l-1}[T:]\] where $T=t_{mix}[\epsilon]$. We claim that
if $T$ is sufficiently high and input sequence $X$ was sampled from $H$, then
with hight probability the reconstructed state path $\pi$ is the state path that
maximized objective criteria of the Posterior decoding.
Time complexity of this algorithm is $O(nm^2)$ and memory complexity is
$O(n+T(k+2)m)$ which can be considerable improvement over $O(\sqrt n m)$
algorithm since $T$ is constant specific for HMM. However in practice $T$ can be
too high.

\def\tmix{t_{mix}(\epsilon)}

$\tmix$ is the mixing time. We can define it in following way:

\[\tmix = \min_{i}\left\{ 
\left(\max_{\mu,\nu} || \prob{\pi[i] = \cdot\mid X,\nu,H} - \prob{\pi[i] = \cdot\mid
X,\mu,H}||_{TV}\right)\leq \epsilon
\right\}
\] 
\todo{koniec tohoto odstavca sa mi nepaci}
where $\mu$ and $\nu$ are initial distribution of HMM $H$ and $X$ is a sequence
sampled from $H$ and $||\cdot||_{TV}$ is the total variation distance.
Intuitively, after $\tmix$ steps the initial distribution does not matter, since
the total variation of the distribution of forward variables $F[\tmix+i,\cdot]$
from the actual ones will be less than $\epsilon$. Same holds for the backward
variables $B$ and therefore the posterior probabilities decoded in the middle of
subsequence $x_i$ will be almost same if we were computing PD on whole sequence
$X$.

Mixing times are defined for Markov chains \cite{Levin2006}. Markov chains are
HMMs without emissions: the state path is observed, not hidden. We want to
extend the Mixing times for HMMs and find way how to compute $\tmix$ for given
HMM.




\section{Pairwise sequence alignment}

Besides theoretical outputs from our research we want to study alignments  and
the 

\label{LastPage}

%Traditionally used algorithm for decoding \abbreviation{Hidden Markov
%Models}{HMMs} is Viterbi \todo{Aj tak si myslim, ze cele toto by som asi mal
%presunut o kapitoly vyssie}
%algorithm. \abbreviation{Viterbi algorithm}{VA} finds the most probable state path, which is
%state path $\pi$ that maximizes $\Pr\left(\pi \mid X\right)$. Other commonly
%used criteria, the \abbreviation{Posterior Decoding}{PD}, is maximizing the sum of the posterior
%probability for every state in state sequence independently. Therefore PD finds
%the state path $\pi$ that maximizes $\sum_{i=1}^{|\pi|} \Pr\left(\pi_i \mid
%X\right)$ \todo{Trosku upravit oznacenia vo vzorcoch}. Both VA and PD can be
%computed in effectively in $O(NM)$ time and memory, where $N$ is the
%length of the sequence and $M$ is the size of the model (number of states plus
%number of transitions). There are also couple of tricks how to lower the memory
%requirements of such algorithms or even exploit repetition of the sequences to
%decrease running time.
%
%In some practical HMMs some states have same role (i.e. they encode same thing)
%\todo{chcelo by to predtym poriadne zaviest co su anotacie a tak -- este v
%prehlade}
%but for some reasons can not be merged into same states. Therefore does not have
%to distinguish between state paths, that differs only in those states. In
%general, we can assign to each state annotation symbol (or "color") and we want
%to account into probability of some state path $\pi$ probabilities of all state
%spaces, that has same sequence of annotation symbols. Therefore we are
%maximizing following criteria
%\[\sum_{\pi'\textrm{ has same annotation as }\pi}\Pr\left( \pi' \mid X \right)\]
%
%However, maximizing such criteria is NP-hard \cite{} and even if we fix the
%HMM \cite{Brejova2005}. Since maximizing this criteria is not tractable, we have
%to use some heuristics that can possibly find only local maxima. It possible to
%find HMM, for which VA finds annotation that has exponentially smaller
%probability than the most probable annotation. Similarly, since PD maximizes
%each state in state path independently, it can reconstruct state path/annotation
%that has zero probability.
%
%We can generalize all those into framework in which we will define gain function
%and where we are trying to find annotation that has highest possible gain with
%respect to correct annotation. Since we do not know the correct annotation, we
%are treating the correct annotation as an random variable with distribution
%$\Pr\left(\cdot\mid X\right)$. We can express all decoding methods that are
%described above in terms of gain functions and therefore finding the annotation
%with the highest expected gain (if gain function is computable function) is
%NP-hard. But some HMM's and gain functions we can find annotation with the
%maximum expected gain in polynomial time. 

%-- proste ze chceme sa venovat takym veciam, ze ktore funkcie sa daju pocitat
%efentivne a ktore nie. A podobne.

%Gain functions described above are not the only ones that can be in practice
%used. There are several ??

%+ extend those measures to pair-hidden Markov Models


%nieco o mierach -- vo vseobecnosti niektore miery su NP-tazke, 
%vo vseobecnosti to moze byt az nerozhodnutelne -- redukcia na PKP, alebo nieco
%take.
%
%Rozne miery su vhodne pre rozne situacie -- niekedy chceme len lokalne
%vlastnosti,  niekedy globalne, optimalizujeme rozne funkcie, nie vsetky
%vlastnosti su zachytene modelom (otazka: preco potom radsej nechceme zmenit
%model?) 

%Hidden Markov models are extensions Markov chains \ref{Levin2006}. Markov chain 
%is the sequence of random variables $X_0,X_1,\dots$ with property that $X_i$
%depends only on $X_{i-1}$. This property is called Markov property:
%\[\prob{X_i\mid X_0,X_1,\dots,X_{i-1}}=\prob{X_i\mid X_{i-1}}\] If state of
%Markov chains if finite, then every chain can be characterized by the transition
%matrix $P[i,j]=\prob{X_k=j\mid X_{k-1}=i}$ (this transition matrix is same as
%for the HMMs). Markov chain is \firstUseOf{ergodic} if there exists $n_0$ such
%that for all $n>n_0$ $P^n[i,j]>0$. Ergodic Markov chain has stationary
%distribution $p$ with property, that $\lim_{n\to \infty}$
%
%Mixing time is the probability that 
%
%
%We want to study 
%We would like to fill this gap and prove that for certain classes of HMMs,
%the average memory complexity is logarithmic or polylogarithmic in the length of
%the input sequence. Since there are 

%that for large HMM
%foe gene finding the average memory requirements were polylogarithmic.

%Memory requirements of uncompressed tree are same
%as for the compressed tree: because in compressed version we still remember
%the sequences of states on the edges. Compressed version reduces memory
%requirements only by constant factor. 



%For example we can reduce the memory
%footprint of the Viterbi algorithm by discarding 

%Similar approach is used in Treeterbi algorithm. There is also variant of this
%technique, that improve memory footprint of k-best algorithm. This approaches
%works in practice, but 

%However analysis of such algorithm consists of 

%+ extend to pHMM

%* Vyut niektore teoreticke vlastnosti na zlepsenie algoritmov -- pamatova
%narocnost,

%Ako sa daju vylepsiet terajsie algoritmy -- cez mixing time -- napriklad
%ak vypocitame mixing time pre model -- definovat sa na napriklad tromi sposobmi
%-- najhorsia sekvencia, nahodna (uniformna) alebo nahodna z modelu.
%Neda sa to robit normalne, lebo nevieme ako vyzera sekvencia. Ale vieme robit
%napriklad normalizovany forward -- teda ako velmi sa budu lisit tieto sekvencie
%aj zafixujeme sekvenciu -- alebo predpokladame ze je nahodna.

%Definovane rovnako ako mixing time -- rozdiel distribucii --

%Takto vieme spravit trenovanie rychlejsie -- jednak ho vieme 
%lahsie distribuovat, dvak vieme znizit pamatovu narocnost. Vieme nieco spravit
%napriklad pre Viterbiho? Da sa urcit nieco z modelu? 

