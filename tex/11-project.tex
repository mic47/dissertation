\chapter{PhD project}
Poznamka pre seba -- toto chcem asi kratsie ako som to zacal pisat. Celkovo by
to mohlo mat tak par stran

\section{Hidden Markov Models}

\subsection{Measures for decoding algorithms}

%Predpoklady co som uz vysvetlil:
% HMM, Viterbi, Posterior, most probable annotation, balls, herd, -- gain
% functions


Traditionally used algorithm for decoding \abbreviation{Hidden Markov
Models}{HMMs} is Viterbi \todo{Aj tak si myslim, ze cele toto by som asi mal
presunut o kapitoly vyssie}
algorithm. \abbreviation{Viterbi algorithm}{VA} finds the most probable state path, which is
state path $\pi$ that maximizes $\Pr\left(\pi \mid X\right)$. Other commonly
used criteria, the \abbreviation{Posterior Decoding}{PD}, is maximizing the sum of the posterior
probability for every state in state sequence independently. Therefore PD finds
the state path $\pi$ that maximizes $\sum_{i=1}^{|\pi|} \Pr\left(\pi_i \mid
X\right)$ \todo{Trosku upravit oznacenia vo vzorcoch}. Both VA and PD can be
computed in effectively in $O(NM)$ time and memory, where $N$ is the
length of the sequence and $M$ is the size of the model (number of states plus
number of transitions). There are also couple of tricks how to lower the memory
requirements of such algorithms or even exploit repetition of the sequences to
decrease running time.

In some practical HMMs some states have same role (i.e. they encode same thing)
\todo{chcelo by to predtym poriadne zaviest co su anotacie a tak -- este v
prehlade}
but for some reasons can not be merged into same states. Therefore does not have
to distinguish between state paths, that differs only in those states. In
general, we can assign to each state annotation symbol (or "color") and we want
to account into probability of some state path $\pi$ probabilities of all state
spaces, that has same sequence of annotation symbols. Therefore we are
maximizing following criteria
\[\sum_{\pi'\textrm{ has same annotation as }\pi}\Pr\left( \pi' \mid X \right)\]

However, maximizing such criteria is NP-hard \cite{???} and even if we fix the
HMM \cite{Brejova2005}. Since maximizing this criteria is not tractable, we have
to use some heuristics that can possibly find only local maxima. It possible to
find HMM, for which VA finds annotation that has exponentially smaller
probability than the most probable annotation. Similarly, since PD maximizes
each state in state path independently, it can reconstruct state path/annotation
that has zero probability.

We can generalize all those into framework in which we will define gain function
and where we are trying to find annotation that has highest possible gain with
respect to correct annotation. Since we do not know the correct annotation, we
are treating the correct annotation as an random variable with distribution
$\Pr\left(\cdot\mid X\right)$. We can express all decoding methods that are
described above in terms of gain functions and therefore finding the annotation
with the highest expected gain (if gain function is computable function) is
NP-hard. But some HMM's and gain functions we can find annotation with the
maximum expected gain in polynomial time. 

-- proste ze chceme sa venovat takym veciam, ze ktore funkcie sa daju pocitat
efentivne a ktore nie. A podobne.

Gain functions described above are not the only ones that can be in practice
used. There are several ??

+ extend those measures to pair-hidden Markov Models


%nieco o mierach -- vo vseobecnosti niektore miery su NP-tazke, 
%vo vseobecnosti to moze byt az nerozhodnutelne -- redukcia na PKP, alebo nieco
%take.
%
%Rozne miery su vhodne pre rozne situacie -- niekedy chceme len lokalne
%vlastnosti,  niekedy globalne, optimalizujeme rozne funkcie, nie vsetky
%vlastnosti su zachytene modelom (otazka: preco potom radsej nechceme zmenit
%model?) 

\subsection{Improving existing algorithms}

Recently it has been shown, that traditional algorithms for working with HMMs
can be improved by various techniques. For example we can reduce the memory
footprint of the Viterbi algorithm by discarding 

Similar approach is used in Treeterbi algorithm. There is also variant of this
technique, that improve memory footprint of k-best algorithm. This approaches
works in practice, but 

However analysis of such algorithm consists of 

+ extend to pHMM

%* Vyut niektore teoreticke vlastnosti na zlepsenie algoritmov -- pamatova
%narocnost,

%Ako sa daju vylepsiet terajsie algoritmy -- cez mixing time -- napriklad
%ak vypocitame mixing time pre model -- definovat sa na napriklad tromi sposobmi
%-- najhorsia sekvencia, nahodna (uniformna) alebo nahodna z modelu.
%Neda sa to robit normalne, lebo nevieme ako vyzera sekvencia. Ale vieme robit
%napriklad normalizovany forward -- teda ako velmi sa budu lisit tieto sekvencie
%aj zafixujeme sekvenciu -- alebo predpokladame ze je nahodna.

%Definovane rovnako ako mixing time -- rozdiel distribucii --

%Takto vieme spravit trenovanie rychlejsie -- jednak ho vieme 
%lahsie distribuovat, dvak vieme znizit pamatovu narocnost. Vieme nieco spravit
%napriklad pre Viterbiho? Da sa urcit nieco z modelu? 

\section{Pairwise sequence alignment}


\label{LastPage}
