\chapter{Introduction}


In this thesis we focus on the computational methods for comparative analysis
and annotation of biological sequences. We propose new methods that produce
alignments and annotations with higher quality. We also study the computational
complexity of the decoding algorithms and give proofs of NP-hardness for some
decoding methods. 

New technologies are giving us the ability to produce more and more data, the
automated analysis of such data is necessary and therefore it is important to
develop such tools. We will study both sequence annotation problem and the
sequence alignment problems. \todo{BLABLABLA}

Now we try to explain the sequence annotation and the sequence alignment from
the computational scientists perspective. We work with generative probabilistic
models, the \abbreviation{hidden Markov Models}{HMM} and their variants. In
general, HMMs are state machines that generates sequence (string) along the
sequence of states (called state path) that was used for generating the
sequence. Since HMM is an probabilistic model, it also defines the probability
of the sequences and the state paths. The state path contain the information
about the structure of the generated sequence. In practice we are given the
generated sequence and the state path is hidden. In sequence annotation, we
group the states of HMM into several classes (called labels, or colors) and the
state path into sequence of labels abstracting from the specific details of the
generation process. Advantage of using the probabilistic models is that apart
from the distribution of the sequences and state paths, we also obtain the
distribution of the annotations. The goal of the decoding algorithm is to
reverse the generation process and obtain the original state path or at least
the annotation. 

When using HMMs for annotation of the biological sequences, we construct HMM in
a way, that the structure of states corresponds to the biological features we
are interested in (feature is usually the biological function of the sequence).
Then we use the decoding algorithm to obtain the state path or an annotation in
such a way, that the result is as close as possible to the true state path or
an annotation. Traditionally used decoding algorithm is the Viterbi algorithm
\cite{}, but other optimization criteria can be used to obtain more
accurate\footnote{Where an accuracy is a some measure that depends on the
application domain.} results \cite{}. In Chapter \ref{} we will study the
computational complexity of some decoding criteria and show NP-hardness for
obtaining the optimal alignments using these criteria. Namely we study the most
probable footprint problem, the problem of the most probable set problem and
the most probable restriction problem. All these problems are parts of the
decoding criteria for the HMMs.

The sequence alignment is the fundamental problem in the computational biology,
and many methods, like comparative analysis methods relies on the sequence
alignment and biases in the underlying alignment introduce artefacts in the
results of these methods. In Chapter \ref{} we propose method for sequence
alignment that reduces the error rate of the alignment over the traditional
methods. In the sequence alignment problem, our goal is to compare the
biological sequences and find related parts of the sequences (what doest the
term related sequences means will be explained in Section \ref{}). This is
usually done by constructing the sequence alignment, which is the
representation of the comparison of two sequences; sequence alignment is
obtained by adding gap symbols ($-$) into the sequences so that they have the
same length and that the related parts of the sequences are in the same columns
(if each sequence is in its own row). There are many ways to insert gaps into
sequence to form alignment, and most of the alignments are not align related
parts of the sequences together. To distinguish between them we use scoring
schemes; the goal of a scoring scheme is to assign high score to biologically
relevant alignments and low score to others. In this thesis we will consider
scoring schemes defined using \abbreviation{pair hidden Markov models}{pHMM}.
pHMM is generative probabilistic model that generate pairs of sequences along
with their alignment (alignment is defined by a state path).  In chapter \ref{}
we introduce model that incorporates the repetitive elements that are prevalent
in the biological sequences (called tandem repeats), and propose several
decoding methods to be used with this model. Our repeat-aware method reduce the
error rate in the simulated experiments. \todo{Tento odstavec sa mi az tak
nepaci}

\begin{comment}
\begin{reformulate*}
Co vlastne chcem povedat
\begin{itemize}[itemsep=-1mm]
\item Chcem mat uvod ako aj pre niekoho kto nerozumie nicomu z biologie
\item ako aj pre niekoho kto nerozumie vela z informatiky
\item Uvod do zakladnych veci, ako dna, sekvencia, baza, mozno evolucia
\item Co vlastne v tejto praci je -- co sme spravili, co sme vylepsili a podobne
\item Ako vyzera struktura prace
\item Oznacenia?
\end{itemize}
\end{reformulate*}

\begin{reformulate*}
vieme dat pomlcky tak, aby sme maximalizovali pocet zhod na rovnakom mieste.
Napriklad pre vstupne sekcencie....

Rovnako vieme pouzit edit distance, pripadne skorovat substitucie rozne, a pridat aj medzery.
Taketo skorovacie schemy sa pouzivaju a funguju prekvapujuco dobre, ale stale obsahuju 
chyby lebo biologia je ovela komplikovanejsia. Tieto schemy vedia byt vyjadreme pomocou 
pHMM, ktore generuju dvojice sekvencii (a zaroven zarvonanie) a pri dekodovani sa
snazime tento proces reverznut. Avsak tieto casti nam umoznuju zakomponovat do
modelu dalsie evolucne operacie, cim sa spresni zarovnanie.
Rovnako mozeme vybrat vhodny dekodovaci algoritmus.

V tejto praci sme sa zamerali na tandemove repeaty, ktorych je v DNA vela a sposobuju problemy pri zarovnavani.
Navrhli sme model, ako aj dekodovacie ktoeria, ktore skvalitnili zarovnania.

Bilogicke:
Co je DNA, ako vyzera, co je evolucia -- ako to vyzera
ake operacie sa tam zhruba deju, preco pouzivame taketo skorovacie schemy.
Co su repeaty, geny a podobne.
\end{reformulate*}

Both problems studied in this proposal, annotation and alignment of biological
sequences, can be addressed by variants of \abbreviation{hidden Markov
models}{HMMs}.  Hidden Markov models are generative probabilistic models which
define a probability distribution of sequences with their annotations
\cite{Durbin1998}. An annotation is the assignments of labels to parts of the
sequence according their meaning, The labels can represent biological function,
the origin of a part of the sequence or any other information. HMMs are used for
probabilistic analysis of sequences in various fields. In our work we focus on
biological sequences. 

A decoding algorithm is an algorithm that from the input sequence $X$ and HMM
$H$ computes the annotation of sequence $X$. The goal is to recover the true
annotation of $X$, but finding such an annotation is impossible due to the
randomness in the generating process. Therefore the decoding algorithm finds
some approximation of the true annotation.  In recent years different decoding
methods were developed to improve the quality of annotations
\cite{Gross2007,Nanasi2010,Nanasi2010mgr,Truszkowski2011}.  Additionally there
are many techniques that are used to lower the computational complexity of the
decoding and training algorithms (training algorithms are algorithms that
compute the parameters of the model from a training data set).

In the field of comparative analysis of biological sequences we focus on the
sequence alignment problem. A sequence alignment is a simple data format (or
data structure) that represents the similarities in the sequences. An alignment
is created by inserting dashes ('-') into both sequences in a way that the
sequences have the same length and similar symbols are on the same positions in
both sequences. In computational biology the alignments are constructed in a way
that the homologous parts of the sequences are aligned together. Homologous
sequences are sequences that have evolved from the same part of the sequence of
the ancestral organism.  The reason for searching and studying homologous
sequences is that homologous sequences are expected to have the same or similar
functions in both organisms.  Therefore by studying homologous sequences we can
transfer knowledge about functions from one species to others.

Alignments can be created using a variant of hidden Markov model called
\abbreviation{pair hidden Markov model}{pHMM}. A pHMM  that generates three
objects: two sequences $X,Y$ and their alignment $A$.  By proper decoding method
we can obtain a good approximation of $A$ for given sequences $X$ and $Y$. Our
goal is to study decoding methods and models that are used for sequence
alignment in order to create software that will produce alignments with high
quality.

\end{comment}
\section{Biological Introduction}

%\todo{zmen nazov kapitoly}

%DNA,RNA,protein, amino-acid, residue, base, gene -- exon, conod, start codon,
%stop codon, cDNA, intron, splice site, donor, acceptor, intergenic region,
%homolog,
In this section we review several biological terms that will be needed. More
information about DNA, proteins and genes can be found in
\cite{BiologyForDummies, UnderstandingBioinformatics}.  Every cell of living
organisms contains one or several \firstUseOf{DNA} molecules. DNA is a double
stranded molecule, consisting of two long sequences of \firstUseOf{nucleotides}
(nucleotides are also referred to as bases or \firstUseOf{residues}). There are
four types of nucleotides in DNA: adenosine, cytosine, guanine and thymine
represented by letters $A,C,G$ and $T$ respectively. In RNA nucleotide $T$ is
replaced with uracil, denoted by $U$. The strands in DNA are complementary. The
nucleotides at the same position in the two strands are connected by hydrogen
bonds and are complementary: $A$ is always connected with $T$ and $C$ is always
connected with $G$. Therefore we can represent DNA molecule by a sequence over
alphabet $\{A,C,G,T\}$ since the complementary strand can be easily computed.

DNA encodes \firstUseOf{proteins}. Proteins play an important role in cell
biology since  they regulate many processes in the cell and are catalysts to
many chemical reactions. Proteins are sequences of \firstUseOf{amino acid}
molecules. There are $20$ amino acids that can be encoded in DNA. Parts of DNA
that encode proteins are called \firstUseOf{genes} (gene is ``substring'' of
DNA that will be translated into one protein). We refer to the DNA molecule/sequence of
the organism as to its \firstUseOf{genome}.

\todo{evolucia a ako to suvisi s operaciami a zarovnaniami}
According to the evolution theory, currently living organisms evolved from
single common ancestor through small changes in the genome of the organisms.
There are many types of changes, from the small scales like substitutions
(change of the nucleotide at some position), insertions and deletions of
sequences from and to the genomic sequence. Other changes includes duplications
(subsequence is copied into different part of the genome), inversions (the
subsequence is inverted), and even large genome rearrangements (the large
subsequences of the genome change their position within genome). The speciation
is event, when the new species is created. This happens mostly due to physical
separation of populations of same organism, and each population evolve
differently. We can represent the evolution of species using a binary tree.
Each leaf represent current organism, each internal vertex represents
speciation event, and root represents the common ancestor of all organisms in
the tree. The branch lengths usually corresponds to the amount of changes in
the genome, or to the time. 

We say that two parts of the biological sequences are \firstUseOf{homologous},
if they originates from the same sequence in the common ancestor of those
sequences. We distinguish between two types of a homologs: orthologs and
paralogs. The orthologs are two different sequences that evolved from the
common ancestor sequence by speciation event. The paralogs are two different
sequences that evolved from the common ancestor using by duplication event
within same organisms. In the sequence alignment problem we search for the
homologous parts of the sequences: we want to align sequences in a way that
homologous parts are together.
\todo{Nieco o identite a tak}

In the sequence alignment problem, we want to search for the related parts of the different sequences.

\section{Thesis outline}

\todo{Sem chcem napisat ake budu nase prinosy}

In the second chapter we survey the relevant literature and describe the
necessary models and algorithm for sequence alignment and sequence annotation.
We cover the hidden Markov models and their variants, various decoding
algorithms used for sequence annotation, the sequence alignment problem and the
pair hidden Markov models that are used for sequence alignment.  In the end of
the second chapter we discuss algorithmic improvements to the discussed
algorithms.

The third and fourth chapters contains our results. In the third chapter we
study the two-stage algorithms. We formalize the definition of such algorithms
and show practical example of such decoding algorithm design. Then we study the
computational complexity of three problems: the most probable set, the most
probable footprint and the most probable restriction. We prove that the first
problem is NP-hard and for the latter two we prove that they are NP-hard. \todo{
+ experiment a co sme s nim mohli dosiahnut: ze two-stage nemusi byt len 
optimalizacia, ale aj zrychlenie
}

In the fourth chapter we propose tractable method for aligning sequence with
tandem repeats. We extend this notion to the most \todo{Co sme spravili}i

\section{Notation}

In this section we summarize notation used in the following
chapters.

All sequences, members of sets, vectors, and rows and columns of matrices will
be indexed from $0$. We will use mostly right-open intervals: $I=[a,b)$ means
that $a\in I$, but $b\notin I$. 

The element at the $k$-th position (zero based) of string (or sequence) $s$ will
be written as $s[k]$. The substring $s[i]s[i+1]\dots s[j-1]$ is denoted
$s[i:j]$.  If $n$ is the length of the string $s$ then $s[:i]$ is equivalent to
$s[0:i]$ and $s[i:]$ is equivalent to $s[i:n]$.  We will use the terms sequence
and string interchangeably.

Let $M$ be a matrix. Then $M[i,j]$ is the element from the $i$-th row and $j$-th
column of $M$ (indices are zero based). Similarly as for strings, submatrix
$M[i:j,k:l]$ is a matrix consisting from the intersection of rows $i,i+1,\dots,
j-1$ with columns $k,k+1,\dots,l-1$. If $M$ is of size $n\times m$ then
$M[:i,j:]$ is equivalent to $M[0:i,j:m]$.  The term $M[i,:]$ is equivalent to
$M[i,0:m]$ which is the $i$-th row of $M$.
