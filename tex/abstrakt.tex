Zaoberáme sa dvomi dôležitými bioinformatickými problémami: anotácia sekvencií
a zarovnávanie sekvencií. V práci používame skryté Markovove modely (HMM),
dobre známe generatívne pravdepodobnostné modely.

V prvej časti študujeme anotáciu sekvencií, konkrétne dvojstupňové algoritmy a
výpočtové problémy ktoré s nimi súvisia. Ukážeme, že dvojstupňové algoritmy
vedia byť použité na zlepšenie presnosti dekodovacích algoritmov a dokážeme
NP-ťažkosť troch súvisiacich problémov: problém najpravdepodobnejšej množiny,
problém najpravdepodobnejšej reštrikcie a problém najpravdepodobnejšej stopy.

Druhá časť sa zaoberá zarovnávaním sekvencií ktoré obsahujú tandemové
opakovania. Tandemové opakovania sú  opakujúce sa časti genomických sekvencií
ktoré spôsobujú chyby v zarovnaniach. Aby sme vyriešili tento problém, vyvinuli
sme nový HMM, ktorý modeluje zarovnania obsahujúce tandemové opakovania a
skombinovali sme to s existujúcimi ako aj novými dekódovacími algoritmami. Náš
prístup sme vyhodnotili experimentálne.

V oboch problémoch sme používali dekódovacie algoritmy na zlepšenie presnosti
predikcií z HMM. Dekódovacie algoritmy sú často podceňované  a väčšina vývoja
ide do vytvárania HMM. Avšak správnym výberom dekódovacej metódy môžeme
dosiahnuť významné zlepšenie predikcií.
